\chapter{Sept.~3 --- Representations of Finite Groups}

\section{Representations of \texorpdfstring{$S_4$}{S4}}
\begin{remark}
  We will write \emph{irrep} for
  ``irreducible representation.''
\end{remark}

\begin{example}
  Consider the symmetric group
  $S_4$, with $|S_4| = 24$. The conjugacy
  classes of $S_4$ are parametrized by
  partitions of $4$: If we have a partition
  \[
    \lambda = (\lambda_1, \lambda_2, \ldots, \lambda_k), \quad
    \lambda_1 \ge \lambda_2 \ge \cdots \ge \lambda_k \ge 1,
    \quad
    \lambda_1 + \lambda_2 + \cdots + \lambda_k = 4,
  \]
  then the corresponding conjugacy class
  has cycle type $\lambda$. For
  example, the conjugacy classes are given by
  \begin{enumerate}
    \item $\lambda_1 = 4$: $[4]$;
    \item $\lambda_1 = 3$, $\lambda_2 = 1$: $[3, 1]$;
    \item $\lambda_1 = 2$, $\lambda_2 = 2$: $[2, 2]$;
    \item $\lambda_1 = 2$, $\lambda_2 = 1$, $\lambda_3 = 1$: $[2, 1, 1]$;
    \item $\lambda_1 = 1$, $\lambda_2 = 1$, $\lambda_3 = 1$, $\lambda_4 = 1$: $[1, 1, 1, 1]$.
  \end{enumerate}
  In particular, this means that
  $S_4$ has $5$ irreps. We can enumerate
  them as follows:
  \begin{enumerate}
    \item We have the $1$-dimensional representations:
      the trivial representation and the
      sign $\sign_4$.
    \item Let $S_4$ act on $\C^4$
      by permuting the basis vectors.
      The span of
      $(x, x, x, x)$ gives a $1$-dimensional
      subrepresentation, but it has
      a unique $3$-dimensional complement
      $\refl_4$.
    \item We can take a tensor
      product $\refl_4 \otimes \sign_4$,
      which is also $3$-dimensional. One
      can check that this is different
      from $\refl_4$ by looking at the
      determinant.
    \item We have found two $1$-dimensional
      and two $3$-dimensional irreps,
      which account for
      $1 + 1 + 9 + 9 = 20$ of the
      $24$ dimensions. Thus there
      is a missing $2$-dimensional
      representation.

      Note that there is a projection
      $\pi : S_4 \to S_3$ which is a
      homomorphism with kernel
      $\Z / 2\Z \times \Z / 2\Z$.
      Figure this out and find the
      last irrep as an exercise.
  \end{enumerate}
\end{example}

\begin{exercise}
  Let $G$ be a finite abelian group.
  Prove that all irreps of $G$ are
  $1$-dimensional.
\end{exercise}

\section{Characters}

\begin{definition}
  Let $G$ be a group, and let $U$ a
  finite-dimensional representation of $G$.
  The \emph{character} $\chi_U : G \to \F$
  is defined by $\chi_U(g) = \tr_U(g)$.
\end{definition}

\begin{exercise}
  Prove the following:
  \begin{enumerate}
    \item $\chi_U$ is constant on
      conjugacy classes of $G$.
    \item $\chi_{U \oplus V} = \chi_U \oplus \chi_V$.
    \item $\chi_{U \otimes V} = \chi_U \chi_V$.
  \end{enumerate}
\end{exercise}

\begin{remark}
  For the rest of this section,
  assume $G$ is finite and $\F = \C$.
  So we know every representation
  of $G$ is completely reducible.
  Denote by $\C[G]$ the algebra of
  complex-valued functions on $G$, and
  $\C[G]^G$ the subalgebra of
  functions constant on conjugacy
  classes (i.e. the $G$-invariant functions).
  Clearly the character $\chi_U$ lies
  in $\C[G]^G$ for any finite-dimensional
  representation $U$ of $G$.
\end{remark}

\begin{definition}
  Define a Hermitian scalar product
  on $\C[G]^G$ (a priori only on the
  characters) by
  \[
    (\chi_1, \chi_2)
    = \frac{1}{|G|} \sum_{g \in G}
    \chi_1(g) \overline{\chi_2(g)}.
  \]
\end{definition}

\begin{prop}
  Let $U, V$ be finite-dimensional
  representations of $G$. Then
  \[
    (\chi_U, \chi_V) = \dim \Hom_G(U, V).
  \]
\end{prop}

\begin{proof}
  We first note that
  $\chi_{U^*} = \overline{\chi}_U$. To
  see this, observe that
  since $G$ is finite, we have
  $g^n = 1$ for some $n$. In particular,
  the eigenvalues $\lambda_i(g)$ of $g$
  have $|\lambda_i(g)| = 1$. Thus
  $\lambda_i(g^{-1}) = \overline{\lambda_i(g)}$, so we see the result after
  taking traces. Another
  way to see this is the following:
  For a representation $\rho : G \to U$,
  we can make each $\rho(g)$ into a unitary
  operator as follows. Begin with a
  pairing $\langle \cdot, \cdot \rangle_0$
  on $U$ and define
  \[
    \langle v, w \rangle
    = \frac{1}{|G|} \sum_{g \in G} \langle \rho(g) v, \rho(g) w \rangle_0, \quad
    v, w \in U.
  \]
  Then $\rho(g)$ is unitary with
  respect to $\langle \cdot, \cdot \rangle$,
  and we get the result.

  Continuing, we have
  $V \otimes U^* = \Hom_\C(U, V)$, so
  $\chi_{\Hom(U, V)} = \chi_V \overline{\chi}_U$.
  Consider the averaging element
  \[
    \epsilon = |G|^{-1} \sum_{g \in G} g \in \C[G].
  \]
  This is a projector on $G$-invariants
  ($W^G$) in any representation $W$. Thus
  $\tr_W(\epsilon) = \dim W^G$.
  Applying this to $W = \Hom(U, V)$ and noting
  that $\Hom_G(U, V) = \Hom(U, V)^G$, we
  get
  \[
    \dim \Hom_G(U, V)
    = \tr_{\Hom(U, V)}(\epsilon)
    = |G|^{-1} \sum_{g \in G} \chi_{\Hom(U, V)}(g)
    = |G|^{-1} \sum_{g \in G} \chi_V(g) \overline{\chi_U(g)}
    = (\chi_V, \chi_U),
  \]
  which proves the desired claim.
\end{proof}

\begin{corollary}
  The characters of irreps form an
  orthonormal basis in $\C[G]^G$.
\end{corollary}

\begin{proof}
  Schur's lemma implies orthonormality.
  Since the number of irreps equals the
  number of conjugacy classes, the
  characters must form a basis.
\end{proof}

\section{Induced Representations}

\begin{remark}
  In this
  section, we only assume $k$ is a
  commutative ring.

  Let $H \subseteq G$, where
  $H, G$ are finite groups, let
  $kH, kG$ be the
  corresponding group algebras, and let
  $U$ be a representation of $H$.
  Treating $kG$ as a $kG$-$kH$-bimodule,
  we can construct the tensor
  product
  \[
    kG \otimes_{kH} U.
  \]
  Similarly, treating $kG$ as a
  $kH$-$kG$-bimodule, we can construct
  the representation
  \[
    \Hom_{kH}(kG, U).
  \]
  In fact, these two representations are
  isomorphic, we call it the
  \emph{induced representation}, denoted
  $\Ind_H^G U$.
\end{remark}

\begin{prop}
  There is a natural
  isomorphism
  $kG \otimes_{kH} U \cong \Hom_{kH}(kG, U)$.
\end{prop}

\begin{proof}
  First treat $kG$ as a $kH$-$kG$-bimodule,
  so we can consider $\Hom_{kH}(kG, kH)$
  since $kG$, $kH$ are both left
  $kH$-modules. So for any element
  $\varphi : kG \to kH$, we have
  \[
    \varphi(hg) = h \varphi(g), \quad
    h \in H, g \in G.
  \]
  with a left $G$-action and
  right $H$-action given by
  \[
    [g \varphi](g')
    = \varphi(g' g) \quad
    \text{and} \quad
    [\varphi h](g') = \varphi(h g').
  \]
  Note that $kG$ is a free left $kH$-module
  with basis given by the orbits of $H$.
  Show as an exercise that
  \begin{align*}
  \Hom_{kH}(kG, kH)
  \otimes_{kH} U
  &\overset{\cong}{\longrightarrow}
  \Hom_{kH}(kG, U) \\
  \alpha \otimes u
  &\longmapsto
  (x \mapsto \alpha(x) u)
  \end{align*}
  is an isomorphism. From here it suffices
  to show that
  \[
    kG \overset{\cong}{\longrightarrow}
    \Hom_{kH}(kG, kH)
  \]
  as $kG$-$kH$-bimodules. Define this map
  via $g \mapsto \varphi_g \in \Hom_{kH}(kG, kH)$,
  where
  \[
    \varphi_g(g') =
    \begin{cases}
      g' g & \text{if } g' g \in H, \\
      0 & \text{otherwise}.
    \end{cases}
  \]
  We need to show that $\varphi$ is
  $H$-equivariant, $G$-equivariant, and
  an isomorphism of $k$-modules.

  To see $H$-equivariance, note that
  $\varphi_{gh}(g')$ and $\varphi_g(g') h$
  are nonzero and equal if and only if
  $g g' \in H$. For the
  $G$-equivariance, note that
  $\varphi_{g_1 g}(g')$ and
  $[g_1 \varphi_g](g')$ are given by
  \begin{align*}
    \varphi_{g_1 g} (g')
    = g' g_1 g
    \quad &\text{if } g' g_1 g \in H, \\
    [g_1 \varphi_g](g')
    = g' g_1 g
    \quad &\text{if } g' g_1 g \in H
  \end{align*}
  and zero otherwise, so they coincide.
  To prove that $\varphi$ is an
  isomorphism of $k$-modules, we need to
  check that the $\varphi_g$ form a
  basis in $\Hom_{kH}(kG, kH)$.
  Let $g_1, \dots, g_\ell$ be representatives
  of the left $H$-orbits in $G$. Then we
  claim that the following map is an
  isomorphism of $k$-modules:
  \begin{align*}
    \Hom_{kH}(kG, kH)
    &\overset{\cong}{\longrightarrow}
    (kH)^{\oplus \ell} \\
    \varphi \mapsto \{\varphi(g_i)\}_{i = 1}^\ell.
  \end{align*}
  This follows since for any $g \in G$ and
  $i \in \{1, \dots, \ell\}$,
  there is a unique element $h \in H$
  such that $h g_i = g^{-1}$, so
  $\varphi_g$ is sent to the corresponding
  summand.
\end{proof}

\begin{corollary}[Frobenius reciprocity]
  Let $U, V$ be representations of
  $H, G$, respectively. Then
  \begin{enumerate}
    \item $\Hom_G(\Ind_H^G(U), V) \cong \Hom_H(U, V)$;
    \item $\Hom_G(V, \Ind_H^G(U)) \cong \Hom_H(V, U)$.
  \end{enumerate}
\end{corollary}

\begin{proof}
  This follows from the Tensor-Hom
  adjunction, check it as an exercise.
\end{proof}

\begin{remark}
  What really is $\Ind_H^G U$?
  Consider the set of maps (of sets)
  $G \to U$, denote it by $\Fun(G, U)$.
  The action of $G$ on itself gives
  $\Fun(G, U)$ the structure of a
  $kG$-module. Then we can define
  \[
    \Fun_H(G, U)
    = \{f \in \Fun(G, U) : f(hg) = h f(g)\}
    \subseteq \Fun(G, U),
  \]
  which we can identify with
  the induced representation
  $\Hom_{kH}(kG, U)$.
\end{remark}
