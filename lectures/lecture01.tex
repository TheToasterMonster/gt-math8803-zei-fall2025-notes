\chapter{Aug.~18 --- Historical Perspective}

\section{Origin of Representation Theory}

One motivation for representation theory is
symmetries in physics.
From a mathematical perspective, we consider
\emph{groups} and \emph{algebras} (a vector
space with a bilinear operation).
In this course, we will study two types of
groups:
\begin{enumerate}
  \item \emph{finite groups}, e.g. the symmetric group;
  \item \emph{Lie groups}, e.g. the rotation group.
\end{enumerate}

\begin{definition}
  A \emph{representation} of a group $G$
  is a homomorphism $G \to \End(V)$, where
  $V$ is some finite-dimensional vector space.
\end{definition}

The history of representation theory is as
follows:
\begin{enumerate}
    \item
    In the late 19th century, people
    were interested in \emph{crystallography},
    in particular crystallographic groups
    and their classification. There are
    related objects called \emph{Bieberbach groups} (e.g. $O(n)$ with translations, i.e.
    $\R^n \rtimes O(n)$).

    Sophus Lie discovered \emph{Lie groups}
    in his main manuscript ``Transformation
    groups.'' From Lie groups, one then derives
    \emph{Lie algebras}.

    \item
    In the early 20th century (1905),
    \emph{special relativity} was discovered,
    which involves the \emph{Lorentz group}
    $\SO(1, 3)$ (the transformations preserving
    the form $-t^2 + x^2 + y^2 + z^2$).
    This is a Lie group.

    Around the same time, E. Cartan developed
  the modern theory of \emph{semisimple Lie groups} and \emph{Lie algebras}, and
    H. Weyl studied their representations.

    \item
    In the period 1920--1930, quantum (``matrix'')
    mechanics was discovered. Here
    one has a Hilbert space $\mathcal{H}$ and
    a self-adjoint Hamiltonian (energy)
    operator $H$ on $\mathcal{H}$.
    The symmetry operator $A$ satisfies
    the commutator relation
    $[H, A] = 0$, and if we set $U = e^{iA}$,
    we have $U H U^\dagger = H$.

    \item
    After the discovery of \emph{spin} by W.
    Pauli, E. Wigner realized that spin was
    directly related to the representation
    theory of the universal cover
    $\pi : \SU(2) \to \SO(3)$.

    In the 1960s, there was a ``zoo'' of
    elementary particles. M. Gell-Mann and
    Y. Neeman realized that all of these
    can be described by representations of
    $\SU(3)$. The led to the discovery of
    \emph{quarks} and the later
    notion of grand unified theories and
    string theory in the 1970s.

    There are also connections to condensed
    matter theory and quantum information.
\end{enumerate}

This course will cover the following topics:
\begin{enumerate}
  \item basics about associative algebras
    and their representations, finite groups
    and their representations in general,
    the symmetric group and its representations,
    Young tableaux;
  \item Lie groups and Lie algebras;
  \item the structure of semisimple Lie
    algebras;
  \item representations of $\SL(n)$.
\end{enumerate}

\section{Introduction to Lie Groups and Lie Algebras}
In general, groups are complicated, whereas
algebras are less complicated. We begin with
finite groups.

\begin{definition}
  Let $G$ be a finite group and $\F$ a field.
  The \emph{group algebra} $\F G$ is
  \[
    \F G = \left\{
      \sum_g a_g g : a_g \in \F
    \right\}.
  \]
  This forms an algebra over $\F$
  with the obvious multiplication operation.
\end{definition}

\begin{example}
  Consider the rotation group, generated by the
  matrices
  \[
    R_z(\theta) =
    \begin{pmatrix}
      \cos \theta & \sin \theta & 0 \\
      -\sin \theta & \cos \theta & 0 \\
      0 & 0 & 1
    \end{pmatrix}, \quad
    R_x(\phi) =
    \begin{pmatrix}
      1 & 0 & 0 \\
      0 & \cos \phi & -\sin \phi \\
      0 & \sin \phi & \cos \phi
    \end{pmatrix}, \quad
    R_y(\psi) =
    \begin{pmatrix}
      \cos \psi & 0 & -\sin \psi \\
      0 & 1 & 0 \\
      \sin \psi & 0 & \cos \psi
    \end{pmatrix}.
  \]
  Letting $\delta$ be an infinitesimal
  value and using
  a Taylor expansion, we can write
  \begin{align*}
    R_z(\delta \theta)
    &= 1 + \delta \theta \begin{pmatrix}
      0 & 1 & 0 \\
      -1 & 0 & 0 \\
      0 & 0 & 0
    \end{pmatrix} = 1 + \delta \theta M_z, \\
    R_x(\delta \phi)
    &= 1 + \delta \phi \begin{pmatrix}
      0 & 0 & 0 \\
      0 & 0 & -1 \\
      0 & 1 & 0
    \end{pmatrix} = 1 + \delta \phi M_x, \\
    R_y(\delta \psi)
    &= 1 + \delta \psi \begin{pmatrix}
      0 & 0 & -1 \\
      0 & 0 & 0 \\
      1 & 0 & 0
    \end{pmatrix} = 1 + \delta \psi M_y.
  \end{align*}
  We can measure the commutativity of these
  matrices via
  \begin{align*}
    R_x(\delta \phi) R_y(\delta \psi)
    R_x^{-1}(\delta \phi)
    R_y^{-1}(\delta \psi)
    &= (1 + M_x \delta \phi)
    (1 + M_y \delta \psi)
    (1 - M_x \delta \phi)
    (1 - M_y \delta \psi) \\
    &= 1 + \delta \phi \delta \psi (M_x M_y - M_y M_x).
  \end{align*}
\end{example}

\begin{exercise}
  Show that $[M_x, M_y] = -M_z$.
\end{exercise}

\begin{remark}
  Thus we have a vector space spanned by
  $M_x, M_y, M_z$ with an operation
  $[\cdot, \cdot]$ satisfying the identity
  $[M_x, M_y] = -M_z$. Note that this property
  is satisfied by the cross product on $\R^3$.
  The cross product also satisfies the
  following
  \emph{Jacobi identity}:
  \[
    [A, [B, C]] = [[A, B], C] + [B, [A, C]].
  \]
  The above properties define a
  \emph{Lie algebra}.
\end{remark}

\begin{definition}
  Let $\{e_k\}$ be a basis of a Lie algebra
  and $[e_i, e_j] = \sum_k c_{ij}^k e_k$.
  The \emph{universal enveloping algebra}
  of the Lie algebra is the free associative
  algebra on $\{e_k\}$, modulo the
  relations $[e_i, e_j] = \sum_k c_{ij}^k e_k$.
\end{definition}

\begin{remark}
  One way to return to the Lie group from
  the Lie algebra is exponentiation, e.g.
  $R_z(\theta) = e^{\theta M_z}$.
\end{remark}

\section{Algebras and Modules}

Let $k$ be a commutative ring (most of the time
$k = \C$). All rings will be associative and
unital.

\begin{definition}
  A \emph{(associative and unital) $k$-algebra}
  is a unital ring $A$ with a homomorphism
  $i : k \to A$ such that
  $i(r) \cdot a = a \cdot i(r)$, i.e. the
  image of $i$ commutes with $A$.
\end{definition}

\begin{example}
  Any ring is a $\Z$-algebra.
\end{example}

\begin{definition}
  A \emph{homomorphism} of $k$-algebras is a
  $k$-linear homomorphism of unital rings.
\end{definition}

\begin{definition}
  Let $A, B$ be unital rings, and $M$ an
  abelian group. Then
  \begin{enumerate}
    \item a \emph{left $A$-module structure}
      on $M$ is a $\Z$-bilinear map
      $A \times M \to M$, associative in
      the sense that
      \[
        a_1(a_2 m) = (a_1 a_2) m, \quad
        \text{for all } a_1, a_2 \in A,\, m \in M,
      \]
      and such that $1_A m = m$ for all $m \in M$;
    \item a \emph{right $A$-module structure}
      on $M$ is a $\Z$-bilinear map
      $M \times B \to M$, associative in
      the sense that
      \[
        (m b_1) b_2 = m (b_1 b_2), \quad
        \text{for all } b_1, b_2 \in B,\, m \in M,
      \]
      and such that $m 1_B = m$ for all $m \in M$;
    \item an \emph{$A$-$B$-bimodule structure}
      on $M$ is a left $A$-module and
      right $B$-module structure on $M$,
      along with the condition that
      $(am) b = a(mb)$ for all
      $a \in A$, $b \in B$, and $m \in M$.
  \end{enumerate}
\end{definition}

\begin{remark}
In general, an $A$-module will mean a left $A$-module by default.
\end{remark}

\begin{definition}
  Let $M, N$ be left $A$-modules.
  An \emph{$A$-module homomorphism} is a map
  $\varphi : M \to N$ such that
  $\varphi(am) = a \varphi(m)$ for all
  $a \in A$ and $m \in M$.
\end{definition}

\begin{example}
  A ring $A$ is both a left/right
  $A$-module and an $A$-$A$-bimodule
  (the \emph{regular bimodule}).
\end{example}

\begin{definition}
  The \emph{direct sum}
  $\bigoplus_{i \in I} M_i$ of
  left $A$-modules $M_i$ is the collection
  of $(m_i)_{i \in I}$ with finitely
  many nonzero entries, with
  component-wise addition and
  scalar multiplication.
\end{definition}

\begin{example}
  Let $I$ be an index set. Then
  $A^{\oplus I}$ is the
  \emph{coordinate $A$-module}.
\end{example}

\begin{definition}
  A \emph{submodule} of $M$ is a nontrivial
  subgroup closed under addition and invariant
  under the action of $A$.
\end{definition}

\begin{example}
  Submodules of the regular left/right
  $A$-module are the left/right ideals of $A$.
\end{example}

\begin{definition}
  Let $M$ be a left $A$-module and
  $M_0$ a submodule of $M$. The
  \emph{quotient module} $M / M_0$
  is the set of equivalence classes
  $m + M_0$, where the action of $A$ is
  given by $a(m + M_0) = am + M_0$.
\end{definition}

\begin{lemma}
  Let $M, N$ be $A$-modules and $M_0 \subseteq M$
  a submodule. Let
  $\varphi : M \to N$ be $A$-linear such that
  $\varphi(M_0) = \{0\}$. Then there exists a
  unique $A$-linear map $\underline{\varphi} : M / M_0 \to N$
  such that $\varphi = \underline{\varphi} \circ \pi$,
  where $\pi : M \to M / M_0$ is the
  canonical projection.
\end{lemma}
