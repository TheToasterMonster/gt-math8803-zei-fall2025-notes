\chapter{Sept.~22 --- Lie Groups}

\section{Young Tableaux}

\begin{remark}
  Recall there is a one-to-one
  correspondence between partitions
  $(n_1, \dots, n_k)$ of $n$ (satisfying
  $n_1 \le \dots \le n_k$ and
  $n_1 + \dots + n_k$) and
  $\Irr(\C S_n)$. Also
  recall the \emph{Young tableau}
  for a partition, which is consists
  of $k$ rows of $n_k$
  boxes stacked on top of each other,
  where the $1$st row is at the top.
\end{remark}

\begin{definition}
  A \emph{standard Young tableau}
  is a Young tableau
  filled with numbers $\{1, \dots, n\}$
  so that they strictly increase
  from bottom to top and from left to right.
  Denote by $\SYT(n)$ the set of
  standard Young tableaux (corresponding
  to a partition of $n$).
\end{definition}

\begin{definition}
  To a Young tableau $T$, assign its
  \emph{content} as follows. Let
  $(x_i, y_i)$ be the coordinates
  of the box numbered $i$. Then the
  content of the box is $x_i - y_i$.
  The content of the tableau is
  \[
    c(T) = (x_1 - y_1, x_2 - y_2, \dots, x_n - y_n).
  \]
\end{definition}

\begin{exercise}
  Show that the map $T \mapsto cT$
  is injective.
\end{exercise}

\begin{prop}
  The map $T \mapsto c(T)$ is a bijection
  $\SYT(n) \to c{\Wt_n}$, and the shape of
  $T$ coincides with the
  partition assigned to $n$.
\end{prop}

\begin{proof}
  This is an exercise in combinatorics.
\end{proof}

\begin{definition}
  The Young tableau with numbers
  $1, \dots, n_1$ in the bottom row,
  $n_1 + 1, \dots, n_1 + n_2$ in the
  second-to-bottom row, and so on
  is called the \emph{normal Young tableau}.
\end{definition}

\begin{corollary}
  Let $\lambda$ be a Young tableau with
  $n$ boxes and $V_\lambda$ the corresponding
  $\C S_n$-module. Then there is a basis
  $\{v_T\}$ in $V_\lambda$ which is labeled
  by $\SYT(n)$ associated to $\lambda$.
  Moreover, each $v_T$ is an eigenvector
  of the Jucys-Murphy's elements such that
  the eigenvalue of $J_i$ is the content
  $x_i - y_i$ of the $i$th box.
\end{corollary}

\begin{remark}
  Take
  $(w_1, \dots, w_n) \in c{\Wt_n}$, and
  consider $(w_1, \dots, w_{n - 1})$.
  What does this mean in terms of
  $\SYT(n)$? The new tableau $T'$ is
  obtained from the original tableau $T$
  by removing the box labeled $n$.
\end{remark}

\begin{corollary}
  Let $\lambda$ be a partition of $n$
  and $V_\lambda$ the corresponding
  irrep of $\C S_n$.
  As $\C S_{n - 1}$-modules,
  \[
    V_{\lambda}
    \cong \bigoplus_{\mu} V_{\mu},
  \]
  where $\mu$ runs through all
  (unlabeled) Young
  tableaux obtained from $\lambda$ by
  removing one box.
\end{corollary}

\begin{definition}
  The \emph{Young graph} is the directed
  graph whose vertices are Young
  tableaux and we have an edge
  $\mu \to \lambda$ if
  $\mu$ is obtained from $\lambda$
  by removing one box.
\end{definition}

\begin{corollary}
  There is a graph isomorphism
  between the Young graph and the
  branching graph.
\end{corollary}

\begin{exercise}
  Prove that tensoring any
  $V_\lambda$ with $\sign_n$ gives a
  transposed Young tableau.
\end{exercise}

\section{Lie Groups}

\begin{remark}
  We will denote a $C^\infty$ manifold by
  $M$, and its tangent space at $m \in M$
  by $T_m M$. Denote by
  \[
    TM = \bigsqcup_{m \in M} T_m M
  \]
  the tangent bundle of $M$, and
  $\Vect(M)$ the sections of $TM$.
  If $f : X \to Y$ is a $C^\infty$ map,
  then we denote its differential at
  $x \in X$ by
  $T_x f : T_x X \to T_{f(x)} Y$.

  Recall that a map
  $f : X \to Y$ is an \emph{immersion}
  if $\rank T_x f = \dim X$ for all
  $x \in X$. In this case, by the inverse
  function
  theorem, we can choose
  local coordinates around $x$ and
  $f(x)$ such that
  \[
    f(x_1, \dots, x_n)
    = (x_1, \dots, x_n, 0, \dots, 0).
  \]
  An \emph{immersed submanifold} $N \subseteq M$
  is a subset with the structure of
  a manifold such that $i : N \hookrightarrow M$
  is an immersion (the topology of $N$
  need not be inherited from $M$). An
  \emph{embedded submanifold} $N \subseteq M$
  is an immersed submanifold such that
  $i : N \hookrightarrow M$ is also a
  homeomorphism onto its image.
\end{remark}

\begin{example}
  The figure eight curve
  $\R \to \R^2$ is an immersed submanifold
  but not embedded.
\end{example}

\begin{definition}
  A  \emph{(real) Lie group} $G$ is a
  group with a manifold structure such that
  the multiplication
  $G \times G \to G$ and
  inversion $G \to G$ are
  $C^\infty$ maps. A \emph{morphism}
  of Lie groups is a $C^\infty$ map
  $f$ such that
  \[f(gh) = f(g) f(h) \quad \text{and} \quad f(1) = 1.\]
\end{definition}

\begin{definition}
  A \emph{complex Lie group} is the
  same as a real Lie group, except
  with a complex manifold structure, i.e.
  there are charts to $\C^n$ such that
  the transition maps
  are analytic.
\end{definition}

\begin{example}
  The following are examples of
  Lie groups:
  \begin{enumerate}
    \item $\R^n$ with addition.
    \item $\R^* = \R \setminus \{0\}$ with
      multiplication, which has two
      components $\R_{\pm} = \{x \in \R : \pm x > 0\}$.
    \item $S^1 = \{z \in \C : |z| = 1\}$ with
      multiplication.
    \item $\GL(n, \R) \subseteq \R^{n^2}$
      with matrix multiplication.
    \item $\SU(2) = \{A \in \GL(2, \C) : A \overline{A}^T = 1, \det A = 1\}$, or
      \[
        \SU(2) = \left\{
          \begin{pmatrix}
            \alpha & \beta \\
            -\overline{\beta} & \alpha
          \end{pmatrix}
          : \alpha, \beta \in \C, |\alpha|^2 + |\beta|^2 = 1
        \right\} \cong S^3 \subseteq \R^4.
      \]
      Note that $\SU(2)$ is a real
      Lie group.
    \item The \emph{classical groups}
      $\SL(n, \R)$, $\SL(n, \C)$, $\OO(n, \R)$,
      $\OO(n, \C)$, $\Sp(n, \R)$, $\Sp(n, \C)$, etc.
  \end{enumerate}
\end{example}

\begin{theorem}
  Let $G$ be a (real or complex) Lie group.
  Denote by $G^0$ the connected component
  of the identity. Then $G^0$ is a normal
  subgroup of $G$ and is a Lie group.
  The quotient is a discrete group.
\end{theorem}

\begin{proof}
  Note that the image of a connected
  topological space under a continuous
  map is connected, so the
  inverse map sends $G^0 \to G^0$. The
  same argument works for multiplication,
  so $G^0$ is a Lie group.

  To show that $G^0$ is normal, let
  $h \in G^0$. Note that for any $g$,
  the map $h \mapsto g h g^{-1}$ is
  continuous, so $g h g^{-1}$ must lie
  in the same connected component
  $G^0$. Thus
  $G^0$ is a normal subgroup of $G$.

  Finally, the quotient is discrete since
  $G^0$ is open and its cosets
  partition $G$.
\end{proof}

\begin{theorem}
  If $G$ is a connected (real or complex)
  Lie group, then its universal cover
  $\widetilde{G}$ has a canonical
  structure of a Lie group such that
  the covering map
  $p : \widetilde{G} \to G$ is a morphism
  of Lie groups, and
  $\ker p \cong \pi_1(G)$ is discrete
  and central.
\end{theorem}

\begin{definition}
  A \emph{closed Lie subgroup} $H$ of a
  (real or complex) Lie group $G$
  is a subgroup which is a
  submanifold (complex submanifold
  in the complex case).
\end{definition}

\begin{theorem}[Cartan's theorem]
  Let $G$ be a (real or complex) Lie group.
  \begin{enumerate}
    \item Any closed Lie subgroup
      is closed in $G$.
    \item Any closed subgroup of a Lie
      group is a closed real Lie subgroup.
  \end{enumerate}
\end{theorem}

\begin{corollary}
  We have the following:
  \begin{enumerate}
    \item If $G$ is a connected (real or complex) Lie group
      and $U$ is a neighborhood of
      $1$, then $U$ generates $G$.
    \item Let $f : G_1 \to G_2$ be a
      morphism of (real or complex) Lie groups,
      where $G_2$ is connected and
      the differential
      $T_1 f : T_1 G_1 \to T_1 G_2$ at
      the identity is
      surjective. Then $f$ is surjective.
  \end{enumerate}
\end{corollary}

\begin{proof}
  (1) Assume $H$ is a subgroup generated
  by $U$. Then $H$ is open in $G$, since
  for any $h \in H$, $h U$ is a
  neighborhood of $h$ in $G$. Since
  $H$ is an open subset of a
  manifold, $H$ is a submanifold. Then
  $H$ is a closed Lie subgroup of $G$,
  so it is also closed. Thus $H = G$ since
  $G$ is connected.

  (2) Check this as an exercise.
\end{proof}
