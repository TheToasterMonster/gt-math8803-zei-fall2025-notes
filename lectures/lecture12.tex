\chapter{Sept.~29 --- The Exponential Map}

\section{Classical Lie Groups}

\begin{example}
  Let $\K$ be $\R$ or $\C$. The
  \emph{classical Lie groups} are
  \begin{enumerate}
    \item the \emph{general linear group} $\GL(n, \K)$,
    \item the \emph{special linear group} $\SL(n, \K)$,
    \item the \emph{orthogonal group} $\OO(n, \K)$,
    \item
      the \emph{special orthogonal groups}
      $\SO(n, \K)$
      and $\SO(p, q)$,
    \item the \emph{symplectic group}
      $\Sp(n, \K)$,
    \item the \emph{unitary groups} $\U(n)$ and $\SU(n)$
      which are real Lie groups,
    \item the \emph{compact symplectic group} $\mathrm{USp}(n) = \Sp(n, \C) \cap \SU(2n)$
      which is a real Lie group.
  \end{enumerate}
\end{example}

\begin{remark}
  How do we compute the dimensions
  of these classical Lie groups?
\end{remark}

\section{The Exponential Map for Matrix Groups}

\begin{remark}
  For $\GL(n, \K)$, write its
  Lie algebra as $\gl(n, \K)$.
\end{remark}

\begin{definition}
  For $x \in \Mat_n(\K) = \gl(n, \K)$,
  define the \emph{exponential map}
  \[
    \exp(x) = \sum_{k = 0}^\infty \frac{x^k}{k!}.
  \]
  This defines an analytic
  map $\gl(n, \K) \to \GL(n, \K)$
  with inverse map in a neighborhood
  of $I$ given by
  \[
    \log(1 + x)
    = \sum_{k = 1}^\infty \frac{(-1)^{k + 1} x^k}{k}.
  \]
\end{definition}

\begin{theorem}
  We have the following:
  \begin{enumerate}
    \item $\log(\exp(x)) = x$ and
      $\exp(\log(x)) = x$.
    \item $\exp(x) = 1 + x + \dots$,
      $\exp(0) = 1$, and
      $d \exp(0) = \id$.
    \item If $xy = yx$, then
      $\exp(x + y) = \exp(x) \exp(y)$;
    if $X$ and $Y$ commute (for
      $X, Y$ in some neighborhood of $I$), then
      $\log(XY) = \log(X) + \log(Y)$;
      also,
      $\exp(-x) \exp(x) = \id$, so
      $\exp(x) \in \GL(n, \K)$.
    \item For any $x \in \gl(n, \K)$,
      the map $\K \to \GL(n, \K)$
      by $t \mapsto \exp(tx)$ is a
      morphism of Lie groups. So in
      particular one has
      $\exp((t + s)x) = \exp(tx) \exp(sx)$.
    \item $\exp(A x A^{-1}) = A \exp(x) A^{-1}$
      and $\exp(x^T) = (\exp(x))^T$.
  \end{enumerate}
\end{theorem}

\begin{theorem}
  For any classical subgroup
  $G \subseteq \GL(n, \K)$, there exists
  a vector space $\g \subseteq \gl(n, \K)$
  such that for some neighborhood
  $U$ of $1$ in $\GL(n, \K)$ and
  some neighborhood $V$ of $0$ in
  $\gl(n, \K)$, the following
  maps are inverses of each other:
  \begin{center}
    \begin{tikzcd}
      (U \cap G) \arrow[r, shift left, "\log"] & (V \cap \g) \arrow[l, shift left, "\exp"]
    \end{tikzcd}
  \end{center}
\end{theorem}

\begin{proof}
  We have already proved this for
  $\GL(n, \K)$ and $\g = \gl(n, \K)$.

  Now consider $\SL(n, \K)$. Let
  $g \in \SL(n, \K)$ be close enough
  to the identity, so that
  $g = \exp(x)$ for some $x \in \gl(n, \K)$.
  Then $1 = \det(g) = \det(\exp(x))$.
  Now recall that
  \[
    \det(\exp(x)) = \exp(\tr x),
  \]
  which can be proved using the
  Jordan normal form. So this
  $\deg(g) = 1$ if and only if
  $\tr x = 0$. Thus we can take
  $\g = \slg(n, \K) = \{x \in \gl(n, \K) : \tr x = 0\}$.

  Next consider $\OO(n, \K)$
  and $\SO(n, \K)$. For
  $g \in \OO(n, \K)$, we have
  $g^T g = I$.
  Writing $g = \exp(x)$ and
  $g^T = \exp(x^T)$, we have
  $\exp(x^T) \exp(x) = I$
  since $x$ and $x^T$ commute.
  This translates to
  $x + x^T = 0$, so
  \[\mathfrak{o}(n, \K) = \{x \in \gl(n, \K) : x + x^T = 0\}.\]
  Note that
  $\mathfrak{o}(n, \K) = \mathfrak{so}(n, \K)$
  ($x + x^T = 0$ implies $\tr x = 0$)
  since $\SO(n, \K)$ is a neighborhood
  of $I$.

  For $\U(n)$, one can check
  we have the condition
  $x + x^\dagger = 0$
  (where $x^\dagger$ denotes the
  conjugate transpose of $x$)
  on the Lie algebra. This time,
  we do not automatically get
  $\tr x = 0$, so the Lie algebra
  of $\SU(n)$ has the
  two conditions
  $x + x^\dagger = 0$ and $\tr x = 0$.

  One can check the remaining
  classical groups similarly.
\end{proof}

\begin{corollary}
  Each classical group is a Lie group
  with tangent space at the identity
  $T_1 G = \g$
  and $\dim G = \dim \g$. Also,
  $\U(n)$, $\SU(n)$, and $\mathrm{USp}(n)$
  are real Lie groups, while
  $\GL(n, \K)$, $\SL(n, \K)$,
  $\SO(n, \K)$, $\OO(n, \K)$,
  and $\Sp(n, \K)$ are real or complex
  depending on $\K$.
\end{corollary}

\section{The Exponential Map in General}

\begin{remark}
  For $\g = T_1 G$, we want to
  define $\exp : \g \to G$ for a
  general Lie group $G$.
\end{remark}

\begin{prop}
  Let $G$ be a (real or complex) Lie
  group, $\g = T_1 G$, and
  $x \in \g$. Then there exists a unique
  morphism of Lie groups
  $\gamma_x : \K \to G$
  such that $\gamma_x(0) = x$.
  Here $\gamma_x(t)$ is known as
  a \emph{1-parameter subgroup}.
\end{prop}

\begin{proof}
  Motivated by the matrix case,
  where $\gamma_x(t) = \exp(tx)$
  satisfies $\dot{\gamma}(t) = \gamma(t) \dot{\gamma}(0) = \gamma(t) x$, we
  define the differential equation
  \[
    \dot{\gamma}(t)
    = T_1 L_{\gamma(t)} \dot{\gamma}(0),
  \]
  for which it suffices to construct
  $\gamma$ satisfying
  $\gamma(t + s) = \gamma(t) \gamma(s)$
  by the uniqueness of solutions to
  the differential equation. So it
  suffices to show that such a $\gamma$
  exists. Let
  $\gamma(t) = \Phi^t(1)$ and
  $\gamma(t + s) = \Phi^{t + s}(1)$,
  where $\Phi$ is the flow of
  a left-invariant vector field.
  By left-invariance, we have
  \[
    \Phi^t(g_1 g_2) = g_1 \Phi^t(g_2)
    \quad \text{and} \quad
    \Phi^{t + s}(1)
    = \Phi^s(\Phi^t(1))
    = \Phi^s(\gamma(t) \cdot 1)
    = \gamma(t) \Phi^s(1)
    = \gamma(t) \gamma(s).
  \]
  Thus $\gamma(t + s) = \gamma(t) \gamma(s)$, and
  we get the desired map
  $\gamma_x : \K \to G$.
\end{proof}

\begin{remark}
  The uniqueness of the
  $1$-parameter subgroups implies that
    $\gamma_x(\lambda t)
    = \gamma_{\lambda x}(t)$
  since
  \[
    \left.\frac{d \gamma_x(\lambda t)}{dt}\right|_{t = 0}
      = \lambda x.
  \]
\end{remark}

\begin{example}
  Let $G = (\R, +)$ with
  $\g = \R$. Then for $a \in \g$,
  we have $\gamma_a(t) = ta$
  and $\exp(a) = a$.
\end{example}

\begin{example}
  Let $G = S^1 = \R / \Z = \{z \in \C : |z| = 1\}$, where
  the identification
  $\R / \Z \to \{z \in \C : |z| = 1\}$
  is given by
  $\theta \mapsto e^{2 \pi i \theta}$
  for $\theta \in \R / \Z$.
  Then $\g = \R$, and for $a \in \g$,
  \[
    \exp(a) = a \Mod{\Z}
    \quad \text{or} \quad
    \exp(a) = e^{2\pi i a},
  \]
  depending on if we view $S^1$
  as $\R / \Z$ or
  as $\{z \in \C : |z| = 1\}$.
\end{example}

\begin{prop}
  Let $G$ be a (real or complex) Lie
  group.
  \begin{enumerate}
    \item Let $v$ be a left-invariant
      vector field on $G$. Then
      time flow of the vector
      field $v$ is given by
      $g \mapsto g \exp(tx)$, where
      $x = v(1)$.
    \item Let $v$ be a right-invariant
      vector field on $G$. Then the time
      flow of the vector field $v$ is given by
      $g \mapsto \exp(tx) g$, where
      $x = v(1)$.
  \end{enumerate}
\end{prop}

\begin{theorem}[Summary]
  Let $G$ be a (real or complex)
  Lie group and
  $\g = T_1 G$. Then
  \begin{enumerate}
    \item $\exp(x) = 1 + x + \dots$,
      $\exp(0) = 1$, and $T_0 \exp : T_1 G \xrightarrow{\id} T_1 G$
    \item The exponential map is a
      diffeomorphism (analytic map
      for complex $G$) between some
      neighborhood of $0$ in $\g$ and
      some neighborhood of $1$ in $G$.
    \item $\exp((t + s)x) = \exp(tx) \exp(sx)$
      for all $t, s \in \K$.
    \item For any morphism of
      Lie groups $\varphi : G_1 \to G_2$ and
      any $x \in T_1 G_1$, we have
      \[
        \exp(T_1 \varphi(x))
        = \varphi(\exp(x)).
      \]
    \item For any $g \in G$ and
      $x \in \g$, we have
      $g \exp(x) g^{-1}
        = \exp(\mathrm{Ad}_g x)$.
  \end{enumerate}
\end{theorem}

\begin{proof}
  (4) Note that $\varphi(\exp(tx))$
  is a one-parameter subgroup
  with tangent vector at identity
  \[
    \left.\frac{d}{dt}\right|_{t = 0}
      \varphi(\exp(tx))
      = T_1 \varphi \cdot x.
  \]
  By the uniqueness of
  one-parameter subgroups, this must
  be equal to $\exp(T_1 \varphi \cdot x)$.

  $(5)$ This follows
  from $(4)$ by taking
  $\varphi$ to be conjugation by $g$.
\end{proof}

\begin{prop}
  Let $G_1, G_2$ be (real or complex)
  Lie groups. If $G_1$ is connected,
  then any Lie group morphism
  $\varphi : G_1 \to G_2$ is
  uniquely determined by the
  linear map
  $T_1 \varphi : T_1 G_1 \to T_1 G_2$.
\end{prop}

\begin{example}
  Consider $\SO(3, \R)$, and let
  \[
    J_x =
    \begin{pmatrix}
      0 & 0 & 0 \\
      0 & 0 & -1 \\
      0 & 1 & 0
    \end{pmatrix}, \quad
    J_y =
    \begin{pmatrix}
      0 & 0 & 1 \\
      0 & 0 & 0 \\
      -1 & 0 & 0
    \end{pmatrix}, \quad
    J_y =
    \begin{pmatrix}
      0 & -1 & 0 \\
      1 & 0 & 0 \\
      0 & 0 & 0
    \end{pmatrix},
  \]
  which form a basis for
  $\g = \mathfrak{so}(3, \R)$.
  Then one can check that
  \[
    \exp(t J_z) =
    \begin{pmatrix}
      \cos t & -\sin t & 0 \\
      \sin t & \cos t & 0 \\
      0 & 0 & 1
    \end{pmatrix},
  \]
  with similar formulas
  for $\exp(t J_x)$ and $\exp(t J_y)$.
\end{example}
