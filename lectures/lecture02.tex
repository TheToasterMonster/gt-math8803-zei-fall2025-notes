\chapter{Aug.~20 --- Algebras and Modules}

\section{More on Algebras and Modules}
\begin{definition}
  A \emph{free} module is a module
  which has a basis.
\end{definition}

\begin{example}
  Consider the coordinate module
  $A^{\oplus I}$. Then a basis is given
  by $e_i = \{\delta_{ij}\}_{j \in I}$
    for $i \in I$.
\end{example}

\begin{prop}
  Let $M$ be a left $A$-module. Let
  $I$ be an index set and let $m_i \in M$
  for $i \in I$. Then
  \begin{enumerate}
    \item There exists a unique $A$-linear
      map $A^{\oplus I} \to M$ which sends
      $e_i \mapsto m_i$.
    \item This map is surjective if and only
      if the elements $m_i$ span $M$.
      In particular, every $M$ is isomorphic
      to a quotient of a free module.
    \item This map is an isomorphism
      if and only if $\{m_i\}$
      form a basis of $M$. In particular,
      every coordinate module is a
      free module.
  \end{enumerate}
\end{prop}

\begin{proof}
  Left as an exercise.
\end{proof}

\begin{example}
  Suppose $M$ is spanned by a single
  element $m$. Then $M \cong A / I$,
  where $I$ is the left ideal
  \[
    I = \{a \in A : am = 0\}.
  \]
\end{example}

\begin{example}
  We can now construct the following
  examples of algebras:
  \begin{enumerate}
    \item Let $\Mat_n(A)$ be the
      set of $n \times n$ matrices with
      entries in $A$. If $A$ is a
      $k$-algebra, then $\Mat_n(A)$
      is also a $k$-algebra.
    \item If $G$ is a group, then the
      group algebra $kG$ (for a ring $k$)
      given by
      \[
        kG = \left\{\sum_{g \in G} a_g g : a_g \in k\right\}
      \]
      is a free module with basis identified
      with the elements of $G$.

      The importance of of this
      object is as follows: Let $G$
      be a group and $B$ an algebra.
      Consider the set of maps satisfying
      $1_G \mapsto 1_B$ and respecting
      the group multiplication. This
      set is in bijection with maps
      $kG \to B$ (they extend by linearity).
      If $V$ is a vector space and
      $B = \End(V)$, then this statement
      says that there is a bijection
      between the representations of the
      group $G$
      and the representations of the group
      algebra $kG$.
    \item If $I$ is a two-sided ideal, then
      $A / I$ has a natural algebra
      structure.
    \item If $A_1, A_2$ are $k$-algebras,
      then the direct sum
      $A_1 \oplus A_2$ is again a
      $k$-algebra (with component-wise multiplication).
      One can extend this by induction to
      a finite direct sum, but note that
      we lose
      the multiplicative identity in
      an infinite direct sum (so we
      do not get an algebra in the infinite
      case).
  \end{enumerate}
\end{example}

\section{Module of Homomorphisms}
\begin{definition}
  Let $k$ be a commutative ring and
  $A$ a $k$-algebra. Let $M, N$ be
  left $A$-modules. Denote by
  $\Hom_A(M, N)$ the set of all
  $A$-module homomorphisms $M \to N$.
  Give $\Hom_A(M, N)$ a $k$-module
  structure via
  \[
    [\varphi_1 + \varphi_2](m) = \varphi_1(m) + \varphi_2(m), \quad
    [r \varphi](m) = r \varphi(m)
  \]
  for $\varphi_1, \varphi_2 \in \Hom_A(M, N)$,
  $r \in k$, and $m \in M$.
\end{definition}

\begin{remark}
  Let $L, M, N$ be left $A$-modules. Then
  we can define a $k$-bilinear map
  \begin{align*}
    \Hom_A(M, N) \times \Hom_A(L, M) &\longrightarrow \Hom_A(L, N) \\
    (\varphi, \psi) &\longmapsto \varphi \circ \psi.
  \end{align*}
\end{remark}

\begin{exercise}
  Let $N_2$ be an $A$-module,
  $N_1 \subseteq N_2$ an $A$-submodule,
  and $N_3 = N_2 / N_1$. Let
  $i : N_1 \hookrightarrow N_2$ be the
  inclusion and $\pi : N_2 \to N_3$ the
  projection. Define the maps
  \begin{align*}
    \widetilde{\iota} : \Hom(M, N_1) &\rightarrow \Hom(M, N_2) \\
    \varphi_1 &\longmapsto i \circ \varphi_1 \\
    \widetilde{\pi} : \Hom(M, N_2) &\rightarrow \Hom(M, N_3) \\
    \varphi_2 &\longmapsto \pi \circ \varphi_2.
  \end{align*}
  Then show that $\widetilde{\iota}$
  is injective and $\im \widetilde{\iota} = \ker \widetilde{\pi}$.
\end{exercise}

\begin{remark}
  Let $B$ be a $k$-algebra and $M$ and
  $A$-$B$-bimodule. Then for all
  $A$-modules $N$, we have that
  $\Hom_A(M, N)$
  is a left $B$-module via
  \[
    [b \varphi](m) = \varphi(mb).
  \]
  Similarly, if $N$ is an
  $A$-$C$-bimodule, then
  $\Hom_A(M, N)$ is a right $C$-module
  via
  \[
    [\varphi c](m) = \varphi(m)c.
  \]
  So if $M$ is an $A$-$B$-bimodule
  and $N$ an $A$-$C$-bimodule,
  then $\Hom_A(M, N)$ is a $B$-$C$-bimodule.
\end{remark}

\begin{remark}
  Let $M$ be a left $A$-module.
  We write $\End_A(M)$ in place of
  $\Hom_A(M, M)$, and composition gives
  $\End_A(M)$ the structure of a
  $k$-algebra. If $M = A^{\oplus n}$,
  then we can identify
  \[
    \End_A(M) = \Mat_n(A^{\mathrm{opp}}),
  \]
  where the opposite algebra exchanges
  the order of multiplication in the
  original algebra (this is because
  $\End_A(M)$ must respect the action by
  $A$). Then
  $M$ becomes an
  $A$-$(\Mat_n(A))^{\mathrm{opp}}$-bimodule.
\end{remark}

\begin{remark}
  If $M, N$ are two left $A$-modules, then
  $\Hom_A(M, N)$ 
  is an $\End_A(N)$-$\End_A(M)$-bimodule
  (by taking into account compositions).
\end{remark}

\section{Tensor Product of Modules}

\begin{remark}
  Let $A$ be a $k$-algebra,
  $M$ a right $A$-module, and $N$ a
  left $A$-module. We want to produce
  a $k$-module $M \otimes_A N$, which
  will be the \emph{tensor product} of
  $M$ and $N$ over $A$.
\end{remark}

\begin{definition}
  Let $L$ be a $k$-module. We say that
  a map $\varphi : M \times N \to L$
  is \emph{$A$-bilinear} if it is $k$-linear
  in both arguments and satisfies
  \[
    \varphi(ma, n) = \varphi(m, an)
  \]
  for any $a \in A$, $m \in M$, and
  $n \in N$.
\end{definition}

\begin{definition}[Universal property of the tensor product]
  There is an $A$-bilinear map
  \begin{align*}
    M \times N &\longrightarrow M \otimes_A N \\
    (m, n) &\longmapsto m \otimes n
  \end{align*}
  such that for any $A$-bilinear map
  $\varphi : M \times N \to L$, there
  exists a unique $k$-linear map
  $\psi : M \otimes_A N \to L$ such that
  $\varphi(m, n) = \psi(m \otimes n)$.
  As a diagram, this says that
  \begin{center}
  \begin{tikzcd}
    M \times N \ar[rr, "{(m, n) \mapsto m \otimes n}"] \ar[dr, "\varphi", swap] & & M \otimes_A N \ar[dl, "\psi"] \\
    & L
  \end{tikzcd}
  \end{center}
\end{definition}

\begin{exercise}
  If we choose $M \otimes'_A N$ with
  bilinear map $(m, n) \mapsto m \otimes' n$,
  then there exists a unique isomorphism
  $i : M \otimes_A N \to M \otimes'_A N$
  given by $i(m \otimes n) = m \otimes' n$.
\end{exercise}

\begin{corollary}
  Assume $M \otimes_A N$ satisfies
  the universal property. Then
  $\{m \otimes n\}$ span $M \otimes_A N$.
\end{corollary}

\begin{theorem}
  The tensor product $M \otimes_A N$
  exists for all right $A$-modules $M$
  and left $A$-modules $N$.
\end{theorem}

\begin{proof}
  We sketch the proof. First take
  $M$ to be free. Then we can define
  $M \otimes_A N$ as $N^{\oplus I}$,
  where we have
  $(e_i a_i) \otimes n = (a_i n)_{i \in I}$.
  The universal property is easy to check
  for this case, and the general
  case can be done by writing $M$ as
  a quotient of a free module.
\end{proof}

\begin{example}
  If $M, N$ are both free and
  $\{e_i\}_{i \in I}$, $\{f_j\}_{j \in J}$
  are bases of $M, N$, respectively,
  then $M \otimes_A N$ is a free
  $k$-module with basis vectors
  $\{e_i \otimes f_j\}_{i \in I, j \in J}$.
\end{example}

\begin{exercise}
  Let $M = A / I$, where $I$ is a right
  ideal. Show that
  $M \otimes_A N = N / IN$. Find out
  what happens when $N = A / J$, where
  $J$ is a left ideal, what can you say
  about $M \otimes_A N$ in terms of
  $A, I, J$?
\end{exercise}

\begin{prop}
  Assume $B$ is a $k$-algebra and
  $M$ a $B$-$A$-module. Then
  $M \otimes_A N$ is a left $B$-module.
\end{prop}

\begin{proof}
  Define $\varphi_b : M \times N \to M \otimes_A N$
  by $(m, n) \mapsto bm \otimes n$.
  This is bilinear, so by the universal
  property, there exists
  $\psi_b : M \otimes_A N \to M \otimes_A N$
  such that $\psi_b(m \otimes n) = bm \otimes n$, which gives
  the $B$-action.
\end{proof}

\begin{definition}
  Let $L$ be a $B$-module. A map
  $\varphi : M \times N \to L$ is
  called \emph{$B$-$A$-linear} if it
  is $k$-linear in both arguments and
  \[
    \varphi(ma, n) = \varphi(m, an), \quad
    \varphi(bm, n) = b \varphi(m, n)
  \]
  for all $m \in M$, $n \in N$, $b \in B$,
  and $a \in A$.
\end{definition}

\begin{prop}
  The left $B$-module $M \otimes_A N$
  has the following universal property:
  \begin{quote}
    Let $L$ be any left $B$-module
    and $\varphi : M \times N \to L$
    a $B$-$A$-linear map. Then there
    exists a unique $B$-linear map
    $\psi : M \otimes_A N \to L$
    such that $\psi(m \otimes n) = \varphi(m, n)$.
  \end{quote}
\end{prop}

\begin{example}
  Let $A_1, A_2$ be $k$-algebras. Then
  \begin{enumerate}
    \item $A_1 \otimes_k A_2$ has
      the structure of a $k$-algebra via
      \[
        (a_1 \otimes a_2)(b_1 \otimes b_2) = (a_1 b_1) \otimes (a_2 b_2),
      \]
      where $1 \otimes 1$ is a unit element.
    \item Let $M_i$ be a left $A_i$-module
      for $i = 1, 2$. Then
      $M_1 \otimes_k M_2$ is a module
      for $A_1 \otimes_k A_2$.
  \end{enumerate}
\end{example}

\section{Tensor-Hom Adjunction}
\begin{prop}[Tensor-Hom adjunction]
  Let $A, B$ be associative algebras,
  $N$ a $B$-module, $M$ an $A$-module,
  and $L$ an $A$-$B$-bimodule. Then
  \begin{enumerate}
    \item $L \otimes_B N$ is an
      $A$-module;
    \item $\Hom_A(L, M)$ is a
      $B$-module.
  \end{enumerate}
  Moreover, there is a natural
  $k$-linear isomorphism
  \[
    \Hom_A(L \otimes_B N, M) \overset{\cong}{\longrightarrow} \Hom_B(N, \Hom_A(L, M)).
  \]
\end{prop}

\begin{proof}
  By the universal property, there is a
  natural map
  \[
    \Hom_A(L \otimes_B N, M)
    \overset{\cong}{\longrightarrow} \Bilin_{A, B}(L \times N, M).
  \]
  So it suffices to find
  \begin{align*}
    \Hom_B(N, \Hom_A(L, M))
    &\overset{\cong}{\longrightarrow} \Bilin_{A, B}(L \times N, M) \\
    f &\longmapsto \varphi_f.
  \end{align*}
  Construct this map by
  $\psi_f(e, n) = [f(n)](e)$, with
  inverse $h \mapsto \psi(\cdot, h)$
  for $\psi \in \Bilin_{A, B}(L \times N, M)$.
\end{proof}

\begin{example}
  If we have an algebra homomorphism
  $B \to A$, where $A$ is a
  an $A$-$B$-bimodule. One can
  show as an exercise that
  $\Hom_A(A, M)$ is naturally identified
  with $M$ as an $A$-module and $B$-module.
  Thus by the Tensor-Hom adjunction,
  we have a natural isomorphism
  \[
    \Hom_A(A \otimes_B N, M)
    \overset{\cong}{\longrightarrow}
    \Hom_B(N, M).
  \]
\end{example}

\begin{definition}
  The $A$-module $A \otimes_B N$ is
  said to be \emph{induced} from $N$.
\end{definition}

\begin{remark}
  Assume there is
  $\Hom$ from $A \to B$. Then $B$
  is an $A$-$B$-bimodule. Take it as $L$
  in the Tensor-Hom adjunction.
  Note that $B \otimes_B N \cong N$
  as $A$-modules, and we have a natural
  isomorphism
  \[
    \Hom_A(N, M)
    \overset{\cong}{\longrightarrow}
    \Hom_B(N, \Hom_A(B, M)).
  \]
\end{remark}

\begin{definition}
  The $B$-module $\Hom_A(B, M)$ is
  said to be \emph{coinduced} from $M$.
\end{definition}
