\chapter{Oct.~8 --- Stabilizers and Center}

\section{Stabilizers}

\begin{prop}
  Consider the left action of $G$
  on itself: $L_g h = gh$. Then for
  all $x \in \g = \Lie(G)$, there is a
  corresponding vector field
  $\xi = L_* x \in \Vect(G)$
  which
  is right-invariant
  and satisfies $\xi(1) = x$.
\end{prop}

\begin{proof}
  The curve $\exp(tx)$ lies in
  $G$ and
  \[
    L_* x(h)
    = \left.\frac{d}{dt}\right|_{t = 0}
      (\exp(tx) h)
    = x h,
  \]
  which is a right-invariant vector
  field.
\end{proof}

\begin{corollary}
  $\g = \Lie(G)$ is isomorphic to
  the Lie algebra of right-invariant
  vector fields.
\end{corollary}

\begin{theorem}\label{thm:stabilizer}
  Let $G$ be a (real or complex)
  Lie group acting on a manifold $M$
  by $\rho$.
  Then
  \begin{enumerate}
    \item $G_m = \{g \in G : gm = m\}$
      is a closed Lie subgroup
      with Lie algebra
      $\mathfrak{h} = \{x \in \g : \rho_*(x)(m) = 0\}$,
      where $\rho_*(x)$ is the
      vector field (generator)
      corresponding to $x$.
    \item The map $G / G_m \to M$
      is an immersion. In particular,
      $\mathcal{O}_m$ is an immersed submanifold.
  \end{enumerate}
\end{theorem}

\begin{proof}
  (1) We have to show that there
  is some neighborhood $U$ around $1$
  in $G$ such that
  $U \cap G_m$ is a submanifold
  with tangent space $T_1 G_m = \mathfrak{h}$.
  Note the following:
  \begin{enumerate}[(i)]
    \item $\mathfrak{h}$ is closed
      under the commutator
      (the vector field commutator)
    \item $\xi = \rho_*(x) \in \mathfrak{h}$
      vanishes at $m$, so
      $\rho(\exp(tx))(m) = \Phi_\xi^t(m) = m$, so
      $\exp(tx) \in G_m$.
  \end{enumerate}
  Write $\g = \mathfrak{h} \oplus \mathfrak{u}$
  as vector spaces.
  Then $\rho_* : \g \to T_m M$ has
  $\ker \rho_* = \mathfrak{h}$, so
  $\rho_*|_\mathfrak{u}$ is injective.
  Thus the map
  \begin{align*}
    \mathfrak{u} &\longrightarrow M \\
    y &\longmapsto \rho(\exp(y))(m)
  \end{align*}
  is injective in some neighborhood
  of $0$ in $\mathfrak{u}$ by
  the implicit function theorem, so
  $\exp(y) \in G_m$ if and only if
  $y = 0$. So in a small neighborhood
  of $1$, we can write any $g \in G$
  as
  \[
    g = \exp(y) \exp(x), \quad
    y \in \mathfrak{u}, x \in \mathfrak{h}
  \]
  by the inverse function theorem. Then
  we have
  \[
    gm = \exp(y) \exp(x) m
    = \exp(y) m,
  \]
  so $g \in G_m$ if and only if
  $g \in \exp(\mathfrak{h})$ (elements
  of this kind generate the submanifold).

  (2) We have seen that
  $T_1(G / G_m) \cong \g / \mathfrak{h} \cong \mathfrak{u}$,
  so the injectivity of
  $\rho_* : \mathfrak{u} \to T_m M$
  shows that the map
  $\rho : G / G_m \to M$ is an immersion,
  as desired.
\end{proof}

\begin{corollary}
  Let $f : G_1 \to G_2$ be a morphism
  of (real or complex) Lie groups
  and $f_*$ the induced map
  of Lie algebras. Then
  $\ker f$ is a closed Lie subgroup
  with Lie algebra
  $\ker f_*$, and the map
  $G_1 / {\ker f} \to G_2$ is an
  immersion. If $\im f$ is a
  submanifold, then we have an isomorphism
  $\im f \cong G_1 / {\ker f}$.
\end{corollary}

\begin{proof}
  Let $G_1$ act on $G_2$ by
  $\rho(g) h = f(g) h$ for
  $g \in G_1$, $h \in G_2$.
  The stabilizer of $1$ is exactly
  $\ker f$, so it is a closed Lie
  subgroup by Theorem
  \ref{thm:stabilizer}. The rest
  also follows from
  Theorem \ref{thm:stabilizer}.
\end{proof}

\begin{corollary}
  Let $V$ be a representation of
  $G$ and $v \in V$. Then the
  stabilizer $G_v$ is a closed Lie
  subgroup in $G$ with
  Lie algebra
  $\{x \in \g : x v = 0\}$.
\end{corollary}

\begin{example}
  Let $V$ be a vector space over
  $\K$ and $B$ a bilnear form on $V$.
  Then
  \[
    \OO(V, B)
    = \{
      g \in \GL(V) :
      B(gv, gw) = B(v, w) \text{ for all } v, w \in V
    \}
  \]
  has Lie algebra
  \[
    \mathfrak{o}(V, B)
    = \{
      x \in \gl(V) :
      B(xv, w) + B(v, xw) = 0
    \}.
  \]
  We claim that $\OO(V, B)$ is
  always a Lie group with such Lie
  algebra: Let $G$ act on the space
  of bilinear forms by
  $g F(v, w) = F(g^{-1} v, g^{-1} w)$,
  then $\OO(V, B)$ is the stabilizer
  of $B$.
\end{example}

\begin{example}
  Let $A$ be a finite-dimensional
  associative algebra with multiplication
  $\mu : A \times A \to A$, and define
  \[
    \Aut(A) =
    \{
      g \in \GL(A) : 
      \mu(ga, gb) = g \mu(a, b) \text{ for all } a, b \in A
    \}.
  \]
  We claim that $\Aut(A)$ is a
  Lie group with Lie algebra
  \[
    \Der(A)
    = \{
      x \in \gl(A) :
      \mu(xa, b) + \mu(a, xb)
      = x \mu(a, b) \text{ for all } a, b \in A
    \}.
  \]
  Let $W$ be the space of all
  linear maps $A \otimes A \to A$, and
  let $\GL(A)$ acts on $W$ by
  \[
    (gf)(a \otimes b)
    = gf(g^{-1} a \otimes g^{-1} b).
  \]
  Then $\Aut(A)$ is exactly the
  stabilizer $G_\mu$. The same
  argument shows that
  \[
    \Aut(\g)
    = \{
      g \in \GL(\g)
      : [ga, gb] = g[a, b]
      \text{ for all } a, b \in \g
    \}
  \]
  is a Lie group with Lie algebra
  \[
    \Der(\g)
    = \{
      x \in \gl(\g) :
      [xa, b] + [a, xb] = x[a, b]
      \text{ for all } a, b \in \g
    \}.
  \]
  Note that $x$ could be $\ad_c$
  for some $c \in \g$, these are
  called the \emph{inner derivations}.
\end{example}

\section{Center}

\begin{definition}
  Let $\g$ be a Lie algebra. The
  \emph{center} of $\g$ is
  \[
    \mathfrak{z}(\g)
    = \{
      x \in \g : [x, y] = 0
      \text{ for all } y \in \g
    \}.
  \]
\end{definition}

\begin{remark}
  The center
  $\mathfrak{z}(\g)$ is an ideal in $\g$.
\end{remark}

\pagebreak

\begin{theorem}
  Let $G$ be a connected Lie group.
  Then its center $\mathcal{Z}(G)$
  is a closed Lie subgroup with
  Lie algebra $\mathfrak{z}(\g)$. If
  $G$ is not connected, then
  $\mathcal{Z}(G)$ is still a closed Lie
  subgroup, however its
  Lie algebra is smaller than
  $\mathfrak{z}(\g)$.
\end{theorem}

\begin{proof}
  Let $g \in G$ and $x \in \g$. Note
  that
  \[
    \exp(\Ad_g tx)
    = g \exp(tx) g^{-1},
  \]
  so $g$ commutes with $\exp(tx)$
  if and only if $\Ad_g x = x$.
  For connected Lie groups,
  the elements
  $\exp(tx)$ for all $x \in \g$
  generate the entire group, so
  $g \in \mathcal{Z}(G)$ if
  and only if $\Ad_g x = x$ for all
  $x \in \g$. Thus we have
  $\mathcal{Z}(G) = \ker \Ad$
  for the adjoint
  action
  $\Ad : G \to \GL(\g)$.
\end{proof}

\begin{example}
  $\OO(2)$ and $\SO(2)$ have the
  same Lie algebra, but
  $\OO(2)$ has center
  $\{\pm I\}$.
\end{example}

\begin{remark}
  Call $G / \mathcal{Z}(G)$ the
  \emph{adjoint group} associated
  to $G$. Denote
  \[
    \Ad G = G / \mathcal{Z}(G)
    = \im (\Ad : G \to \GL(\g))
    \quad\text{and} \quad
    \ad \g
    = \g / \mathfrak{z}(\g)
    = \im (\ad : \g \to \gl(\g)).
  \]
\end{remark}

\begin{example}
  Consider $\SL(2, \R)$. Then
  $\Ad(\SL(2, \R)) = \SL(2, \R) / \{\pm I\} = \PSL(2, \R)$.
\end{example}

\section{The Baker-Campbell-Hausdorff Formula}

\begin{theorem}
  Let $x, y \in \g$ such that
  $[x, y] = 0$. Then
  \[
    \exp(x) \exp(y)
    = \exp(x + y)
    = \exp(y) \exp(x).
  \]
\end{theorem}

\begin{proof}
  Let $\xi, \eta$ be the right-invariant
  vector fields corresponding to $x, y$,
  and $\Phi^t_\xi, \Phi^t_\eta$
  the corresponding time flows.
  Then the formula
  \[
    \Phi^t_{\xi}
    \Phi^s_{\eta}
    \Phi^{-t}_{\xi}
    \Phi^{-s}_{\eta}
    = ts[\xi, \eta] + \cdots
  \]
  implies that $[\xi, \eta] = 0$
  since this is an isomorphism of
  Lie algebras. Now observe that
  \[
    (\Phi^s_{\xi})_*
    \left.\frac{d}{dt}\right|_{t = 0}
    (\Phi^t_{\eta})_* \eta = 0,
  \]
  so $\frac{d}{dt} (\Phi^t_{\eta})_* \eta = 0$, which
  implies that
  $(\Phi^t_{\eta})_* \eta = \eta$ since
  the flow of $\xi$ preserves $\eta$.
  Then
  \[
    \Phi^t_{\xi} \Phi^s_{\eta}
    \Phi^{-t}_{\xi}
    = \Phi^s_{\eta},
  \]
  and applying this to $1$ gives
  $\exp(tx) \exp(sy) \exp(-tx) = \exp(sy)$.
\end{proof}

\begin{theorem}[Baker-Campbell-Hausdorff formula]
  For small enough $x, y \in \g$, we have
  \[
    \exp(x) \exp(y)
    \exp(\mu(x, y)),
  \]
  where $\mu$ is given by
  \[
    \mu(x, y) = x + y + \sum_{n \ge 2} \mu(x, y)
    = x + y + \frac{1}{2} [x, y]
    + \frac{1}{12} ([x, [x, y]] + [y, [y, x]])
    + \cdots.
  \]
  In the above, the $\mu_n$ are
  degree-$n$ commutators
  of $x$ and $y$.
\end{theorem}
