\chapter{Oct.~15 --- Representations of Lie Groups}

\section{More Examples of Representations}

\begin{example}
  Let $V, W$ be representations of
  $G$ or $\Lie(G)$. Then
  \begin{enumerate}
    \item $\End(V) \cong V \otimes V^*$
      is a representation, with
      $G$-action
      \[
        g : A \mapsto
        \rho_V(g) A \rho_V(g)^{-1}
      \]
      for $g \in G$ and $\g$-action
      $x : A \mapsto \rho_V(x) A - A \rho_V(x)$
      for $x \in \g$.
    \item $\Hom(V, W)$ is a representation.
      Write down the $G$- and $\g$-actions
      as an exercise.
    \item The space of bilinear forms
      is a representation with
      $G$-action $g B(v, w) = B(g^{-1} v, g^{-1} w)$.
  \end{enumerate}
\end{example}

\section{Invariants}

\begin{definition}
  Let $V$ be a representation of a Lie
  group $G$. Then $v \in V$ is called an
  \emph{invariant} of the Lie group action
  if $\rho(g) v = v$ for all $g \in G$.
  The space of invariants is denoted by $V^G$.
  For a Lie algebra $\g$, we say
  $v$ is an \emph{invariant} if
  $\rho(x) v = 0$ for all $x \in \g$.
  Similarly, $V^\g$ denotes the
  space of invariants.
\end{definition}

\begin{prop}
  If $G$ is connected, then
  $V^G = V^\g$ where
  $\g = \Lie(G)$.
\end{prop}

\begin{example}
  Let $V, W$ be representations. Then
  $G$ acts on
  $\Hom(V, W)$ by
  \[
    g : A \mapsto \rho_W(g) A \rho_V(g)^{-1},
  \]
  and we have $(\Hom(V, W))^G = \Hom_G(V, W)$.
  In particular, $V^G = (\Hom(\C, V))^G = \Hom_G(\C, V)$,
  where $\C$ is the trivial representation
  of $G$.
\end{example}

\begin{example}
  Let $B$ be a bilinear form. The space of
  bilinear forms on $V \otimes W$ form
  a representation of $G$ if $V, W$
  are representations of $G$. The invariants
  are those $B$ such that
  $B(g v, g w) = B(v, w)$.
\end{example}

\begin{prop}
  A bilinear form $B$ is an invariant if
  and only if $v \mapsto B(v, \cdot)$ is a
  morphism of representations.
\end{prop}

\section{Irreducible Representations}

\begin{definition}
  A representation $V$ of $G$ or $\g = \Lie(G)$
  is called \emph{irreducible} if it has
  no subrepresentations except for
  $V$ and $\{0\}$. Otherwise, we say that
  $V$ is \emph{reducible}.
\end{definition}

\begin{example}
  $\SL(n, \C)$ acting
  on $\C^n$ is irreducible.
\end{example}

\begin{remark}
  Note that if $W$ is a subrepresentation
  of $V$, then we have an exact sequence
  \[
    0 \longrightarrow W \longrightarrow V
    \longrightarrow V/W \longrightarrow 0.
  \]
  Do we have $V = W \oplus V/W$?
  If this is true for every $W$, then $V$ is \emph{completely reducible} (see below).
\end{remark}

\begin{definition}
  A representation $V$ is called \emph{completely reducible}
  if
  $V \cong \bigoplus_i V_i$
  for $V_i$ irreducible.
\end{definition}

\begin{remark}
  In general, we can write a
  completely reducible representation
  $V$ as
  \[
    V \cong \bigoplus_i n_i V_i,
  \]
  where the $n_i$ are multiplicities,
  i.e. contributions of isomorphic
  summands.
\end{remark}

\begin{example}
  Let $G = \R$ with $\g = \R$. Then a
  representation of $V$ is a linear
  map $\R \to \End(V)$, so it is
  of the form $t \mapsto tA$ for any
  element $A \in \End(V)$. On the
  group level this is $t \mapsto \exp(tA)$.
  Note that $A$ has some Jordan normal form,
  so it will not split into
  irreducibles unless $A$ is diagonalizable.
\end{example}

\begin{remark}
  The following are the classic goals
  of representation theory:
  \begin{enumerate}
    \item For a given group $G$, classify
      all irreducible representations.
    \item For a given representation $V$
      of $G$, if it is completely reducible,
      find the decomposition of $V$ into
      irreducibles.
    \item For which Lie groups $G$ are
      all representations completely
      reducible?
  \end{enumerate}
\end{remark}

\begin{lemma}
  Let $\rho : G \to \GL(V)$ be a
  representation of a Lie group $G$
  (with corresponding Lie algebra
  representation $\rho_* : \g \to \gl(V)$).
  Let $A : V \to V$ be a diagonalizable
  intertwining operator, and let
  $V_\lambda \subseteq V$ be the eigenspace
  of $A$ with eigenvalue $\lambda$. Then
  the following holds for both $G$ and $\g$:
  \begin{enumerate}
    \item Each $V_\lambda$ is a
      subrepresentation.
    \item As a consequence of (1),
      $V = \bigoplus_\lambda V_\lambda$.
  \end{enumerate}
\end{lemma}

\begin{lemma}
  Let $V$ be a representation of $G$ and
  $z \in \mathcal{Z}(G)$ be a
  central element such that
  $\rho(z)$ is diagonalizable. Then
  $V = \bigoplus_\lambda V_\lambda$
  where $V_\lambda$ is the eigenspace of
  $\rho(z)$ with eigenvalue $\lambda$.
\end{lemma}

\begin{example}
  Let $\GL(n, \C)$ act on $\C^n \otimes \C^n$.
  Then $P : v \otimes w \mapsto w \otimes v$
  is an intertwining operator (i.e. it
  commutes with the action of $G$).
  Then $S^2 \C^n$ and $\wedge^2 \C^n$
  are eigenspaces of $P$, so
  \[
    \C^n \otimes \C^n =
    S^2 \C^n \oplus \wedge^2 \C^n,
  \]
  but we do not know at this point if
  these are irreducible.
\end{example}

\begin{lemma}[Schur's lemma]
  We have the following:
  \begin{enumerate}
    \item If $V$ is a complex irreducible
      representation of $G$, then the
      space of intertwining operators if
      $\Hom_G(V, V) = \C \id$.
      In other words, any endomorphism of
      an irreducible representation of $G$
      is multiplication by a scalar.
    \item If $V, W$ are non-isomorphic
      complex irreducible representations,
      then $\Hom_G(V, W) = 0$.
  \end{enumerate}
\end{lemma}

\begin{example}
  Let $G = \GL(n, \C)$. Then $\C^n$ is an
  irreducible representation of $G$, so
  every operator commuting with
  $\GL(n, \C)$ must be scalar.
  Thus $\mathcal{Z}(\GL(n, \C)) = \{\lambda \id : \lambda \in \C^\times\}$.
  Similarly, we can use this to see that
  $\mathfrak{z}(\gl(n, \C)) = \{\lambda \id : \lambda \in \C\}$
  for the Lie algebra.
\end{example}

\pagebreak
\begin{exercise}
  Extend the above to all other classical
  Lie groups.
\end{exercise}

\begin{corollary}
  Assume $V$ is a completely reducible
  representation of a Lie group $G$ (or
  Lie algebra $\g$). Then we have
  the following:
  \begin{enumerate}
    \item If $V = \bigoplus V_i$ with
      $V_i$ irreducible and pairwise
      non-isomorphic, then any intertwining
      operator $\Phi : V \to V$ is of the
      form $\Phi = \bigoplus_i \lambda_i \id_{V_i}$.
    \item If $V = \bigoplus_i n_i V_i = \bigoplus_i (\C^{n_i} \otimes V_i)$,
      then any intertwining operator
      $\Phi : V \to V$ is given by
      \[
        \Phi = \bigoplus_i (A_i \otimes \id_{V_i}), \quad
        A_i \in \End(\C^{n_i}).
      \]
  \end{enumerate}
\end{corollary}

\begin{prop}
  If $G$ is a commutative group
  (resp. if $\g$ is a commutative Lie algebra),
  then every irreducible complex representations
  of $G$ (resp. $\g$) is 1-dimensional.
\end{prop}

\begin{proof}
  Note that
  $\rho(g)$ commutes with the action of
  $G$ for every $g$, so
  $\rho(g) \sim \id$.
\end{proof}

\begin{example}
  Let $G = \R$ and $\g = \R$. Then
  a 1-dimensional representation
  of $\g$ is given by
  $a \mapsto \lambda a$ for
  $a \in \g$. Then for $b \in G$, we have
  $\rho(b) = \exp(\lambda b)$.
  So $\lambda$ labels representations.
  For $U(1) = S^1 = \R / \Z$ and
  $\g = \R$, we have $\rho(a) = 1$ for all
  $a \in \Z$, so $\rho(a) = \exp(2\pi i k a)$.
  Here $k$ labels the representations.
\end{example}

\section{Unitary Representations}

\begin{remark}
  Recall that for finite groups,
  every representation is completely
  reducible.
\end{remark}

\begin{definition}
  A complex representation of a real Lie
  group $G$ is called \emph{unitary}
  if there exists an invariant inner product
  (i.e. a Hermitian positive-definite
  bilinear form)
  on $V$, i.e.
  \[(\rho(g) v, \rho(g) w) = (v, w).\]
  One can make a similar definition for
  Lie algebras by differentiating.
\end{definition}

\begin{theorem}\label{thm:unitary}
  Every unitary representation is
  completely reducible.
\end{theorem}

\begin{proof}
  We induct on dimension. Assume that
  $V$ is not irreducible. Then
  we can write
  $V = W \oplus W^\perp$ for some
  nontrivial subrepresentation $W$.
  We claim $W^\perp$ is
  also a subrepresentation: For
  $w \in W^\perp$, $v \in W$,
  \[
    (gw, v)
    = (w, g^{-1} v)
    = 0
  \]
  since $g^{-1} v \in W$, so
  $gw \in W^\perp$. Thus
  $W^\perp$ is a subrepresentation, and
  $\dim W,\, \dim W^\perp < \dim V$.
\end{proof}

\begin{remark}
  Let $G$ be finite.
  Recall that if $B(V, W)$ is any inner
  product on $V$, then we can define
  \[
    (v, w) = \frac{1}{|G|} \sum_{g \in G} B(g v, g w),
  \]
  which one can check is positive-definite
  and $G$-invariant. So we get
  that every complex representation of a
  finite group is completely reducible.
  We can apply a similar idea to
  Lie groups.
\end{remark}

\section{Compact Lie Groups and Haar Measure}

\begin{definition}
  A \emph{right Haar measure} on a real
  Lie group $G$ is a Borel measure $dg$
  which is invariant under the right
  action of $G$ on itself, i.e. we have
  \[
    \int f(gh)\, dg
    = \int f(g)\, dg
  \]
  for any integrable function $f$ and
  $h \in G$. One defines a
  \emph{left Haar measure} similarly.
\end{definition}

\begin{lemma}\label{lem:1drep}
  Let $V$ be a 1-dimensional real
  representation of a compact Lie group $G$.
  Then for any $g \in G$, we have
  $|\rho(g)| = 1$.
\end{lemma}

\begin{proof}
  If $|\rho(g)| < 1$, then we have
  $\rho(g)^n \to 0$ as $n \to \infty$.
  But $\rho(G)$ is a compact subset of
  $\R^\times$, so it cannot contain
  a sequence converging to
  $0$. A similar argument works for
  $|\rho(g)| > 1$.
\end{proof}

\begin{theorem}
  Let $G$ be a real Lie group. Then
  \begin{enumerate}
    \item $G$ is orientable. Moreover,
      the orientation can be chosen such
      that the right action of $G$ on
      itself preserves orientation.
    \item If $G$ is compact, then
      for a fixed choice of right-invariant
      orientation on $G$, there exists
      a unique right-invariant top
      form $\omega$ such that
      \[
        \int_G \omega = 1.
      \]
    \item $\omega$ is also left-invariant
      if $G$ is connected (otherwise it is
      left-invariant up to sign).
  \end{enumerate}
\end{theorem}

\begin{proof}
  Choose an element in $\bigwedge^n \g^*$
  where $n = \dim G$. It can be uniquely
  translated by right-invariance to
  give a non-vanishing form $\widetilde{\omega}$.
  Then let
  \[
    I = \int_G \widetilde{\omega},
  \]
  then $\omega = \widetilde{\omega} / I$
  gives the desired $\omega$. Now
  note that $\bigwedge^n \g^*$ is
  a $1$-dimensional representation of
  $G$ via the adjoint action.
  Left-invariance of $\omega$ follows
  from Lemma \ref{lem:1drep}, and the
  sign is obtained as follows.
  Let $i : g \mapsto g^{-1}$ be the
  inversion map. If $\omega$ is
  left-invariant, then $i^*(\omega)$ is
  right-invariant, and
  $\omega$ and $i^*(\omega)$ are equal
  up to a sign: $i_* : \g \to \g$ is given
  by $x \mapsto -x$, so
  $i^* : \bigwedge^n \g^* \to \bigwedge^n \g^*$
  is given by $\omega \mapsto (-1)^n \omega$.
\end{proof}

\begin{theorem}
  Let $G$ be a compact real Lie group. Then
  it has a canonical Borel measure $dg$,
  called the \emph{Haar measure},
  which is both left and right invariant,
  invariant under $g \mapsto g^{-1}$, and
  satisfies
  \[
    \int_G dg = 1.
  \]
\end{theorem}

\begin{theorem}
  Any finite-dimensional representation
  of a compact Lie group is unitary and is
  thus completely reducible.
\end{theorem}

\begin{proof}
  Let $B(v, w)$ be a positive-definite
  Hermitian inner product. Then
  one can check
  \[
    \overline{B}(v, w)
    = \int_G B(g v, g w)\, dg
  \]
  is unitary, and the representation is
  completely reducible by Theorem
  \ref{thm:unitary}.
\end{proof}
