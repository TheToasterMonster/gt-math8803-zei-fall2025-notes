\chapter{Aug.~27 --- Semisimple Algebras}

\section{Semisimple Algebras}

\begin{definition}
  A finite-dimensional $\F$-algebra $A$
  is called \emph{semisimple} if it
  is isomorphic to a direct sum of
  simple algebras.
\end{definition}

\begin{remark}
  If $\F$ is algebraically closed, then
  $A$ is a direct sum of matrix
  algebras, i.e.
  $\bigoplus_{i = 1}^k \End_\F(U_i)$.
\end{remark}

\begin{theorem}\label{thm:semisimple-direct-sum}
  Let $U_1, \dots, U_k$ be
  finite-dimensional vector spaces
  over $\F$. Let $A = \bigoplus_{i = 1}^k \End_\F(U_i)$,
  so that $U_i$ is an irreducible
  $A$-module. Then every finite-dimensional
  $A$-module $V$ is isomorphic to a direct
  sum of several copies of $U_1, \dots, U_k$.
\end{theorem}

\begin{proof}
  Left as an exercise.
\end{proof}

\begin{corollary}
  Let $\F$ be algebraically closed, and
  $A$ be semisimple and finite-dimensional.
  Then
  \begin{enumerate}
    \item The number of isomorphism
      classes of irreducible
      $A$-modules is equal to
      $\dim \mathcal{Z}(A)$.
    \item Different irreducible
      modules have different central
      characters.
  \end{enumerate}
\end{corollary}

\begin{proof}
  $(1)$
  Let $A = \bigoplus_{i = 1}^k \End_\F(U_i)$.
  By Theorem \ref{thm:semisimple-direct-sum},
  the number of irreducible representations
  is $k$. We can also write
  \[
    \mathcal{Z}\left(\bigoplus_{i = 1}^k A_k\right)
    = \bigoplus_{i = 1}^k \mathcal{Z}(A_i),
  \]
  where $A_i = \End_\F(U_i)$.
  Since $\dim \mathcal{Z}(A_i) = 1$, we
  have $\dim \mathcal{Z}(\bigoplus_{i = 1}^k A_k) = k$
  as well.
  
  $(2)$ Use the projections
  $\mathcal{Z} \to \mathcal{Z}(A_i) \to \F$,
  which correspond to the central characteres.
\end{proof}

\section{Characterizations of Semisimple Algebras}
\begin{definition}
  Let $A$ be a finite-dimensional
  algebra. We say that a two-sided
  ideal $I \subseteq A$ is
  \emph{nilpotent} if $I^n = \{0\}$
  for some $n$.
\end{definition}

\begin{exercise}
  If $I, J$ are nilpotent, then
  show that $I + J$ is also nilpotent.
\end{exercise}

\begin{definition}
  The maximal nilpotent ideal of $A$,
  denoted $\rad(A)$, is
  called the \emph{radical} of $A$.
\end{definition}

\begin{theorem}
  Let $\F$ be algebraically closed and
  $A$ a finite-dimensional algebra.
  Then the following are equivalent:
  \begin{enumerate}
    \item $A$ is semisimple;
    \item all finite-dimensional
      representations of $A$ are completely
      reducible;
    \item $\rad(A) = \{0\}$.
  \end{enumerate}
\end{theorem}

\begin{proof}
  $(1 \Rightarrow 2)$ We have already
  shown this.

  $(2 \Rightarrow 3)$ Let $I = \rad(A)$,
  so $I^n = \{0\}$ for some $n \in \N$.
  Let $N$ be a finite-dimensional
  $A$-module. Then
  $I^\ell N$ is an $A$-submodule
  for $\ell = 0, \dots, n$. Since
  $N$ is completely reducible
  and $I^{\ell + 1}N \subseteq I^{\ell} N$, we have
  \[
    I^\ell N = N_\ell \oplus I^{\ell + 1}.
  \]
  Acting on both sides by $I$, we get
  $I N_{\ell} \subseteq I^{\ell + 1}$,
  so $IN_{\ell} = \{0\}$.
  Continuing, we get $IN = 0$, so
  $A = N$.

  $(3 \Rightarrow 1)$ Take $N_1, \dots, N_k$
  to be
  pairwise non-isomorphic irreducible
  $A$-modules. We have an
  epimorphism $A \to \bigoplus_{i = 1}^k \End_\F(N_i)$.
  Let $I$ be the kernel, so
  $I$ acts trivially on every
  irreducible $A$-module. We claim that
  $I$ is nilpotent.
  Take $A$ to be the regular module.
  Take a filtration
  \[
    A = A_0 \supseteq A_1 \supseteq \cdots
    \supseteq A_n = \{0\},
  \]
  where
  $A_i / A_{i + 1}$ is irreducible.
  Now $I$ acts trivially
  on $A_i / A_{i + 1}$, so
  $I A_i \subseteq A_{i + 1}$ for all
  $i$, thus $I^n = \{0\}$.
\end{proof}

\begin{remark}
  Assume $\Char(\F) = 0$.
  Consider the following bilinear
  form on $A$:
  \[
    (a, b)_U = \tr_U(ab),
  \]
  where $U$ is any $A$-module. Note that
  $U$ could be $A$.
\end{remark}

\begin{theorem}
  Let $\Char(\F) = 0$, and let
  $A$ be a finite-dimensional $\F$-algebra.
  Then $A$ is semisimple if and only if
  $(a, b)_A$ is nondegenerate.
\end{theorem}

\begin{proof}
  $(\Rightarrow)$ Assume $A$ is
  semisimple, so
  $A = \bigoplus_{i = 1}^k \End(U_i)$.
  Note that the restriction of
  $(\cdot, \cdot)_A$ to the
  direct summand $\End_\F(U_i)$
  coincides with
  $(\cdot, \cdot)_{\End_\F(U_i)}$.
  Let $E_{j \ell}$ denote the matrix with
  all $0$s except a single $1$ in the
  $(j, \ell)$ entry. Then we can compute
  that
  \[
    (E_{j \ell}, E_{j' \ell'})_{\End_\F(U_i)}
    = \delta_{ej'} \tr_{\End_\F(U_i)}(E_{j\ell'})
    = \delta_{e\ell} \delta_{j \ell'}
    \dim U_i.
  \]
  So if $\{E_{j \ell}\}$ is a basis, then
  $\{(\dim U_i)^{-1} E_{j \ell}\}$
  is the dual basis.
  This is nondenegerate if
  $\Char(\F) = 0$.

  $(\Leftarrow)$ Suppose $(\cdot, \cdot)_A$
  is nondegenerate. If $I$ is a nilpotent
  ideal, then for any $a \in I$
  such that $a^n = 0$. Then $\tr_A(a) = 0$
  for any $a \in I$, so
  $I \in \ker(\cdot, \cdot) = 0$.
  Since $(\cdot, \cdot)$ is nondegenerate,
  we have $I = \{0\}$.
\end{proof}

\section{Double Centralizer Theorem}

\begin{theorem}[Double centralizer theorem]
  Let $V$ be a finite-dimensional vector
  space over $\F$. Let
  $A \subseteq \End_\F(V)$ be a semisimple
  algebra, and
  set $B = \End_A(V)$. Then
  $A = \End_B(V)$.
\end{theorem}

\begin{proof}
  Let $A = \bigoplus_{i = 1}^k \End(U_i)$
  and $V$ be a faithful representation
  of $A$, so $V$ is completely reducible:
  \[
    V \cong \bigoplus_{i = 1}^k U_i \oplus M_i,
  \]
  where the $M_i$ are multiplicity spaces.
  Let $a = (\varphi_1, \dots, \varphi_k) \in A$
  (for $\varphi_i \in \End(U_i)$)
  act on $\End_{\F}(V)$ by
  \[
    (\varphi_1, \dots, \varphi_k)
    \longmapsto
    \sum_{i = 1}^k \varphi_i \otimes \id_{M_i}.
  \]
  Note that the $M_i$ are nonzero
  since $V$ is faithful. Then 
  $B = \bigoplus_{i = 1}^n \End(M_i)$
  embeds into $\End_\F(V)$ via
  \[
    (\psi_1, \dots, \psi_k)
    \longmapsto
    \sum_{i = 1}^k {\id_{U_i}} \otimes \psi_i,
  \]
  which completes the proof.
\end{proof}

\section{Representations of Finite Groups}

\begin{remark}
  Recall that to any group $G$ we can
  associate the group algebra
  $\F G$. For any representation of
  $G$, there is a representation
  of $\F G$ and vice versa.
\end{remark}

\begin{remark}
  Consider the following operations
  with representations. Let $U, V$
  be representations of $G$.
  \begin{enumerate}
    \item the \emph{tensor product} $U \otimes_\F V$, where
      $g(u \otimes v) = (g u) \otimes (gv)$;
    \item the \emph{dual} $U^*$
      defined by
      $\langle g \alpha, u \rangle = \langle \alpha, g^{-1} u \rangle$
      for $u \in U$, $\alpha \in U^*$,
      $g \in G$;
    \item $\Hom_{\F}(U, V)$, with
      action given by
      $[g \varphi](h) = g[\varphi(g^{-1} u)]$
      for $\varphi \in \Hom_\F(U, V)$.
  \end{enumerate}
\end{remark}

\begin{exercise}
  Show the following:
  \begin{enumerate}
    \item The tensor product of
      representations satisfies
      associativity, distributivity, and
      commutativity.
    \item There is an isomorphism
      of representations
      $U^* \otimes V \to \Hom(U, V)$.
    \item $\Hom_G(U, V) \subseteq \Hom(U, V)$
      coincides with the space
      of $G$-invariant elements.
  \end{enumerate}
\end{exercise}

\begin{remark}
  For the rest of this section,
  assume $\F$ is algebraically closed
  and $\Char \F = 0$.
\end{remark}

\begin{theorem}
  The group algebra $\F G$ is semisimple.
\end{theorem}

\begin{proof}
  It suffices to show that
  $(\cdot, \cdot)_{\F G}$ is nondegenerate.
  Take $g, g' \in G$, and note
  that $gg' : h \mapsto gg' h$, so
  \[
    (g, g')_{\F G}
    = \tr_{\F G}(gg') = \delta_{1, gg'} |G|,
  \]
  which is nondegenerate.
  Moreover,
  the basis $\{g\}$ in $\F G$
  corresponds to the dual basis
  $\{|G|^{-1}g^{-1}\}$.
\end{proof}

\begin{corollary}
  (Let $\F$ be algebraically closed
  and $\Char \F = 0$.)
  \begin{enumerate}
    \item Every finite-dimensional
      representation of $G$ is completely
      reducible.
    \item The number of isomorphism classes
      of irreducible representations
      is equal to the number of conjugacy
      classes of $G$.
    \item If $U_1, \dots, U_k$
      are all of the pairwise
      non-isomorphic irreducible
      representations of $G$, then
      \[
        |G| = \sum_{i = 1}^k (\dim U_i)^2.
      \]
  \end{enumerate}
\end{corollary}

\begin{proof}
  $(1)$ This follows from the semisimplicity
  of $\F G$.

  $(2)$ It suffices to show that
  $\dim \mathcal{Z}(\F G)$ equals the
  number of conjugacy classes of $G$.
  We have
  \[
    \mathcal{Z}(\F G)
    = \left\{
      \sum_{g \in G} a_g g :
      a_g \text{ is constant on conjugacy classes}
    \right\},
  \]
  i.e. we must have
  $a_{hgh^{-1}} = a_g$ for any
  $h \in G$. So the dimension is
  the number of conjugacy classes.
  
  $(3)$ This automatically follows
  from looking at the dimension of
  $\F G$.
\end{proof}
