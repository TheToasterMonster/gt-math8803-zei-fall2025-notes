\chapter{Sept.~15 --- Representations of \texorpdfstring{$S_n$}{Sn}, Part 3}

\section{More on Branching Graphs}
\begin{remark}
  Consider $\Hom_{\C S_m}(V^m, V^n)$.
  When $m = 1$, we may identify
  $\Hom_{\C S_1}(V^1, V^n) = \Hom_{\C}(\C, V^n)$
  with $V^n$ itself. For
  $P \in \Path(V^n)$, we will write
  $v_P$ for $\varphi_P$.
\end{remark}

\begin{corollary}
  We have the following:
  \begin{enumerate}
    \item the vectors $v_P$ for
      $P \in \Path(V^n)$ form a basis
      in $V^n$;
    \item each $v_P$ is an eigenvector
      for $J_k$ with eigenvalue
      $w_k$ for $k = 1, \ldots, n$. Note $w_1 = 0$ since $J_1 = 0$.
  \end{enumerate}
\end{corollary}

\begin{example}\label{ex:refl-paths}
  Consider the following:
  \begin{enumerate}
    \item $V^n = \refl_n$. We have
      $\refl_n \cong \refl_{n - 1} \oplus \triv_{n - 1}$.
      When $n = 2$, we have
      $\refl_2 = \triv_1$. Then any path
      $P \in \Path(V^n)$ must be of the form
      \[
        P = \triv_1 \to \dots \to \triv_i \to \refl_{i + 1} \to \dots \to \refl_n.
      \]
      The corresponding weights are
      $w_P = (0, 1, \dots, i - 1, -1, i, \dots, n - 2)$:
      Recall from before that $J_k$ acts on $\refl_{k - 1} \subseteq \refl_k$
      by $k - 2$ and
      $\triv_{k - 1} \subseteq \refl_k$ by
      $-1$.
  \end{enumerate}
\end{example}

\begin{exercise}
  Check that
  $v_P = (1, \dots, 1, -i, 0, \dots, 0)$
  in Example \ref{ex:refl-paths} (there are
  $i$ ones).
\end{exercise}

\begin{exercise}
  Let $V = \C^2$ be a representation
  of $S_4$. Write down two elements
  in $\Path(\C^2)$ and find the
  corresponding weights.
\end{exercise}

\begin{corollary}
  Let $m < n$ and $V^m \in \Irr(\C S_m)$,
  $V^n \in \Irr(\C S_n)$,
  $\underline{P} \in \Path(V^m)$,
  $\overline{P} \in \Path(V^m, V^n)$.
  Let $P$ be the path obtained by
  concatenating $\underline{P}$
  and $\overline{P}$. Then
  $v_P$ is proportional to
  $\varphi_{\overline{P}}(v_{\underline{P}})$.
\end{corollary}

\begin{proof}
  Both are clearly nonzero and
  lie in $V^1(P)$, which is one-dimensional.
\end{proof}

\section{Properties of Weights}
\begin{theorem}\label{thm:path-unique-weight}
  Let $P, P' \in \Path_n$. If
  $w_P = w_{P'}$, then $P = P'$.
\end{theorem}

\begin{proof}
  The proof is by induction. The
  $n = 1$ case is trivial. Now suppose
  the statement is true for $n - 1$.
  Let $\underline{P}, \underline{P}' \in \Path_{n - 1}$
  be truncations of $P, P' \in \Path_n$.
  Assume that
  \[
    \begin{cases}
      w_P = (w_1, \dots, w_n), \\
      w_{P'} = (w_1', \dots, w_n'),
    \end{cases}
  \]
  so $w_{\underline{P}} = (w_1, \dots, w_{n - 1})$
  and $w_{\underline{P}'} = (w_1', \dots, w_{n - 1}')$.
  If $w_P = w_{P'}$, then we have
  $w_{\underline{P}} = w_{\underline{P}'}$
  and thus $\underline{P} = \underline{P}'$
  by the inductive hypothesis.

  Now assume
  $V, V'$ are the endpoints of
  $P, P'$, respectively,
  $V, V' \in \Irr(\C S_n)$. We need
  to show that $V \cong V'$.
  Let $U \in \Irr(\C S_{n - 1})$
  be the endpoint of $\underline{P} = \underline{P}'$.
  Note that each $z \in \mathcal{Z}_{n - 1}(n)$
  acts on $U \subseteq V$ and $U \subseteq V'$ as a scalar.
  Denote these scalars by
  $\chi(z)$ and $\chi'(z)$, and note
  that $\chi(z) = \chi'(z)$:
  We know that $\mathcal{Z}_{n - 1}(n)$
  is generated by
  $\mathcal{Z}_{n - 1}$ and $J_n$, any
  $z \in \mathcal{Z}_{n - 1}$ acts on
  $U$ as a scalar with $\chi(z) = \chi'(z)$,
  and $J_n$ acts on both
  $U$'s embedded in $V, V'$ by $w_n$,
  so $\chi(J_n) = \chi'(J_n) = w_n$.

  Let $\mathcal{Z}_n(n)$ be the center of
  $\C S_n$, which is contained in
  $\mathcal{Z}_{n - 1}(n)$. Every
  $z \in \mathcal{Z}_n(n)$ acts on
  $V$ and $V'$ as scalars
  $\chi_V(z)$ and $\chi_{V'}(z)$,
  which must be the same scalars
  by which $z$ acts on $U$. Then
  $\chi_V$ and $\chi_{V'}$ are the
  same central characters, so we
  find that $V \cong V'$.
\end{proof}

\begin{definition}
  Define
  $\Wt_n = \{w_P : p \in \Path_n\}$.
  We say that two elements in
  $\Wt_n$ are \emph{$r$-equivalent}
  (the $r$ is for ``representation'')
  if the weights of the two paths
  are in the same irreducible module.
\end{definition}

\begin{remark}
  Theorem \ref{thm:path-unique-weight}
  states that there is a one-to-one
  correspondence
  $\Path_n \longleftrightarrow \Wt_n$.
  Moreover, $r$-equivalence is an
  equivalence relation and gives a
  one-to-one correspondence
  between equivalence classes and
  isomorphism classes of irreducible
  representations.

  Theorem \ref{thm:path-unique-weight} also
  implies that the basis vectors
  $v_P$ for $P \in \Path(V^n)$ are in
  bijection with weights in the
  corresponding equivalence class. Thus
  it suffices to study weights going
  forward.
\end{remark}

\begin{remark}
  We now see what happens when we
  vary paths. Consider a path
  \[P = (V^1 \to \dots \to V^n) \in \Path_n.\]
  Pick $i \in \{1, \ldots, n - 1\}$, and
  consider the space of all paths
  of the form
  \[
    P' = (V'^1 \to \dots \to V'^n),
    \quad \text{where } V'^j = V^j
    \text{ for } j \ne i.
  \]
  Denote this set by $\Path(P, i)$.
  We will prove the following theorem later:
\end{remark}

\begin{theorem}\label{thm:path-varying}
  Let $w_P = (w_1, \dots, w_n)$. Then
  the following are true:
  \begin{enumerate}
    \item $w_i \ne w_{i + 1}$;
    \item if $w_{i + 1} = w_i \pm 1$, then
      $\Path(P, i) = \{P\}$;
    \item if $w_{i + 1} \ne w_i \pm 1$, then
      $\Path(P, i)$ consists of
      two elements $P, P'$ and
      $w_{P'}$ is obtained from
      $w_P$ by permuting
      $w_i, w_{i + 1}$;
    \item if $i < n - 1$, then
      $w_i = w_{i + 1} \pm 1$ implies
      $w_{i + 2} \ne w_i$.
  \end{enumerate}
\end{theorem}

\begin{remark}
  To simplify notation, denote
  $V = V^n$, $\mathcal{Z}_{i - 1}(i + 1) \subseteq \C S_n$, and
  \[
    V_{P, i} = \Span\{v_{P'} : P' \in \Path(P, i)\}.
  \]
  Note that
  the $v_{P'}$ actually form a basis of
  $V_{P, i}$.
\end{remark}

\begin{prop}
  The subspace $V_{P, i} \subseteq V$
  is an irreducible $\mathcal{Z}_{i - 1}(i + 1)$-module.
\end{prop}

\begin{proof}
  Let $P = P_0 P_1 P_2$, where
  $P_0 \in \Path(V^{i - 1})$,
  $P_1 \in \Path(V^{i - 1}, V^{i + 1})$,
  and $P_2 \in \Path(V^{i + 1}, V^n)$.
  Then
  $\Path(P, i)$ consists of paths of
  the form $P_0 P_1' P_2$,
  where $P_1' \in \Path(V^{i - 1}, V^{i + 1})$.
  We have
  \[
    V_{P_0 P_1' P_2}
    = \varphi_{P_2}(\varphi_{P_1'}(v_{P_0})).
  \]
  Now consider the linear map
  \begin{align*}
    \Hom_{\C S_{i - 1}}(V^{i - 1}, V^{i + 1})
    &\longrightarrow V \\
    \psi &\longmapsto \varphi_{P_2}(\psi(v_{P_0})).
  \end{align*}
  Note that we have $\varphi_{P_1'} \mapsto v_{P_0 P_1' P_2}$
  in $V_{P, i}$, where the
  $v_{P_0 P_1' P_2}$ form a basis of
  $V_{P, i}$
  and the $\varphi_{P_1'}$ form a basis
  in $\Hom_{\C S_{i - 1}}(V^{i - 1}, V^{i + 1})$.
  In particular, this map is
  injective with image $V_{P, i}$.

  It only remains to show that this map
  is $\mathcal{Z}_{i - 1}(i + 1)$-linear,
  which is left as an exercise.
\end{proof}

\section{The Degenerate Affine Hecke Algebra}

\begin{remark}
  We want to study $\mathcal{Z}_{i - 1}(i + 1) \subseteq \C S_n$
  better.
  We know $\mathcal{Z}_{i - 1}(i + 1)$
  is generated by
  $\mathcal{Z}_{i - 1}(i - 1)$, $J_i, J_{i + 1}$, and
  $(i, i + 1)$,
  and we know that 
  $V_{P, i}$ is an irreducible representation
  for $\mathcal{Z}_{i - 1}(i + 1)$. Note
  that the elements in $\mathcal{Z}_{i - 1}(i - 1)$
  act as scalars, so we only need
  to worry about $J_i, J_{i + 1}$,
  and $(i, i + 1)$.
\end{remark}

\begin{lemma}
  We have the following relations:
  \begin{enumerate}
    \item $J_i J_{i + 1} = J_{i + 1} J_i$;
    \item $(i, i + 1)^2 = 1$;
    \item $(i, i + 1) J_i = J_{i + 1} (i, i + 1) - 1$.
  \end{enumerate}
\end{lemma}

\begin{proof}
  We already know $(1)$ and $(2)$. For
  \[
    (i, i + 1) J_i (i, i + 1)
    = \sum_{j = 1}^{i - 1} (j, i + 1)
    = J_{i + 1} - (i, i + 1),
  \]
  which becomes $(3)$ after right-multiplying
  by $(i, i + 1)$.
\end{proof}

\begin{definition}
  Define the \emph{degenerate affine Hecke algebra}
  $\HH(2)$ to be the algebra
  with generators $X_1, X_2, T$ and relations
  $X_1 X_2 = X_2 X_1$,
  $T^2 = 1$, and
  $T X_1 = X_2 T - 1$ (equivalently,
  $X_1 T = T X_2 - 1$).
\end{definition}

\begin{remark}
  There is a unique homomorphism
  $\HH(2) \to \mathcal{Z}_{i - 1}(i + 1)$
  given by
  \[
    X_1 \mapsto J_i, \quad
    X_2 \mapsto J_{i + 1}, \quad
    T \mapsto (i, i + 1).
  \]
\end{remark}

\begin{corollary}
  Let $M$ be an irreducible module
  for $\mathcal{Z}_{i - 1}(i + 1)$.
  Then $M$ stays irreducible as an
  $\HH(2)$-module.
\end{corollary}

\begin{proof}
  Note that $\mathcal{Z}_{i - 1}(i - 1)$
  is the central subalgebra of
  $\mathcal{Z}_{i - 1}(i + 1)$. Any element
  of the center acts as a scalar
  on an irreducible $\mathcal{Z}_{i - 1}(i + 1)$-module,
  so a subspace invariant under
  $\mathcal{Z}_{i - 1}(i + 1)$ is
  also invariant under $\HH(2)$. This
  proves the claim.
\end{proof}

\begin{remark}
  A basis of $\HH(2)$ is
  given by
  $\{X_1^{d_1} X_2^{d_2} \sigma : \sigma \in \{1, T\}\}$.
\end{remark}

\begin{remark}
  One can generalize this construction
  to $\mathcal{Z}_{i}(d)$ to get
  $\HH(d)$, with
  generators $X_1, \ldots, X_d$ and
  $T_1, \ldots, T_{d - 1}$, with similar
  relations.
\end{remark}

\begin{example}\label{ex:hh2-module}
  We consider finite-dimensional
  irreps of $\HH(2)$. Note that
  $X_1, X_2$ commute, so they have a
  common eigenvector $m \in M$. So
  $X_1 m = a m$ and $X_2 m = b m$
  for $a, b \in \C$. We have two cases:
  \begin{enumerate}
    \item $Tm \sim m$. Since $T^2 = 1$,
      we have two options:
      \begin{enumerate}
        \item $Tm = m$. Then
          $T X_1 m = a m$, and applying
          $T X_1 = X_2 T - 1$ to $m$, we get
          \[
            (X_2 T - 1) m = (b - 1) m.
          \]
          Thus we must have $b = a + 1$.
        \item $Tm = -m$. Then
          one can check that $b = a - 1$
          as an exercise.
      \end{enumerate}
    \item $m, Tm$ are linearly independent.
      Then
      \begin{align*}
        X_1 (Tm) &= (T X_2 - 1)m
        = b(Tm) - m, \\
        X_2 (Tm) &= (T X_1 + 1)m
        = a(Tm) + m.
      \end{align*}
      In particular, $\Span\{m, Tm\}$
      is stable under $\HH(2)$.
      Since $M$ is irreducible,
      $\{m, Tm\}$ is a basis of $M$.
      In this case, one can check that
      \[
        T \mapsto
        \begin{pmatrix}
          0 & 1 \\
          1 & 0
        \end{pmatrix}, \quad
        X_1 \mapsto
        \begin{pmatrix}
          a & 0 \\
          -1 & b
        \end{pmatrix}, \quad
        X_2 \mapsto
        \begin{pmatrix}
          b & 0 \\
          1 & a
        \end{pmatrix}
      \]
      defines an $\HH(2)$-module on
      $\C^2$, denoted as $M(a, b)$.
  \end{enumerate}
\end{example}

\begin{lemma}\label{lem:hh2-irreps}
  $M(a, b)$ is irreducible if and only if
  $a \ne b \pm 1$. If
  $a \ne b \pm 1$, then
  $M(a, b) \cong M(a', b')$ if and only
  if $(a, b) = (a', b')$ or
  $(b, a) = (a', b')$.
\end{lemma}
