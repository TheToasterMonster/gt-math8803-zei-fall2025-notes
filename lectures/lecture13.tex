\chapter{Oct.~1 --- Lie Algebras}

\section{Commutator Structure}

\begin{remark}
  Recall that for small enough
  $x, y \in \g = T_1 G$, we have
  \[
    \exp(x) \exp(y)
    = \exp(\mu(x, y)), \quad
    \mu(x, y) \in \g.
  \]
\end{remark}

\begin{lemma}
  The Taylor series for $\mu(x, y)$
  is given by
  \[
    \mu(x, y) = x + y + \lambda(x, y)
    + \dots,
  \]
  where $\dots$ stands for
  higher order terms in $x, y$, and
  $\lambda : \g \times \g \to \g$
  is bilinear and skew-symmetric.
\end{lemma}

\begin{proof}
  We can write
  $\mu(x, y) = \alpha_1(x) + \alpha_2(y) + Q_1(x) + Q_2(y) + \lambda(x, y) + \dots$,
  where $\alpha_1, \alpha_2$ are
  linear maps, $Q_1, Q_2$ are quadratic forms,
  and $\lambda$ is bilinear.
  Setting $y = 0$ gives
  $\mu(x, 0) = x$, so
  $\alpha_1 = x$ and $Q_1 = 0$. Similarly,
  setting $x = 0$ gives $\alpha_2 = y$
  and $Q_2 = 0$. So it suffices to
  show that $\lambda$ is skew-symmetric.
  To see this, note that
  $\exp(x) \exp(x) = \exp(2x)$, so
  $\lambda(x, x) = 0$, which implies
  $\lambda$ is skew-symmetric.
\end{proof}

\begin{definition}
  A \emph{commutator} of
  two elements $x, y \in \g$ is
  $[x, y] = 2\lambda(x, y)$.
\end{definition}

\begin{prop}
  We have the following:
  \begin{enumerate}
    \item Let $\varphi : G_1 \to G_2$
      be a morphism of (real or complex)
      Lie groups and $T_1 \varphi : \g_1 \to \g_2$.
      Then
      \[
        T_1 \varphi[x, y]
        = [T_1 \varphi x, T_1 \varphi y]
        \quad \text{for all $x, y \in \g_1$}.
      \]
    \item The adjoint action of
      $G$ on $\g$ satisfies
      $\Ad_g[x, y] = [\Ad_g x, \Ad_g y]$.
    \item $\exp(x) \exp(y) \exp(-x) \exp(-y) = \exp([x, y] + \dots)$,
      where $\dots$ denotes
      higher order terms in $x, y$.
  \end{enumerate}
\end{prop}

\begin{proof}
  (1) This follows from the fact that
  morphisms ``commute'' with the
  exponential map.

  (2) Apply (1) to the conjugation
  morphism $\varphi(h) = g h g^{-1}$.
\end{proof}

\begin{corollary}
  If $G$ is a commutative Lie group,
  then $[x, y] = 0$ for all $x, y \in \g$.
\end{corollary}

\begin{example}
  Consider a Lie subgroup
  $G \subseteq \GL(n, \K)$.
  Then $[x, y] = xy - yx$
  (expand $\log(e^x e^y)$).
\end{example}

\begin{remark}
  Consider $[\cdot, \cdot] : \g \times \g \to \g$
  and associate to
  a morphism $\varphi$ of Lie groups
  to a morphism $T_1 \varphi$
  of $\g$. Note that there is a
  representation
  $\Ad : G \to \GL(\g)$ given by
  $g \mapsto \Ad_g$.
\end{remark}

\begin{lemma}
  $\ad = T_1{\Ad} : \g \to \gl(\g)$
  is a map of tangent spaces
  satisfying
  \begin{enumerate}
    \item $\ad_x y = [x, y]$,
    \item $\Ad_{\exp(x)} = \exp(\ad_x)$.
  \end{enumerate}
\end{lemma}

\begin{proof}
  By definition, we have
  \[
    \Ad_g y
    = \left.\frac{d}{dt}\right|_{t = 0}
      g \exp(ty) g^{-1}.
  \]
  Then we can write
  \begin{align*}
    \ad_x y
    &= \left.\frac{d}{ds}\right|_{s = 0}
      \left.\frac{d}{dt}\right|_{t = 0}
        \exp(sx) \exp(ty) \exp(-sx) \\
    &= \left.\frac{d}{ds}\right|_{s = 0}
    \left.\frac{d}{dt}\right|_{t = 0}
      \exp(ty + ts[x, y] + \dots)
    = [x, y],
  \end{align*}
  which proves (1). Then
  (2) follows since
  $\Ad$ is a morphism of Lie groups.
\end{proof}

\section{Lie Algebras}
\begin{example}
  For matrices, we have
  $e^x A e^{-x} = e^{\ad_x} A$,
  where $\ad_x = [x, \cdot]$.
\end{example}

\begin{theorem}\label{thm:jacobi}
  Let $G$ be a (real or complex)
  Lie group, $\g = T_1 G$, and
  $[\cdot, \cdot] : \g \times \g \to \g$
  the commutator.
  Then $[\cdot, \cdot]$ satisfies the
  following (equivalent) versions of
  the \emph{Jacobi identity}:
  \begin{enumerate}
    \item $[x, [y, z]] = [[x, y], z] + [y, [x, z]]$,
    \item $[x, [y, z]] + [y, [z, x]] + [z, [x, y]] = 0$,
    \item $\ad_x [y, z] = [\ad_x y, z] + [y, \ad_x z]$,
    \item $\ad_{[x, y]} = \ad_x \ad_y - \ad_y \ad_x$.
  \end{enumerate}
\end{theorem}

\begin{proof}
  These are clearly all equivalent,
  so it suffices to prove (4).
  Let $\Ad : G \to \GL(\g)$ and
  note that
  $\ad : \g \to \gl(\g)$ preserves
  the commutator. In $\gl(\g)$, we have
  $[A, B] = AB - BA$, so
  \[
    \ad_{[x, y]}
    = \ad_x \ad_y - \ad_y \ad_x.
  \]
  This proves the identity (4).
\end{proof}

\begin{definition}
  A \emph{Lie algebra} over a field $\K$
  is a vector space $\g$ with a
  bilinear map $[\cdot, \cdot] : \g \times \g \to \g$
  which is skew-symmetric and
  satisfies the Jacobi identity.
\end{definition}

\begin{example}
  Any vector space has a structure of a
  Lie algebra on it by $[v, v] = 0$.
  This is called the \emph{abelian Lie algebra}.
\end{example}

\begin{example}
  Any associative algebra over $\K$
  can be made into Lie algebra
  by $[x, y] = xy - yx$.
\end{example}

\begin{theorem}
  Let $G$ be a (real or complex)
  Lie group. Then $\g = T_1 G$ has a
  canonical structure of a Lie algebra
  with commutator defined as
  $2\lambda(x, y)$. We sometimes write
  $\g = \Lie(G)$.
  Moreover, every morphism
  of Lie groups $\varphi : G_1 \to G_2$
  induces a morphism of Lie algebras
  $\varphi_* : \g_1 \to \g_2$.
  If $G$ is connected, then the
  map $\Hom(G_1, G_2) \to \Hom(\g_1, \g_2)$
  by $\varphi \mapsto \varphi_*$
  is injective.
\end{theorem}

\begin{definition}
  Let $\g$ be a Lie algebra over $\K$.
  A subspace $\mathfrak{h} \subseteq \g$
  is called a \emph{Lie subalgebra}
  if it is closed under the commutator,
  and $\mathfrak{h} \subseteq \g$
  is called an \emph{ideal} if
  $[x, y] \in \mathfrak{h}$
  for all $x \in \g$ and
  $y \in \mathfrak{h}$.
\end{definition}

\begin{corollary}
  If $\mathfrak{h}$ is an ideal in
  $\g$, then
  $\g / \mathfrak{h}$ has a canonical
  structure of a Lie algebra.
\end{corollary}

\begin{theorem}
  Let $G$ be a (real or complex)
  Lie group and $\g = \Lie(G)$.
  \begin{enumerate}
    \item Let $H$ be a subgroup in $G$
      (not necessarily a closed Lie
      subgroup). Then
      $\mathfrak{h} = T_1 H$
      is a Lie subalgebra in $\g$.
    \item Let $H$ be a normal closed
      Lie subgroup in $G$. Then
      $\mathfrak{h} = T_1 H$ is an ideal
      in $\g$ and
      $\Lie(G / H) = \g / \mathfrak{h}$.
      Conversely, if $H$ is a closed
      Lie subgroup such that
      $H, G$ are connected and
      $\mathfrak{h} = T_1 H$ is an ideal,
      then $H$ is normal.
  \end{enumerate}
\end{theorem}

\begin{proof}
  (1) If $x \in T_1 H$, then we have
  $\exp(tx) \in H$ for all $t \in \K$.
  Then using $\lambda(x, y)$ as the
  commutator implies that
  $[x, y] \in \mathfrak{h}$ for
  $x, y \in \mathfrak{h}$.

  (2) If $H$ is a normal closed
  Lie subgroup, then we have
  \[
    \exp(x) \exp(y) \exp(-x) \in H
  \]
  for all $x \in \g$ and $y \in \mathfrak{h}$.
  So $[x, y] \in \mathfrak{h}$, i.e.
  $\mathfrak{h}$ is an ideal.
  If $\mathfrak{h}$ is an ideal, then
  \[
    \Ad_{\exp(x)} \mathfrak{h}
    \subseteq \mathfrak{h}
    \quad \text{for all $x \in \g$}
  \]
  since $\Ad_{\exp(x)} = \exp(\ad_x)$.
  Since $g \exp(y) g^{-1} = \exp(\Ad_g y)$
  for all $y \in \mathfrak{h}$
  and $g \in G$, we have
  \[g \exp(y) g^{-1} \in H,\]
  i.e. we have
  $g h g^{-1} \in H$ for all $h \in H$.
  Thus $H$ is normal.
\end{proof}

\section{The Lie Algebra of Diffeomorphisms}

\begin{definition}
  Let $M$ be a manifold. Then
  $\Diff(M)$ is the
  \emph{group of diffeomorphisms}
  of $M$.
\end{definition}

\begin{remark}
  Note that $\Diff(M)$ is
  \emph{not} a Lie group, since it
  is infinite-dimensional. However,
  we can still think of a ``Lie algebra''
  in this setting. Let
  $\varphi^t : M \to M$ be a 1-parameter
  family of diffeomorphisms.
  Then $\phi^t(m)$ for $m \in M$
  defines a curve in $M$.
  Taking its derivative, we have
  \[
    \left.\frac{d}{dt}\right|_{t = 0} \varphi^t(m)
      \in T_m M.
  \]
  If we look at all $m \in M$, we
  get a vector field $\left.\frac{d}{dt}\right|_{t = 0} \varphi^t$ on $M$.
\end{remark}

\begin{definition}
  Define
  the \emph{Lie algebra of diffeomorphisms} to be
  $\Lie(\Diff(M)) = \Vect(M)$.
\end{definition}

\begin{remark}
  For $\xi \in \Vect(M)$,
  $\exp(t\xi)$ generates a 1-parameter
  family of diffeomorphisms with
  derivative $\xi$ at $t = 0$. So
  we get a differential equation
  \[
    \left.\frac{d}{dt}\right|_{t = 0} \varphi^t(m)
      = \xi(m),
  \]
  which defines a time flow $\Phi^t$
  for $\xi$.
  Then we can define 
  $\exp(t \xi) = \Phi^t_{\xi}$.
\end{remark}

\begin{prop}\label{prop:commutator}
  We have the following:
  \begin{enumerate}
    \item Let $\xi, \eta \in \Vect(M)$.
      There exists a unique vector
      field $[\xi, \eta]$ such that
      \[
        \Phi^t_{\xi}
        \Phi^s_{\eta}
        \Phi^{t}_{-\xi}
        \Phi^{s}_{-\eta}
      = \Phi^{ts}_{[\xi, \eta]} + \dots.
      \]
    \item The commutator defines a
      structure of a Lie algebra on
      $\Vect(M)$.
    \item $[\xi, \eta] = \left.\frac{d}{dt}\right|_{t = 0} (\Phi^t_{\xi})_* \eta$,
        and $\partial_{[\xi, \eta]} f = (\partial_\eta \partial_\xi - \partial_\xi \partial_\eta) f$
        satisfies
        \[
          \left[\sum_i f_i \partial_i, \sum_i g_i \partial_i\right]
          = \sum_{i, j} (g_i \partial_i f_j - f_i \partial_i g_j) \partial_j.
        \]
  \end{enumerate}
\end{prop}

\begin{remark}
  The minus sign in (3)
  is since $\Phi : M \to M$ acts on $f$
  by $(\Phi f)(m) = f(\Phi^{-1}(m))$.
\end{remark}

\begin{example}
  For $x \mapsto x + t$, we have
  $\Phi^t f(x) = f(x - t)$ and
  \[
    \partial_x f
    = -\left.\frac{d}{dt}\right|_{t = 0} \Phi^t f.
  \]
\end{example}

\begin{theorem}
  Let $G$ be a finite-dimensional
  Lie group action on $M$ and let
  $\rho : G \to \Diff(M)$.
  \begin{enumerate}
    \item This action defines a
      linear map $\rho_* : \g \to \Vect(M)$.
    \item $\rho_*[x, y] = [\rho_* x, \rho_* y]$, where
      the right-hand side is the commutator
      of vector fields.
  \end{enumerate}
\end{theorem}

\begin{example}
  Let $\GL(n, \R)$ act on $\R^n$.
  Let $a \in \gl(n, \R)$ and
  $\Phi^t_a = e^{ta}$. Then for
  $\vec{x} \in \R^n$,
  \[
    \left. \frac{d}{dt}\right|_{t = 0}
      \Phi^t_a f(\vec{x})
      = \left. \frac{d}{dt}\right|_{t = 0}
        f(e^{-ta} \vec{x})
      = \left. \frac{d}{dt}\right|_{t = 0}
        f(\vec{x} - t a \vec{x} + \dots)
        = - \sum_{i, j} a_{i, j} x_j \partial_{x_i} f(\vec{x}).
  \]
  Check as an exercise that
  $\rho_*$ maps
  $a \mapsto - a_{i, j} x_j \partial_{x_i}$
  for matrices, which matches
  the above.
\end{example}
