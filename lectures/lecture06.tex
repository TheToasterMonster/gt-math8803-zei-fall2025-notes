\chapter{Sept.~8 --- Representations of \texorpdfstring{$S_n$}{Sn}}

\section{Motivation for Studying \texorpdfstring{$S_n$}{Sn} and Summary}

\begin{remark}
  The finite \emph{simple} groups
  (those with no nontrivial normal subgroups)
  are classified as follows:
  \begin{enumerate}
    \item abelian groups: cyclic groups of
      finite order;
    \item alternating groups:
      $U_n \subseteq S_n$ (the subgroup of
      even permutations) for $n \ge 5$;
    \item $26$ exceptional finite simple
      groups;
    \item finite simple groups of
      \emph{Lie type} (analogues of
      Lie groups for finite fields).
  \end{enumerate}
  The final parts of the classification
  were done by
  Gorenstein (1960--1980s) and
  Aschbacher-Smith (2004).
\end{remark}
  
\begin{remark}
  We study $S_n$ because it is
  easier to work with than directly
  studying $U_n$, and we can recover
  representations of $U_n$
  from those of $S_n$ via
  Frobenius reciprocity.
\end{remark}

\begin{remark}
  We have previously seen the following
  using our abstract theory:
  \begin{enumerate}
    \item Representations of $S_n$
      are the same as representations of
      $\C S_n$.
    \item The algebra $\C S_n$ is
      semisimple:
      $\C S_n \cong \bigoplus_V \End_\C(V)$,
      where $V$ runs over the isomorphism
      classes of irreps of $S_n$.
    \item The number of irreps of $S_n$
      (up to isomorphism) coincides with
      the number of conjugacy classes.
  \end{enumerate}
\end{remark}

\begin{remark}
  In the case of $S_n$, the conjugacy
  classes are enumerated by
  partitions of $n$:
  \[
    (n_1, n_2, \dots, n_k), \quad
    n_1 \ge n_2 \ge \cdots \ge n_k.
  \]
  We can write repeated parts via
  $(m_1^{d_1}, \dots, m_e^{d_e})$, where
  $m_1 > m_2 > \cdots > m_e$. So for
  $S_6$, we have
  \[
    (2, 2, 1, 1) \longleftrightarrow
    (2^2, 1^2).
  \]
\end{remark}

\section{The Inductive Approach: Background}

\begin{remark}
  We will follow the \emph{inductive approach},
  due to Okounkov-Vershik. Consider
  the inclusions
  \[
    \{1\} = S_1 \subseteq S_2
    \subseteq \cdots \subseteq S_{n - 1} \subseteq S_n.
  \]
  Note that if $H \subseteq G$ are finite groups,
  then an irrep of $\C G$ decomposes
  into irreps of $\C H$.

  In general, if $B \subseteq A$ are
  finite-dimensional associative algebras
  and $\tau : B \to A$ is a homomorphism,
  then any $A$-module is also a $B$-module
  by the homomorphism $\tau$.
  We have isomorphisms
  \begin{align*}
    A &\overset{\cong}{\longrightarrow} \bigoplus_{V \in \Irr(A)} \End_\C(V), \\
    B &\overset{\cong}{\longrightarrow} \bigoplus_{U \in \Irr(B)} \End_\C(U).
  \end{align*}
  Let $M_{V, U} = \Hom_B(U, V)$ be
  multiplicity spaces. Then there is a
  $B$-linear isomorphism
  \begin{align*}
    \bigoplus_i U_i \otimes M_{V, U_i}
    &\overset{\cong}{\longrightarrow} V \\
    \sum_i u_i \otimes \varphi_i
    &\longmapsto \sum_i \varphi_i(u_i).
  \end{align*}
  We can compute $M_{V, U}$ from
  an algebraic perspective.
\end{remark}

\begin{definition}
  Define the \emph{centralizer}
  of $B$ in $A$ to be
  \[
    \mathcal{Z}_B(A)
    = \{a \in A : a \tau(b) = \tau(b) a \text{ for all } b \in B\}.
  \]
\end{definition}

\begin{exercise}
  Prove the following:
  \begin{enumerate}
    \item $\ZZ_A(A) = \ZZ(A)$.
    \item $\ZZ_B(A)$ is a subalgebra of $A$.
  \end{enumerate}
\end{exercise}

\begin{lemma}
  There is an isomorphism
  $\ZZ_B(A) \cong \bigoplus_{U, V} \End(M_{V, U})$,
  with $U, V$ such that $M_{V, U} \ne 0$.
\end{lemma}

\begin{proof}
  We have the isomorphism
  \[
    A \overset{\cong}{\longrightarrow}
    \bigoplus_V \End(V),
  \]
  and we can view $\tau : B \to A$ as
  $(\tau_V)_{V \in \Irr(A)}$, where
  $\tau_V : B \to \End(V)$. Similarly,
  we can
  view an element $a \in A$ as $(a_V) \in \bigoplus_V \End(V)$.
  Then $a \in \ZZ_B(A)$ if and only if
  $a_V \in \ZZ_B(\End(V))$ for all $V$, so
  \[
    \ZZ_B(A)
    = \bigoplus_V \ZZ_B(\End(V)).
  \]
  Then
  $\ZZ_B(\End(V)) \cong \End_B(V) \cong \bigoplus_U \End(M_{V, U})$,
  which completes the proof.
\end{proof}

\begin{remark}
  Show that the following actions of
  $\ZZ_B(A)$ on
  $\End(M_{V, U}) = \Hom_B(U, V)$
  are the same:
  \begin{enumerate}
    \item $\End_B(V)$ acts on
      $\Hom_B(U, V)$ via
      \begin{align*}
        \End_B(V) \times \Hom_B(U, V)
        &\longrightarrow \Hom_B(U, V) \\
        (\alpha, \varphi) &\longmapsto \alpha \circ \varphi;
      \end{align*}
    \item for $z \in \ZZ_B(A)$,
      $\varphi \in \Hom_B(U, V)$,
      we can define $z \varphi \in \Hom_B(U, V)$
      by
      \[
        [z \varphi](u) = z \varphi(u),
      \]
      where the right-hand side is
      the $A$-action on $V$.
  \end{enumerate}
\end{remark}

\begin{corollary}
  The following conditions are equivalent:
  \begin{enumerate}
    \item for all $U \in \Irr(B)$
      and $V \in \Irr(A)$, we have
      $\dim \Hom_B(U, V) \le 1$;
    \item $\ZZ_B(A)$ is commutative.
  \end{enumerate}
\end{corollary}

\begin{proof}
  $\ZZ_B(A) = \bigoplus_{U, V} \End(M_{V, U})$ is
  commutative if and only if
  $\End(M_{V, U})$ has dimension
  $1$ or $0$.
\end{proof}

\begin{example}
  Let $A = \Mat_4(\C) \oplus \Mat_3(\C)$
  and $B = \Mat_2(\C) \oplus \C^{\oplus 2}$.
  Define $\tau : B \to A$ by
  \[
    \tau(x_1, x_2, x_3)
    = (\diag(x_1, x_2, x_2), \diag(x_1, x_3)), \quad
    x_1 \in \Mat_2(\C), x_2, x_3 \in \C.
  \]
  We have $B$-modules
  $U_1, U_2, U_3$ of dimensions
  $2, 1, 1$ and $A$-modules
  $V_1, V_2$ of dimensions $4, 3$. Note
  that $M_{V_1, U_2}$ is $2$-dimensional,
  and $M_{V_1, U_1}$, $M_{V_2, U_1}$,
  $M_{V_2, U_3}$ are $1$-dimensional.
  So far, we have
  \[
    \ZZ_B(A) \cong \Mat_2(\C) \oplus \C^{\oplus 3}.
  \]
  To verify this directly, we know that
  $\ZZ_B(A)$ consists of pairs
  $(y_1, y_2) \in \Mat_4(\C) \oplus \Mat_3(\C)$
  such that $y_1$ commutes with
  $\diag(x_1, x_2, x_2)$ and $y_2$
  commutes with $\diag(x_1, x_3)$. So
  \[
    y_1 =
    \begin{pmatrix}
      a & 0 & 0 & 0 \\
      0 & a & 0 & 0 \\
      0 & 0 & b & c \\
      0 & 0 & d & e
    \end{pmatrix}, \quad
    y_2 =
    \begin{pmatrix}
      f & 0 & 0 \\
      0 & f & 0 \\
      0 & 0 & g
    \end{pmatrix}.
  \]
  So $\ZZ_B(A)$ is parametrized by the
  $2 \times 2$ matrix and the $3$ scalars
  $a, f, g$.
\end{example}

\section{The Inductive Approach: Representations of \texorpdfstring{$S_n$}{Sn}}

\begin{remark}
  Let $S_m \subseteq S_n$ for
  $m < n$, and let
  $\ZZ_m(n)$ be the corresponding
  centralizer for group algebras.
\end{remark}

\begin{lemma}
  Let $H \subseteq G$ be finite groups.
  Then $\Z_{\C H}(\C G) \subseteq \C G$
  consists of elements of the form
  $\sum_{g \in G} a_g g$ such that
  $a_{h g h^{-1}} = a_g$ for all
  $h \in H$. In particular,
  $\ZZ_{\C H}(\C G)$ has a basis indexed
  by the $H$-conjugacy classes in $G$,
  given by (for a conjugacy class $C$)
  \[
    C \longmapsto b_C =
    \sum_{g \in C} g \in \ZZ_{\C H}(\C G).
  \]
\end{lemma}

\begin{example}
  Note that for $\C S_m \subseteq \C S_n$,
  conjugation permutes the first $m$
  elements. For example, for
  $S_3 \subseteq S_6$, we can
  write a conjugacy class as
  $({*}\ {*}\ 4)(5\ {*})(6)$, which
  contains elements like
  $(1\ 2\ 4)(5\ 3)$ and
  $(2\ 3\ 4)(5\ 1)$.
  For $m = n - 1$, consider the conjugacy
  class $({*}\ n)$, which consists of
  \[
    (1\ n), \quad (2\ n), \quad \ldots, \quad (n - 1\ n).
  \]
  Then the basis element $b_{({*}\, n)}$
  (called the $n$th \emph{Jucys-Murphy element}) is given by
  \[
    b_{({*}\, n)} = \sum_{i = 1}^{n - 1} (i\ n).
  \]
\end{example}

\begin{remark}
  We will now determine
  algebra generators of
  $\ZZ_m(n)$. It contains
  \begin{enumerate}
    \item $\ZZ_m(m)$: the center of
      $\C S_m$;
    \item $S_{[m + 1, n]}$: the subgroup
      of $S_n$ containing permutations
      fixing $1, \dots, m$;
    \item $J_k = \sum_{i = 1}^{k - 1} (i\ k)$
      for $k = m + 1, \dots, n$.
  \end{enumerate}
  Note that $J_{m + 1}, \dots, J_n$
  pairwise commute (check this as
  an exercise).
\end{remark}

\begin{theorem}
  The algebra $\ZZ_m(n)$ is generated by
  the subalgebras $\ZZ_m(m)$,
  $\C S_{[m + 1, n]}$, and the elements
  $J_{m + 1}, \dots, J_n$.
\end{theorem}
