\chapter{Oct.~13 --- Fundamental Theorems of Lie Theory}

\section{Fundamental Theorems of Lie Theory}
\begin{remark}
  We know the following about
  Lie theory so far:
  \begin{enumerate}
    \item Every (real or complex) Lie
      group $G$ defines a Lie algebra
      $\g = \Lie(G)$, and every morphism
      $\varphi : G_1 \to G_2$ gives a
      morphism $\varphi_* : \g_1 \to \g_2$.
      The map $\Hom(G_1, G_2) \to \Hom(\g_1, \g_2)$
      is injective.
    \item Every Lie subgroup $H \subseteq G$
      defines a Lie subalgebra
      $\mathfrak{h} \subseteq \g$.
    \item The group law can be recovered
      from $[\cdot, \cdot]$ on $\g$.
  \end{enumerate}
  Now we want to understand the following:
  \begin{enumerate}
    \item Given a morphism
      $\Lie(G_1) = \g_1 \to \g_2 = \Lie(G_2)$,
      can we lift it to a morphism
      $G_1 \to G_2$?

      We will see that the answer is
      no in general.
    \item Given a Lie subalgebra
      $\mathfrak{h} \subseteq \g$,
      does there always exist a subgroup
      $H$ such that $\mathfrak{h} = \Lie(H)$?

      We will see that the answer is
      yes.
    \item Can every finite-dimensional
      Lie algebra be obtained as
      the Lie algebra of a Lie group?
  \end{enumerate}
\end{remark}

\begin{example}
  Let $G = S^1 = \R / \Z$ and
  $G_2 = \R$. Then $\g_1 = \g_2 = \R$.
  Consider the identity map
  \[\id : \g_1 \to \g_2.\]
  If this lifted to a morphism $G_1 \to G_2$, then
  we must have $\theta \mapsto \theta$.
  But $f(\Z) = 0$, so this is impossible.
  Thus morphisms
  $\g_1 \to \g_2$ cannot always be lifted
  to morphisms $G_1 \to G_2$ in general.
\end{example}

\begin{theorem}
  For any (real or complex) Lie group
  $G$, there is a bijection between
  connected Lie subgroups $H \subseteq G$
  and Lie subalgebras
  $\mathfrak{h} \subseteq \g$ given by
  $H \mapsto \mathfrak{h} = \Lie(H) = T_1 H$.
\end{theorem}

\begin{theorem}
  If $G_1, G_2$ are Lie groups and
  $G_1$ is simply connected, then
  \[
    \Hom(\g_1, \g_2) = \Hom(G_1, G_2),
  \]
  where $\g_i = \Lie(G_i)$ for $i = 1, 2$.
\end{theorem}

\begin{theorem}\label{thm:lie-3rd}
  Any finite-dimensional (real or complex)
  Lie algebra is isomorphic to the
  Lie algebra of a (real or complex)
  Lie group.
\end{theorem}

\begin{corollary}
  For any (real or complex)
  finite-dimensional Lie algebra
  $\g$, there exists a unique (up to
  isomorphism) connected and simply
  connected (real or complex) Lie group $G$
  with $\Lie(G) = \g$. Any other connected
  Lie group $G'$ with Lie algebra
  $\g$ must be of the form
  $G / Z'$ for some discrete central
  subgroup $Z' \subseteq Z \subseteq G$.
\end{corollary}

\begin{proof}
  We sketch the proof. Theorem
  \ref{thm:lie-3rd} says that there is a
  Lie group $\widetilde{G}$ with $\Lie(\widetilde{G}) = \g$.
  Take a neighborhood of the identity, and
  construct the universal cover to get
  $G$. Then for any other
  $G'$, there is a covering map
  $G \to G'$. One can check that
  the deck transformations
  are a normal subgroup and
  that they form a
  subgroup of the center $Z$.
  Then we may take $Z' = \pi_1(G')$.
\end{proof}

\begin{corollary}
  The categories of simply connected
  finite-dimensional Lie groups and
  Lie algebras are equivalent.
\end{corollary}

\section{Complex and Real Forms}

\begin{definition}
  Let $\g$ be a real Lie algebra.
  Its \emph{complexification} is
  \[
    \g_\C = \g \otimes_\R \C
    = \g \oplus i \g,
  \]
  where the commutator extends naturally
  from $\g$ to $\g_\C$.
\end{definition}

\begin{example}
  Consider the following:
  \begin{enumerate}
    \item For $\g = \mathfrak{sl}(n, \R)$,
      $\g_\C = \mathfrak{sl}(n, \C)$.
    \item For $\g = \mathfrak{u}(n)$
      (the skew-Hermitian matrices),
      $\g_\C = \gl(n, \C)$ as
      $i \g$ are the Hermitian
      matrices.
  \end{enumerate}
\end{example}

\begin{definition}
  Let $G$ be a connected complex Lie
  group and $\g = \Lie(G)$. If
  $K \subseteq G$ be a closed real Lie
  subgroup such that $\mathfrak{k} = \Lie(K)$
  is a \emph{real form} of $\g$ (meaning
  that $\mathfrak{k}_\C = \g$), then
  $K$ is called a \emph{real form} of $G$.
\end{definition}

\begin{remark}
  It can be shown that if
  $\g = \Lie(G)$ where $G$ is simply
  connected and complex, then for any
  form $\mathfrak{k} \subseteq \g$,
  one can obtain a real form
  $K \subseteq G$ with
  $\Lie(K) = \mathfrak{k}$.

  It is not true that one can represent
  any real Lie group as a subgroup of some
  complex Lie group, an example is
  the universal cover of $\SL(2, \R)$.
\end{remark}

\begin{example}
  We will study
  $\mathfrak{so}(3, \R)$,
  $\mathfrak{su}(2)$, and
  $\mathfrak{sl}(2, \C)$.
  Recall that $\mathfrak{so}(3, \R)$
  has a basis
  \[
    J_x = \begin{pmatrix}
      0 & 0 & 0 \\
      0 & 0 & -1 \\
      0 & 1 & 0
    \end{pmatrix}, \quad
    J_y = \begin{pmatrix}
      0 & 0 & 1 \\
      0 & 0 & 0 \\
      -1 & 0 & 0
    \end{pmatrix}, \quad
    J_z = \begin{pmatrix}
      0 & -1 & 0 \\
      1 & 0 & 0 \\
      0 & 0 & 0
    \end{pmatrix}
  \]
  with commutator relations
  $[J_{\{ x}, J_y] = J_{z\}}$ (the brackets
  mean that $x, y, z$ can be replaced
  with any cyclic permutations).
  A basis of $\mathfrak{su}(2)$ is
  given by the \emph{Pauli matrices}
  times $i$:
  \[
    i \sigma_1 =
    \begin{pmatrix}
      0 & i \\
      i & 0
    \end{pmatrix}, \quad
    i \sigma_2 =
    \begin{pmatrix}
      0 & 1 \\
      -1 & 0
    \end{pmatrix}, \quad
    i \sigma_3 =
    \begin{pmatrix}
      i & 0 \\
      0 & -i
    \end{pmatrix}
  \]
  with commutator relations
  $[i \sigma_{\{ 1}, i \sigma_2] = -2 i \sigma_{3\}}$.
  Note that there is an isomorphism
  $\mathfrak{su}(2) \cong \mathfrak{so}(3, \R)$
  by $i \sigma_1 \mapsto -2 J_x$,
  $i \sigma_2 \mapsto -2 J_y$,
  $i \sigma_3 \mapsto -2 J_z$.
  A basis for $\mathfrak{sl}(2, \C)$
  is given by
  \[
    e = \begin{pmatrix}
      0 & 1 \\
      0 & 0
    \end{pmatrix}, \quad
    f = \begin{pmatrix}
      0 & 0 \\
      1 & 0
    \end{pmatrix}, \quad
    h = \begin{pmatrix}
      1 & 0 \\
      0 & -1
    \end{pmatrix}
  \]
  with commutation relations
  $[h, e] = 2e$, $[h, f] = -2f$,
  $[e, f] = h$.
  One can check as an exercise that
  \[(\mathfrak{su}(2))_\C \cong (\mathfrak{so}(3, \R))_\C \cong \mathfrak{so}(3, \C) \cong \mathfrak{sl}(2, \C).\]
  The $h$  element will play a similar
  role to the Jucys-Murphy's elements
  in representation theory.
\end{example}

\section{Representations of Lie Groups and Lie Algebras}

\begin{definition}
  Recall the definition of a representation:
  \begin{enumerate}
    \item A \emph{representation} of a
      Lie group $G$ is a vector space $V$
      with a morphism of Lie groups
      \[\rho : G \to \GL(V).\]
    \item A \emph{representation} of 
      of a Lie algebra is a vector space
      $V$ with a morphism of Lie algebras
      \[
        \rho : \g \to \gl(V).
      \]
  \end{enumerate}
  A \emph{morphism} between two
  representations $V, W$ of $G$ is a
  map $f : V \to W$ such that
  \[
    f \rho(g) = \rho(g) f.
  \]
  Similarly one defines \emph{morphisms}
  between representations of Lie algebras.
  Denote the set of morphisms of
  representations by
  $\Hom_G(V, W)$ and $\Hom_\g(V, W)$,
  these are also known as \emph{intertwining
  operators}.
\end{definition}

\begin{theorem}\label{thm:lie-rep-corr}
  Let $G$ be a (real or complex) Lie group
  and $\g = \Lie(G)$. Then
  \begin{enumerate}
    \item Every representation
      $\rho : G \to \GL(V)$ defines a
      representation $\rho_* : \g \to \gl(V)$,
      and every morphism of representations
      of $G$ is a morphism of representations
      of $\g$.
    \item If $G$ is simply-connected and
      connected, then $\rho \mapsto \rho_*$
      gives an equivalence of the
      categories of representations
      of $G$ and representations of $\g$.
      In particular, every representation
      of $\g$ can be lifted to a
      representation of $G$.
  \end{enumerate}
\end{theorem}

\begin{remark}
  Note the following:
  \begin{enumerate}
    \item The representations of
      $\mathfrak{su}(2)$ is the same
      set as the representations
      of $\SU(2)$ (since $\SU(2) \cong S^3$
      is simply-connected).
    \item For $G$ simply-connected,
      Theorem \ref{thm:lie-rep-corr}
      can be used to describe representations
      of $\widetilde{G} = G / \widetilde{Z}$
      for $\widetilde{Z} \subseteq Z$
      (where $Z$ is the center of $G$).
      These are the representations of $G$
      such that $\rho(\widetilde{Z}) = \id$.
  \end{enumerate}
\end{remark}

\begin{lemma}\label{lem:real-to-complex-rep}
  Let $\g$ be a real Lie algebra and
  $\g_\C$ its complexification. Then any
  complex representation of $\g$ has a
  unique structure of a representation
  of $\g_\C$. Moreover,
  \[
    \Hom_\g(V, W)
    = \Hom_{\g_\C}(V, W).
  \]
  In particular, the categories of
  complex representations of $\g$ and
  $\g_\C$ are equivalent.
\end{lemma}

\begin{proof}
  Let $\rho : \g \to \gl(V)$ be a
  representation. For $x, y \in \g$, define
  $\rho$ on $x + iy \in \g_\C$ by
  \[
    \rho(x + iy) = \rho(x) + i\rho(y).
  \]
  One can check that this gives the
  desired representation of
  $\g_\C$.
\end{proof}

\begin{example}
  Using
  Lemma \ref{lem:real-to-complex-rep},
  we see that the categories of
  complex representations of
  $\SL(2, \C)$, $\mathfrak{sl}(2, \C)$, $\SU(2)$, $\mathfrak{su}(2)$
  are all equivalent.
\end{example}

\begin{remark}
  We can define the following
  operations on representations:
  \begin{enumerate}
    \item Let $W \subseteq G$ be a
      subrepresentation of $G$ or $\g$
      (meaning that $W$ is invariant).
      Then we can construct the
      \emph{quotient representation}
      $V / W$.
    \item The \emph{dual}, the
      \emph{direct sum}
      and the \emph{tensor product}.
  \end{enumerate}
\end{remark}

\begin{lemma}
  Let $V, W$ be representations of $G$
  (resp. of $\Lie(G) = \g$). There is a
  canonical structure of a representation
  on $V^*$, $V \oplus W$, and
  $V \otimes W$.
\end{lemma}

\begin{proof}
  The direct sum is easy to define.
  For the tensor product, one defines
  \[
    \rho(g)(v \otimes w)
    = (\rho(g)v) \otimes (\rho(g)w)
  \]
  for $g \in G$.
  If $x \in \g$, then we can choose
  a curve $\gamma$ such that
  $\left.\frac{d}{dt}\right|_{t = 0} \gamma(t) = x$
  and $\gamma(0) = 1$,
  and define
  \begin{align*}
    \rho(x)(v \otimes w)
    &= \left.\frac{d}{dt}\right|_{t = 0}
    (\rho(\gamma(t)) v \otimes \rho(\gamma(t)) w) \\
    &= \rho(\dot{\gamma}(0))v \otimes \rho(\gamma(0))w
    + \rho(\gamma(0))v \otimes \rho(\dot{\gamma}(0))w \\
    &= \rho(x)v \otimes w + v \otimes \rho(x)w \\
    &= (\rho(x) \otimes \id + \id \otimes \rho(x))(v \otimes w).
  \end{align*}
  For the dual, let
  $\langle \cdot, \cdot \rangle : V \otimes V^* \to \C$
  be the natural pairing. Then for
  $g \in G$, we need
  \[
    \langle \rho(g) v, \rho(g) v^* \rangle
    = \langle v, v^* \rangle,
  \]
  so we can define
  $\rho_{V^*}(g) = \rho(g^{-1})^T$.
  On the level of Lie algebras, we need
  \[
    \langle \rho(x) v, v^* \rangle
    + \langle v, \rho(x) v^* \rangle = 0,
  \]
  so we can define
  $\rho_{V^*}(x) = -\rho(x)^T$. One can
  check as an exercise that these
  definitions work.
\end{proof}
