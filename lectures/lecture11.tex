\chapter{Sept.~24 --- Lie Groups, Part 2}

\section{More on Lie Groups}

\begin{theorem}
  Let $G$ be a (real or complex) Lie group
  with $\dim G = n$, and
  $H \subseteq G$ a closed Lie subgroup
  with $\dim H = k$. Then the coset
  space $G / H$ has the structure of a
  manifold with dimension $n - k$, such
  that $p : G \to G / H$ is a fiber
  bundle with fibers diffeomorphic
  to $H$. The tangent space at $\overline{1} = p(1)$
  is given by $T_1 G / T_1 H$.
\end{theorem}

\begin{proof}
  Consider $p : G \to G / H$, which
  sends $g \mapsto \overline{g} = p(g)$.
  Note that $gH \subseteq G$ is a
  submanifold (since multiplication by
  $g$ is a diffeomorphism). Choose a
  submanifold $M$ which is transversal
  to $gH$ (i.e. such that
  $T_g G = T_g(gH) \oplus T_g M$).
  Let $U$ be a
  sufficiently small neighborhood of $g$
  so that $UH = \{uh : u \in U, h \in H\}$
  is open in $G$, which exists by the
  inverse function theorem applied to
  the multiplication map
  $U \times H \to G$. Then let
  $\overline{U} = p(U)$. Since
  $p^{-1}(\overline{U}) = U$ is open,
  $\overline{U}$ is an open neighborhood
  of $\overline{g}$ in $G / H$.
  This gives $p : G \to G / H$ the
  natural structure of a fiber bundle.

  For the tangent space, consider
  the map $T_1 p : T_1 G \to T_{\overline{1}} G / H$, and
  note that
  $\ker(T_1 p) = T_1 H$.
\end{proof}

\begin{corollary}
  Let $H$ be a closed Lie subgroup of $G$.
  \begin{enumerate}
    \item If $H$ is connected, then
      the set of connected components
      satisfies $\pi_0(G) = \pi_0(G / H)$.
    \item If $G, H$ are connected, then
      there is an exact sequence
      \[
        \pi_2(G / H)
        \longrightarrow \pi_1(H)
        \longrightarrow \pi_1(G)
        \longrightarrow \pi_1(G / H)
        \longrightarrow \{1\}.
      \]
  \end{enumerate}
\end{corollary}

\begin{remark}
  Often, $\pi_2(G / H)$ and
  $\pi_1(G / H)$ are known, which allows
  us to compute $\pi_1(G)$.
\end{remark}

\begin{example}
  Let $G_1 = \R$ and $G_2 = \R^2 / \Z^2$
  (the torus). Define
  $f : G_1 \to G_2$ by
  \[
    f(t) = (t \Mod{\Z}, \alpha t \Mod{\Z}),
  \]
  for some fixed irrational $\alpha$. Then
  the image of $f$ is everywhere dense
  in $G_2$.
\end{example}

\begin{definition}
  A \emph{Lie subgroup} $H$ in a (real or
  complex)
  Lie group $G$ is an immersed submanifold
  which is also a subgroup.
\end{definition}

\begin{theorem}
  Let $f : G_1 \to G_2$ be a morphism
  (in the real or complex sense). Then
  $H = \ker f$ is a normal closed Lie
  subgroup of $G_1$, and $f$ gives rise
  to an injective map $G_1 / H \to G_2$
  which is an immersion. In particular,
  $\im f$ is a Lie subgroup of $G_2$.
  If $\im f$ is an embedded submanifold,
  then it is a closed Lie subgroup.
  Moreover, $f$ gives an isomorphism
  (of Lie groups)
  $G_1 / H \cong \im f$.
\end{theorem}

\begin{proof}
  We will prove this later using
  Lie algebras.
\end{proof}

\section{Actions of Lie Groups on Manifolds}

\begin{definition}
  An \emph{action} of a real Lie group $G$
  on a real manifold $M$ is an
  assignment
  \[
    g \mapsto \rho(g) \in \Diff(M)
  \]
  with $\rho(1) = \id$ and
  $\rho(g) \rho(h) = \rho(gh)$,
  such that
  the map $G \times M \to M$ by
  $(g, m) \mapsto g . m$ is smooth.
  When $G$ is a complex Lie group and $M$
  is a complex manifold, we require
  $G \times M \to M$ to be analytic.
\end{definition}

\begin{example}
  The following are examples of actions
  on Lie groups:
  \begin{enumerate}
    \item $\GL(n, \R)$ acts on $\R^n$.
    \item $\OO(n, \R)$ acts on
      $S^{n - 1} \subseteq \R^n$.
    \item $\U(n)$ acts on
      $S^{2n - 1} \subseteq \C^n$.
  \end{enumerate}
\end{example}

\begin{definition}
  A \emph{representation} of a (real or
  complex) Lie group $G$ is a vector space
  $V$ (complex if $G$ is complex and real or
  complex if $G$ is real) together with
  a homomorphism $\rho : G \to \GL(V)$.
  If $V$ is finite-dimensional, we require
  $\rho$ to be smooth (or analytic if
  $G$ is complex).

  A \emph{morphism} between two
  representations $\rho_V$ and
  $\rho_W$ is a map
  $f : V \to W$ such that
  it commutes with the $G$-action, i.e.
  $f \rho_V(g) = \rho_W(g) f$ for all
  $g \in G$.
\end{definition}

\begin{remark}
  Any action of $G$ on a manifold $M$
  gives the following infinite-dimensional
  representations:
  \begin{enumerate}
    \item Space of functions
      (the space of analytic functions
      $\mathcal{O}(M)$ in the complex
      case or $C^\infty(M)$ in the real case),
      given by
      $\rho(g) f(m) = f(g^{-1} . m)$.
    \item Vector fields on $M$
      (denoted $\Vect(M)$), given by
      the \emph{pushforward}
      \[(\rho(g) v)(m) = g_* v = T_{g^{-1} . m} (g) (v(g^{-1} . m)).\]
    \item Assume $m$ is a fixed
      point of $G$, i.e. $g . m = m$
      for all $g \in G$. Then $G$ acts
      on $T_m M$ by differentials
      \[
        T_m g : T_m M \to T_m M.
      \]
      This representation is
      finite-dimensional if $\dim M < \infty$.
  \end{enumerate}
\end{remark}

\section{Orbits and Homogeneous Spaces}
\begin{definition}
  Define the \emph{orbit} of a point
  $m \in M$ to be
  \[\Ocal_m = G . m = \{g . m : g \in G\}.\]
  and the \emph{stabilizer} of $m$ to be
  $G_m = \{g \in G : g . m = m\}$.
\end{definition}

\begin{theorem}
  Let $M$ be a manifold with action of
  Lie group $G$ (or complex manifold
  with action of complex $G$). Then
  for all $m \in M$, the stabilizer
  $G_m$ is a closed Lie subgroup of $G$,
  and $g \mapsto g . m$ forms an
  injective immersion
  $G / G_m \hookrightarrow M$ whose
  image coincides with $\Ocal_m$.
\end{theorem}

\begin{corollary}
  The orbit $\Ocal_m$ is an immersed
  submanifold in $M$ with tangent
  space \[T_m \Ocal_m = T_1 G / T_1 G_m.\]
  If $\Ocal_m$ is a submanifold, then
  $g \mapsto g . m$ gives a diffeomorphism
  $G / G_m \to \Ocal_m$.
\end{corollary}

\begin{definition}
  If the action of $G$ on $M$ is
  transitive (i.e. there is just one orbit),
  then we call $M$ a
  \emph{homogeneous space} for $G$.
\end{definition}

\begin{corollary}
  Let $M$ be a $G$-homogeneous space.
  Then the map $G \to M$
  by $g \mapsto g . m$ is a fiber bundle
  over $M$ with fiber $G_m$.
\end{corollary}

\begin{example}
  Consider the following:
  \begin{enumerate}
    \item $\SO(n, \R)$ acting
      on $S^{n - 1} \subseteq \R^n$.
      Then $S^{n - 1}$ is a
      homogeneous space, and the stabilizer
      of any point in $S^{n - 1}$
      (which can be moved to $(1, 0, \dots, 0)$)
      is $\SO(n - 1, \R)$. So we have
      the diagram
      \begin{center}
      \begin{tikzcd}
        \SO(n - 1, \R) \ar[r] &
        \SO(n, \R) \ar[d, "p"] \\
        & S^{n - 1}
      \end{tikzcd}
      \end{center}
    \item $\SU(n)$ acting on
      $S^{2n - 1} \subseteq \C^n$.
      Here we have
      \begin{center}
      \begin{tikzcd}
        \SU(n - 1) \ar[r] &
        \SU(n) \ar[d] \\
        & S^{2n - 1}
      \end{tikzcd}
      \end{center}
  \end{enumerate}
\end{example}

\begin{remark}
  The action of $G$ can be used to
  define a smooth structure on $M$.
  If $M$ is a set with a transitive
  action by $G$, then $M$ is in
  bijection with $G / H$, where
  $H = \Stab_G(m)$. Then $M$ has a natural
  structure of a manifold of dimension
  $\dim G - \dim H$.
\end{remark}

\begin{example}
  A \emph{(full) flag} in $\R^n$ is
  a collection of subspaces
  \[
    \{0\} \subseteq V_1 \subseteq V_2 \subseteq \cdots \subseteq V_n = \R^n,
  \]
  where $\dim V_i = i$. Denote by
  $\mathcal{F}_n(\R)$ the space of
  all flags in $\R^n$. There is an
  action of $\GL(n, \R)$ on
  $\mathcal{F}_n(\R)$.
  We can move any flag to the
  \emph{standard flag}
  \[
    V^{\mathrm{st}}
    = \{0\} \subseteq \langle e_1 \rangle
    \subseteq \langle e_1, e_2 \rangle
    \subseteq \cdots
    \subseteq \langle e_1, \dots, e_n \rangle,
  \]
  which has stabilizer
  $\Stab V^{\mathrm{s, t}} = \B(n, R) \subseteq \GL(n, \R)$,
  the subgroup upper-triangular
  matrices, so
  \[
    \mathcal{F}_n(\R)
    \cong \frac{\GL(n, \R)}{\B(n, \R)}.
  \]
  Now $\dim \B(n, \R) = n(n + 1) / 2$, so
  we can see that
  \[
    \dim \mathcal{F}_n(\R)
    = \dim \GL(n, \R) - \dim \B(n, \R)
    = n^2 - \frac{n(n + 1)}{2}
    = \frac{n(n - 1)}{2}.
  \]
\end{example}

\section{Actions of a Lie Group on Itself}

\begin{remark}
  We can define actions
  $L_g : G \to G$ and
  $R_g : G \to G$ by
  \[
    L_g(h) = gh \quad\text{and}\quad
    R_g(h) = hg^{-1}.
  \]
  There is also an \emph{adjoint action}
  $\Ad_g : G \to G$
  by $\Ad_g(h) = L_g R_g(h) = g h g^{-1}$.

  For $v \in T_m G$, we will
  write $g . v$ for $T_m L_g$
  and $v . g$ for $T_m R_{g^{-1}}$.
\end{remark}

\begin{exercise}
  Check that the above
  agrees with matrix multiplication
  for $G = \GL(n, \R)$.
\end{exercise}

\begin{remark}
  Note that $\Ad_g$ sends
  $1 \mapsto 1$, so there is a
  representation
  $\Ad_g : T_1 G \to T_1 G$, called the
  \emph{adjoint representation}
  of a Lie group $G$.
\end{remark}

\begin{definition}
  A vector field $v \in \Vect(G)$ is
  called \emph{left-invariant}
  if $g . v = v$ for all $g \in G$, and
  $v$ is called
  \emph{right-invariant} if
  $v . g = v$ for all $g \in G$.
\end{definition}

\begin{theorem}
  The map $v \mapsto v(1)$ (where $1$ is
  the identity of $G$) defines an
  isomorphism of the vector space
  of left-invariant vector fields on $G$
  with $T_1 G$. Similarly, one has the
  same isomorphism for the
  vector space of right-invariant
  vector fields.
\end{theorem}

\begin{theorem}
  The map $v \mapsto v(1)$ defines an
  isomorphism of the vector space of
  bi-invariant vector fields on $G$
  with the vector space of invariants
  under the adjoint action, i.e.
  \[
    (T_1 G)^{\Ad G}
    = \{x \in T_1 G : \Ad_g(x) = x \text{ for all } g \in G\}.
  \]
\end{theorem}
