\chapter{Sept.~10 --- Representations of \texorpdfstring{$S_n$}{Sn}, Part 2}

\section{Properties of \texorpdfstring{$\C S_n$}{CSn}, Continued}

\begin{remark}
  We will now determine
  algebra generators of
  $\ZZ_m(n)$. It contains
  \begin{enumerate}
    \item $\ZZ_m(m)$: the center of
      $\C S_m$;
    \item $S_{[m + 1, n]}$: the subgroup
      of $S_n$ containing permutations
      fixing $1, \dots, m$;
    \item $J_k = \sum_{i = 1}^{k - 1} (i\ k)$
      for $k = m + 1, \dots, n$.
  \end{enumerate}
  Note that $J_{m + 1}, \dots, J_n$
  pairwise commute (check this as
  an exercise).
\end{remark}

\begin{theorem}
  The algebra $\ZZ_m(n)$ is generated by
  the subalgebras $\ZZ_m(m)$,
  $\C S_{[m + 1, n]}$, and the elements
  $J_{m + 1}, \dots, J_n$.
\end{theorem}

\begin{proof}
  Let $C$ be an $S_m$-conjugacy class in
  $S_n$.
  Define $\deg(C)$ to be the number of
  elements in $\{1, \dots, n\}$ which
  are moved by the corresponding
  permutations (for instance,
  $({*}\ n) = ({*}\ {*})$ has degree $2$).
  Note that we either have
  $\deg(C) = 0$ or $\deg(C) \ge 2$.

  Let $A$ be the subalgebra of
  $\ZZ_m(n)$ generated by
  $\ZZ_m(m)$, $\C S_{n-1}$, and
  $J_{m + 1}, \dots, J_n$. We need
  to show that $b_C \in A$ for every $C$.
  Assume it is not true, and pick
  $C$ of minimal degree such that
  $b_C \notin A$. First we show that
  $\deg(C) > 2$. If $\deg(C) = 2$, then
  we have two possibilities:
  \begin{enumerate}
    \item $C = ({*}\ k)$ for $k > m$.
      Then
      \[
        b_{({*}\, k)}
        = \sum_{i = 1}^m (i\ k)
        = J_k - \sum_{i = m + 1}^{k - 1} (i\ k).
      \]
      Then $J_k \in A$ and
      $\sum_{i = m + 1}^{k - 1} (i\ k) \in \C S_{[m + 1, n]} \subseteq A$, so
      we are good.
    \item $C = (k\ \ell)$ for
      $m < k < \ell \le n$. Then
      $b_{(k\, \ell)} \in \C S_{[m + 1, n]} \subseteq A$.
    \item $C = ({*}\ {*})$. Then
      $b_{({*}\, {*})} \in \ZZ_m(m) \subseteq A$.
  \end{enumerate}
  So $\deg(C) > 2$.
  Now assume that $C$ has more than
  $1$ cycle of degree $\ge 2$. Write
  $C = C' C''$, then
  \[
    b_{C'} b_{C''}
    = \alpha b_C + \sum_{C_0, \deg C_0 < \deg C} \alpha_{C_0} b_{C_0}.
  \]
  Since $b_{C'}, b_{C''}, b_{C_0} \in A$
  by minimality of $C$, we also get
  $\alpha b_C \in A$, so
  $b_C \in A$ since $\alpha \ne 0$ (note
  that we may have characteristic
  issues here if we are not working over $\C$).

  So  we may assume $C$ is a
  single cycle. Pick a cycle
  $(i_1\ i_2\ \cdots\ i_k) \in S_n$.
  Then if $j \notin \{i_1, \dots, i_k\}$,
  \[
    (i_1\ i_2\ \cdots\ i_k)
    (i_s\ j)
    = (i_1\ i_2\ \cdots\ i_{s - 1}\ j\ i_{s + 1}\ \cdots\ i_k).
  \]
  If $j \in \{i_1, \dots, i_k\}$,
  then $(i_1\ i_2\ \cdots\ i_k)(i_s\ j)$ either
  splits into two cycles or
  reduces the degree by $1$.

  So suppose a cycle in $C$ has
  elements from $\{1, \dots, m\}$
  and $k \in \{m + 1, \dots, n\}$.
  We can assume that $k$ is next to $*$.
  Denote by $C'$ the cycle obtained
  after eliminating $*$. Then
  \[
    b_{C'} b_{({*}\, k)}
    = \alpha b_C + \sum_{C_0} \alpha_{C_0} b_{C_0},
  \]
  where $C_0$ either contains disjoint
  cycles or cycles
  of smaller degree. Thus we get
  $b_{C'}, b_{({*}\, k)}, b_{C_0} \in A$
  by the minimality of $C$, so
  $b_C \in A$ as well.

  Thus we may assume the elements in our
  $1$-cycle $C$ sit in either
  $\{1, \dots, m\}$ or $\{m + 1, \dots, n\}$.
  In the first case, $b_C \in \ZZ_m(m) \subseteq A$, and
  in the second case,
  $b_C \in \C S_{[m + 1, n]} \subseteq A$.
\end{proof}

\begin{corollary}
  We have the following:
  \begin{enumerate}
    \item $\ZZ_{n - 1}(n)$ is
      commutative;
    \item for all $U \in \Irr(\C S_{n - 1})$
      and $V \in \Irr(\C S_{n})$,
      the multiplicity of $U$ in $V$ is
      either $0$ or $1$;
    \item the element $J_n$ acts on
      each irreducible $\C S_{n - 1}$-submodule
      of $V \in \Irr(\C S_n)$
      by a scalar.
  \end{enumerate}
\end{corollary}

\begin{proof}
  $(1)$ $\ZZ_{n - 1}(n)$ is generated
  by $\ZZ(n - 1)$ and $J_n$, which commute.

  $(2)$ This follows from the statement
  about abelian centralizers for algebras.

  $(3)$ This follows from Schur's lemma.
\end{proof}

\begin{example}
  We will determine how $J_n$
  acts on various modules and how they
  decompose:
  \begin{enumerate}
    \item $V = \refl_n$, which is a
      $\C S_n$-module and is given by
      \[
        \refl_n = \{(x_1, \dots, x_n) \in \C^n : x_1 + \dots + x_n = 0\}.
      \]
      As a $\C S_{n - 1}$-module,
      $\refl_n$ decomposes as follows
      \begin{itemize}
        \item $U_1 = \{(x_1, \dots, x_{n - 1}, 0) \in \C^n : x_1 + \dots + x_{n - 1} = 0\}$.
          This is $\refl_{n - 1}$.
        \item $U_0 = \{(-x, \dots, -x, (n - 1)x) \in \C^n\}$.
          This is the trivial representation.
      \end{itemize}
      Note that $J_n = \sum_{i = 1}^{n - 1} (i\ n)$
      acts on $(x_1, \dots, x_n)$ by
      \[
        (x_1, \dots, x_n)
        \longmapsto
        ((n - 2) x_1 + x_n, \dots, (n - 2) x_{n - 2} + x_n, x_1 + \dots + x_n).
      \]
      On $\refl_{n - 1}$, the eigenvalue
      is $n - 2$, and on the trivial
      subrepresentation,
      the eigenvalue is $-1$.
    \item When $n = 4$, there was a
      representation $V$ of dimension $2$,
      given by the pull-back of $\refl_3$
      under the
      projection $S_4 \to S_3$. The kernel
      of the projection is the normal
      subgroup
      \[
        \{e, (1\ 2)(3\ 4), (1\ 3)(2\ 4), (1\ 4)(2\ 3)\},
      \]
      where $S_3$ permutes
      $(1\ 2) (3\ 4)$, $(1\ 3)(2\ 4)$, and
      $(1\ 4)(2\ 3)$. Now
      \[
        J_4 = (1\ 4) + (2\ 4) + (3\ 4),
      \]
      and we are looking for an action
      of $J_4$ on $V$. We can take
      \[
        J_4|_V
        = (2\ 3) + (1\ 3) + (1\ 2),
      \]
      which is an element of $\C S_3$. Note
      that $\refl_3$ is given by
      \[
        \refl_3
        = \{(x_1, x_2, x_3) \in \C^3 : x_1 + x_2 + x_3 = 0\}.
      \]
      When $J_4|_V$ acts on
      $\refl_3$, we get
      $x_1 + x_2 + x_3 = 0$ in every
      coordinate,
      for any $(x_1, x_2, x_3) \in \refl_3$,
      so the eigenvalue in this case is $0$.
  \end{enumerate}
\end{example}

\section{Branching Graphs}

\begin{remark}
  Let $V^n$ be an irrep for $\C S_n$.
  We know that $V^n$ decomposes into a
  direct sum of non-isomorphic
  $\C S_{n - 1}$-modules. These then
  decompose into $\C S_{n - 2}$-modules,
  and so on.
\end{remark}

\begin{definition}
  The \emph{branching graph} is a
  directed graph, where the vertices
  are labeled by isomorphism classes
  of $\C S_n$-modules (for all $n$), and
  the edge $U \to V$ exists if
  $V$ is an irreducible module for
  $\C S_n$ and $U$ is an irreducible
  module for $\C S_{n - 1}$ which
  occurs in the decomposition of $V$.
\end{definition}

\begin{example}
  The following is the branching
  graph up to $S_4$:
  \begin{center}
    \begin{tikzcd}[sep=small]
      \triv_4 & & \refl_4 & \C^2 & \refl_4 \otimes \sign_4 & & \sign_4 \\
              & \triv_3 \arrow[lu] \arrow[ru] & & \refl_3 \arrow[lu] \arrow[u] \arrow[ru] & & \sign_3 \arrow[lu] \arrow[ru] \\
              & & \triv_2 \arrow[lu] \arrow[ru] & & \sign_2 \arrow[lu] \arrow[ru] \\
              & & & \triv_1 \arrow[lu] \arrow[ru]
    \end{tikzcd}
  \end{center}
  Note that there is a left-right
  symmetry in the graph, which
  comes from tensoring with $\sign_n$.
\end{example}

\begin{definition}
  Let $V^m \in \Irr(\C S_m)$
  and $V^n \in \Irr(\C S_n)$ for
  $m < n$. Define
  $\Path(V^m, V^n)$ to be the set of all
  paths from $V^m$ to $V^n$ in the
  branching graph. If $m = 1$, we write
  $\Path(V^n) = \Path(V^1, V^n)$, and
  we denote $\Path_n = \bigsqcup_{V^n \in \Irr(\C S_n)} \Path(V^n)$.
\end{definition}

\begin{remark}
  For $\overline{P} = (V^m \to V^{m + 1} \to \cdots \to V^n) \in \Path(V^m, V^n)$,
  denote by $V^m(\overline{P})$ a copy
  of $V^m$ in $V^n$ according to the path
  $\overline{P}$. Then we can write the
  decomposition of $V^n$ by
  \[
    V^n = \bigoplus_{V^m \in \Irr(\C S_m)}
    \bigoplus_{\overline{P} \in \Path(V^m, V^n)}
    V^m(\overline{P}).
  \]
\end{remark}

\begin{definition}
  Denote by $\varphi_{\overline{P}} : V^m \to V^n$
  the homomorphism sending $V^m$ to
  its copy in $V^n$ according to the
  path $\overline{P}$, which is
  defined uniquely
  up to rescaling, and define
  \[
    w_{\overline{P}}
    = (w_{m + 1}, \dots, w_n)
    \in \C^{n - m}
  \]
  where $w_k$ is the scalar
  by which $J_k$ acts on $V^{k - 1} \subseteq V^k$.
  Call $w_{\overline{P}}$ the
  \emph{weight} of $\overline{P}$.
\end{definition}

\begin{remark}
  Recall that
  $\Hom_{\C S_m}(V^m, V^n)$ is an
  irreducible $\ZZ_m(n)$-module from
  properties of centralizers.
\end{remark}

\begin{lemma}
  We have the following:
  \begin{enumerate}
    \item The elements $\varphi_{\overline{P}}$
      form a basis in
      $\Hom_{\C S_m}(V^m, V^n)$.
    \item Each $\varphi_{\overline{P}}$
      is an eigenvector for $J_k$
      with eigenvalue $w_k$, for each
      $k = m + 1, \dots, n$, where
      \[
        (w_{m + 1}, \dots, w_n) = w_{\overline{P}}.
      \]
  \end{enumerate}
\end{lemma}

\begin{proof}
  $(1)$ We can write
  \begin{align*}
    \Hom_{\C S_m}(V^m, V^n)
    &= \bigoplus_{V'^m \in \Irr(S_m)}
    \bigoplus_{\overline{P} \in \Path(V'^m, V^n)}
    \Hom(V^m, V'^m(\overline{P})) \\
    &= \bigoplus_{\overline{P} \in \Path(V^m, V^n)}
    \Hom(V^m, V^m(\overline{P})),
  \end{align*}
  where the second equality is
  by Schur's lemma. By
  Schur's lemma again,
  $\Hom(V^m, V^m(\overline{P})) \cong \C$.
  Since the $\varphi_{\overline{P}}$
  correspond to these summands, this
  proves $(1)$.

  $(2)$ For any $u \in V^m$, we have
  $[J_k \varphi_{\overline{P}}](u) = J_k [\varphi_{\overline{P}}(u)]$.
  By construction, $V^m(\overline{P})$
  lies in some copy of $V^{k - 1}$ in
  $V^k$ for $k = m + 1, \dots, n$,
  so $J_k \varphi_{\overline{P}} = w_k \varphi_{\overline{P}}$
  implies $(2)$.
\end{proof}
