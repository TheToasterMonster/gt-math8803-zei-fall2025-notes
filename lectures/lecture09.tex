\chapter{Sept.~17 --- Combinatorial Weights}

\section{More on the Degenerate Affine Hecke Algebra}

\begin{proof}[Proof of Lemma \ref{lem:hh2-irreps}]
  Assume $a \ne b$. Then $X_1, X_2$
  have two distinct eigenvalues, hence
  they are diagonalizable.
  Since $a \ne b$, every subspace in
  $M(a, b)$ stable under
  $X_1$ (or $X_2$) must be the sum
  of these eigenspaces. If
  one has a $1$-dimensional
  submodule for $\HH(2)$, then
  $T$ must preserve it. If
  $a = b \pm 1$, then $m \pm Tm$ is an
  eigenvector for $X_1, X_2, T$, so
  $M(a, b)$ is not irreducible.

  For the last part, we can simply
  switch the two eigenvalues.
\end{proof}

\begin{prop}
  The finite-dimensional irreps of
  $\HH(2)$ are classified by
  pairs of complex numbers $(a, b)$,
  $(a, b) \mapsto L(a, b)$, where
  $L(a, b) \cong L(b, a)$ if $b \ne a, a \pm 1$.
  Moreover, we have
  \begin{enumerate}
    \item If $b = a + 1$, then
      $L(a, b) = \C$ with
      $T \mapsto 1$, $X_1 = a$, $X_2 = b$.
    \item If $b = a - 1$, then
      $L(a, b) = \C$ with
      $T \mapsto -1$, $X_1 = a$, $X_2 = b$.
    \item If $b \ne a \pm 1$, then
      $L(a, b) \cong M(a, b)$.
    \item The action of $X_1, X_2$
      on $L(a, b)$ is diagonalizable
      if and only if $a \ne b$.
  \end{enumerate}
\end{prop}

\begin{proof}
  This is Example
  \ref{ex:hh2-module} and
  Lemma \ref{lem:hh2-irreps}.
\end{proof}

\begin{proof}[Proof of Theorem \ref{thm:path-varying}]
  Let $w_P = (w_1, \dots, w_n)$,
  $P' \in \Path(V, i)$, and
  $w_{P'} = (w_1', \dots, w_n')$,
  where the $w_j'$ depend only on
  $V_j, V_{j - 1}$. Note that 
  $V_j' = V_j$ for all $j \ne i$ implies
  $w_j' = w_j$ for all $j \ne i$. We
  have shown that $V_{P, i}$ is an
  irreducible $\mathcal{Z}_{i - 1}(i + 1)$-module
  and also an irreducible
  $\HH(2)$-module, and that
  $X_1, X_2$ ($J_i, J_{i + 1}$) are
  diagonalizable with
  eigenvalues $(w_i, w_{i + 1})$
  and $(w_i', w_{i + 1}')$.
  This proves $(1)$-$(3)$.

  $(4)$ If $w_{i + 1} = w_i \pm 1$, then
  $w_{i + 2} \ne w_i$ (check this
  as an exercise).
  By $(2)$, $w_{i + 1} = w_i \ne 1$
  implies that $V_{P, i + 1}$ is also
  $1$-dimensional, and
  $\C v_P$ is invariant under
  $(i, i + 1)$, $(i + 1, i + 2)$. Now
  observe that
  \[
    (i, i + 1)(i + 1, i + 2)(i, i + 1)
    = (i, i + 2)
    = (i + 1, i + 2)(i, i + 1)(i + 1, i + 2),
  \]
  which is the same element. But
  $(i, i + 1)$ and $(i + 1, i + 2)$
  act on $v_P$ by $\pm 1$ and $\mp 1$,
  respectively, so the above implies
  that $\mp 1 = \pm 1$, which is a
  contradiction.
\end{proof}

\section{Combinatorial Weights}

\begin{definition}
  We say two elements of $\C^n$
  are \emph{$c$-equivalent} (the
  $c$ is for ``combinatorial'')
  if one can be obtained from the other
  through a sequence of \emph{admissible}
  transpositions (those where the difference
  between two adjacent entries in the
  transposition is not $\pm 1$).
\end{definition}

\begin{definition}
  A \emph{combinatorial weight}
  is an element of $\C^n$ such that
  every element $(w_1, \dots, w_n) \in \C^n$ combinatorially
  equivalent to it satisfies:
  \begin{enumerate}
    \item $w_1 = 0$;
    \item for all $i = 1, \dots, n - 1$,
      $w_i \ne w_{i + 1}$;
    \item for all $i = 1, \dots, n - 2$,
      we have $w_{i + 1} = w_{i} \pm 1$
      implies $w_{i + 2} \ne w_i$.
  \end{enumerate}
  Denote the set of combinatorial weights
  by $c{\Wt_n}$.
\end{definition}

\begin{corollary}
  We have the following:
  \begin{enumerate}
    \item $\Wt_n \subseteq c{\Wt_n}$, so
      $\Wt_n$ is a collection of
      $c$-equivalence classes.
    \item $c$-equivalence implies
      $r$-equivalence. Moreover,
      $|{\Wt_n} / {\sim_r}| \le |{\Wt_n} / {\sim_c}| \le |{c{\Wt_n}} / {\sim_c}|$.
    \item There is a one-to-one correspondence
      ${\Wt_n} / {\sim_r} \longleftrightarrow \Irr(\C S_n)$.
  \end{enumerate}
\end{corollary}

\begin{lemma}\label{lem:comb-weights}
  Every $c$-equivalence class contains
  elements of the form
  \[(0, 1, \dots, n_1 - 1, -1, 0, 1, \dots, n_2 - 2, -2, \dots, (1 - k), \dots, n_k - k),\]
  where $n_1 \ge n_2 \ge \cdots \ge n_k$
  and $n_1 + \dots + n_k = n$.
\end{lemma}

\begin{proof}
  First we show that all components of
  combinatorial weights are integers.
  Suppose not, and
  let $i$ be the minimal number such
  that $w_i \notin \Z$. Then we can make
  admissible transformations from right
  to left until it reaches the first slot,
  which is a contradiction since
  $w_1 = 0 \in \Z$.

  Consider the lexicographic order
  on $c {\Wt_n}$, i.e.
  $(w_1, \dots, w_n) > (w_1', \dots, w_n')$
  if there exists $i$ such that
  $w_j = w_j'$ for each $1 \le j < i$
  and $w_i > w_i'$. Let
  $(w_1, \dots, w_n)$ be a maximal
  element in this equivalence class.
  We need to show that this maximal
  element is of the desired form.

  To do this, first take $n_1$ such that
  $n_1 - 1 = \max\{w_i\}$. Let $k$
  be the smallest index such that
  $w_k = n_1 - 1$. We claim that
  $k = n_1$ and $w_i = i - 1$ for all
  $i < n_1$. Assume not. Then pick
  the largest index $j < k$ with
  $w_j \ne n_1 - 1 - (k - j)$. By the
  choice of $k$, we have
  $w_j < n_1$. We also have
  $w_j \ge j - 1$ (otherwise one can
  permute $j$ and $j + 1$, which
  increases the order). Note that if
  $w_j \ge j$, then we can make admissible
  transformations to the left until
  we arrive to $(w_j, w_j)$,
  $(w_j, w_{j \pm 1}, w_j)$, or
  $w_j$ in the first position, which are
  all impossible. Thus $w_j = n_1 - 1 - (k - j)$
  for all $j < k$. But $w_1 = 0$, so
  $k = n_1$.

  Thus we have shown that we can take
  an element starting with
  $0, 1, \dots, n_1 - 1$. Now if
  $n_1 = n$, then we are done. Otherwise,
  we need to prove that $w_{n_1 + 1} = -1$.
  Note that $w_{n_1 + 1} \le n_1 - 1$
  by our choice of $n_1$, and
  $w_{n_1} \ne n_1 - 1$ since
  $w_{n + 1} \ne w_n$. If
  we move $w_{n + 1}$ to the left,
  then we encounter
  \[
    (w_{n_1 + 1}, w_{n_1 + 1} + 1, w_{n_1 + 1})
  \]
  for any $w_{n + 1} \ge 0$. If
  $w_{n + 1} < -1$, then we can move it
  to the first position, which is impossible
  since we always have $w_1 = 0$. So the only possibility
  is $w_{n + 1} = -1$.

  Now we can repeat the above argument
  to get the rest of the form.
\end{proof}

\begin{remark}
  Lemma \ref{lem:comb-weights}
  implies the following:
  \begin{enumerate}
    \item $c {\Wt_n} = \Wt_n$;
    \item ${\sim_c} = {\sim_r}$;
    \item $n_1, \dots, n_k$ uniquely
      characterize the equivalence class.
  \end{enumerate}
\end{remark}

\begin{example}
  Consider the following:
  \begin{enumerate}
    \item $\triv_4$ for $S_4$, i.e.
      $(x, x, x, x)$.
      Here $(w_1, w_2, w_3, w_4) = (0, 1, 2, 3)$, so
      $k = 1$, $n_1 = 4$.
    \item $\refl_4$ with path
      $P = \triv_1 \to \dots \to \triv_i \to \refl_{i + 1} \to \dots \to \refl_n$. In
      this case, we have seen that
      \[
        (w_1, \dots, w_n)
        = (0, 1, \dots, i - 1, i, \dots, n - 2).
      \]
      For $\refl_4$, we can get
      $(0, -1, 1, 2)$, $(0, 1, -1, 2)$,
      $(0, 1, 2, -1)$. The last one has
      $k = 2$, $n_1 = 3$, $n_2 = 1$.
  \end{enumerate}
\end{example}

\begin{exercise}
  Compute the combinatorial weights
  for $\C^2$ (for $S_4$).
\end{exercise}
