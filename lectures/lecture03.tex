\chapter{Aug.~25 --- Complete Reducibility}

\section{Reducibility of Modules}

\begin{remark}
  Consider an associative algebra
  $A$ over a field $\F$. We proceed
  to study completely reducible
  representations of $A$.
  Let $U$ be an $A$-module.
\end{remark}

\begin{definition}
  An $A$-module $U$ is \emph{irreducible}
  if it only has two distinct submodules
  ($\{0\}$ and $U$).
\end{definition}

\begin{remark}
  With this definition, $\{0\}$ is
  not irreducible.
\end{remark}

\begin{definition}
  An $A$-module $U$ is \emph{completely reducible}
  if for any submodule $U' \subseteq U$,
  there exists an $A$-submodule
  $U''$ such that $U = U' \oplus U''$.
\end{definition}

\begin{exercise}
  Show that any submodule and any quotient
  module of a completely reducible
  $A$-module is also completely reducible.
\end{exercise}

\begin{example}
  Consider $A = \End_\F(U)$. Then $U$ is
  an $A$-module and is irreducible
  (there is a linear operator
  $\alpha : U \to U$ taking
  $u \mapsto v$ for any $u, v \in U$, so
  there are no nontrivial invariant
  subspaces).
\end{example}

\begin{prop}
  Let $U_1, U_2$ be completely reducible
  $A$-modules. Then $U_1 \oplus U_2$ is
  completely reducible.
\end{prop}

\begin{proof}
  Left as an exercise.
\end{proof}

\begin{corollary}
  Let $U$ be a finite-dimensional
  $A$-module. Then the following are
  equivalent:
  \begin{enumerate}
    \item $U$ is completely reducible;
    \item $U$ is isomorphic to a direct
      sum of irreducible submodules.
  \end{enumerate}
\end{corollary}

\begin{exercise}
  Show that every irreducible $A$-module is
  isomorphic to a quotient
  module for a regular module (i.e. one
  isomorphic to $A$).
  In particular, every irreducible module
  over a finite-dimensional associative
  $\F$-algebra is finite-dimensional.
\end{exercise}

\section{Schur's Lemma}
\begin{theorem}[Schur's lemma]
  Let $A$ be an associative $\F$-algebra
  and $U, V$ irreducible $A$-modules.
  Then
  \begin{enumerate}
    \item if $U, V$ are not
      isomorphic, then
      $\Hom_A(U, V) = 0$;
    \item $\End_A(U)$ is a skew
      field (i.e. a division ring).
      Furthermore, if
      $U$ is finite-dimensional
      and $\F$ is algebraically closed, then
      $\dim \End_A(U) = 1$.
  \end{enumerate}
\end{theorem}

\begin{proof}
  $(1)$ Assume we have a nonzero homomorphism
  $\varphi : U \to V$. Then
  $\ker \varphi \subsetneq U$, and
  $\im \varphi \subseteq V$ is
  nontrivial, so by irreducibility
  $\varphi$ must be an isomorphism.

  $(2)$ Let $\varphi \in \End_A(U)$.
  From $(1)$, we know that
  $\varphi$ is an isomorphism, so
  $\varphi$ has an inverse, i.e.
  $\End_A(U)$ is a skew field.
  For the second part, since
  $\F$ is algebraically closed, we can
  find an eigenvalue $z$ for $\varphi$.
  Then $\varphi - z \id_U$ is not
  invertible, so we have
  $\varphi - z \id_U = 0$ by $(1)$.
\end{proof}

\begin{exercise}
  Consider $1, i, j, k$, where
  $i^2 = j^2 = k^2 = -1$ and
  $ij = -ji = k$. The
  \emph{quaternion algebra} over $\R$ is
  given by
  \[
    \mathbb{H}_\R
    = \{q = w + xi + yj + zk : w, x, y, z \in \R\}
  \]
  Note that $\overline{q} = w - xi - yj - zk$
  satisfies
  $q \overline{q} = w^2 + x^2 + y^2 + z^2$,
  so $q^{-1} = \overline{q} / (w^2 + x^2 + y^2 + z^2)$, i.e.
  $\mathbb{H}_\R$ is a skew field.
  Show that
  $\End_{\mathbb{H}_\R}(\mathbb{H}_\R) \cong \mathbb{H}_\R^\mathrm{opp}$.
\end{exercise}

\begin{remark}
  We have an embedding
  $\mathbb{H}_\R \hookrightarrow \Mat_2(\C)$
  given by
  \[
    q \longmapsto
    \begin{pmatrix}
      w + xi & y + zi \\
      -y + zi & w - xi
    \end{pmatrix}.
  \]
  If we replace $\R$ with $\C$, then
  $\mathbb{H}_\C \cong \Mat_2(\C)$,
  which is reducible (consider the sum of
  column spaces).
\end{remark}

\begin{definition}
  Let $U$ be an $A$-module. We say that
  $U$ is \emph{endotrivial} if
  $\End_A(U)$ consists only of scalar maps,
  i.e. maps of the form $z \id$.
\end{definition}

\begin{remark}
  Suppose $\F$ is algebraically closed and
  uncountable (e.g. $\C$), $A$ has
  countable dimension over $\F$, and
  $U$ an irreducible $A$-module. Then
  $U$ is endotrivial.
\end{remark}

\begin{definition}
  Define the \emph{center} of $A$ to be
  \[
    \mathcal{Z}(A) = \{z \in A : za = az \text{ for all } a \in A\}.
  \]
  Note that this is a commutative algebra.
\end{definition}

\begin{exercise}
  Schur's lemma gives a description of
  the center of $A$.
  Let $U$ be an endotrivial $A$-module
  (e.g. a finite-dimensional irreducible
  module over
  $A$ if $\F$ is algebraically closed).
  Show that $z \in \mathcal{Z}(A)$ 
  acts as a scalar on $U$. We call the
  algebra homomorphism
  $\mathcal{Z}(A) \to \F$ the
  \emph{central character} of $U$.
\end{exercise}

\section{Completely Reducible Modules}

\begin{remark}
  Consider finite direct sums of
  endotrivial irreducible modules:
  \[
    \bigoplus_{i = 1}^k U_i \otimes M_i,
  \]
  where the $U_i$ are endotrivial
  modules and the $M_i$ are
  vector spaces known as
  \emph{multiplicity spaces}. Note that
  $U_1^{\oplus i} = U_1 \otimes \F^i$.
  The
  $A$-action on the direct sum for
  $a \in A$ is given by
  \[
    a(u_1 \otimes m_1, \dots, u_k \otimes m_k)
    = (au_1 \otimes m_1, \dots, au_k \otimes m_k).
  \]
  We will use Schur's lemma to understand
  homomorphisms between such modules.

  Write $U^j = \bigoplus_{i = 1}^k U_i \otimes M_i^j$
  for $j = 1, 2$. We can produce a linear
  map
  \[
    \bigoplus_{i = 1}^k \Hom_\F(M_i^1, M_i^2)
    \longrightarrow \Hom_A(U^1, U^2)
  \]
  in the following manner:
  For $\underline{\varphi} = (\varphi_1, \dots, \varphi_k) \in \bigoplus_{i = 1}^k \Hom_\F(M_i^1, M_i^2)$,
  we can define
  \[
    \psi_{\underline{\varphi}}\left(
      \sum_{i = 1}^k u_i \otimes m_i^1
    \right)
    = \sum_{i = 1}^k u_i \otimes \varphi_i(m_i^1).
  \]
\end{remark}

\begin{theorem}
  We have the following:
  \begin{enumerate}
    \item The map $\underline{\varphi} \mapsto \psi_{\underline{\varphi}}$
      defines a vector space isomorphism
      \[
        \bigoplus_{i = 1}^k \Hom_\F(M_i^1, M_i^2)
        \overset{\cong}\longrightarrow \Hom_A(U^1, U^2).
      \]
    \item Every $A$-module homomorphism
      $U_1 \to U_2$ sends
      $U_i \otimes M_i^1$ to
      $U_i \otimes M_i^2$ for any $i$.
  \end{enumerate}
\end{theorem}

\begin{proof}
  Left as an exercise (use Schur's lemma).
\end{proof}

\begin{corollary}\label{cor:hom-isomorphism}
  We have the following:
  \begin{enumerate}
    \item there is an isomorphism
      $\Hom_A(U_i, U) \xrightarrow{\cong} M_i$;
    \item there is an isomorphism
      $\bigoplus_{i = 1}^k U_i \otimes \Hom_A(U_i, U) \cong U$
      given by
      \[
        \sum_{i = 1}^k u_i \otimes \varphi_i
        \longmapsto \sum_{i = 1}^k \varphi_i(u_i).
      \]
  \end{enumerate}
\end{corollary}

\begin{prop}\label{prop:submodule-collection}
  For any $A$-submodule $U' \subseteq U$,
  there exists a unique collection of
  determined subspaces $M_i' \subseteq M_i$
  such that
  $U' = \bigoplus_{i = 1}^k U_i \otimes M_i'$
  as submodules of $U$.
\end{prop}

\begin{proof}
  Note that $\Hom_A(U_i, U') \subseteq \Hom_A(U_i, U)$,
  set $M_i' = \Hom_A(U_i, U')$,
  and use Corollary \ref{cor:hom-isomorphism}.
\end{proof}

\begin{theorem}
  Let $U_i$ be irreducible modules for
  $A$ and consider maps
  $\beta_i : A \to \End_\F(U_i)$. Set
  \[
    \beta = \beta_1 \oplus \cdots \oplus \beta_k : A \longrightarrow \bigoplus_{i = 1}^k \End_\F(U_i),
  \]
  where the $U_i$ are pairwise
  non-isomorphic. Then the
  homomorphism $\beta$ is surjective.
\end{theorem}

\begin{proof}
  Replace $A$ by $A /{\ker \beta}$, so that
  $\beta$ is injective. Then
  $\beta$ equips $\bigoplus_{i = 1}^k \End(U_i)$
  with an $A$-bimodule structure, and
  there is a natural isomorphism
  $\End_\F(U_i) \cong U_i \otimes U_i^*$.
  View $U_i$ as the multiplicity space
  for the right $A$-module and $U_i^*$ as
  the multiplicity space for the left
  $A$-module. By Proposition
  \ref{prop:submodule-collection},
  \[
    A = \bigoplus_{i = 1}^k U_i \otimes V_i
  \]
  as a left $A$-module for some
  $V_i \subseteq U_i^*$. Similarly for
  the right $A$-module, we have
  \[
    A = \bigoplus_{i = 1}^k W_i \otimes U_i^*
  \]
  for some $W_i \subseteq U_i$.
  Then we must have
  $U_i \oplus V_i = W_i \oplus U_i^*$,
  so $U_i \cong W_i$ and $V_i \cong U_i^*$
  (the identity $1 \in A$ guarantees that
  no component is zero). Thus
  $\beta$ is surjective.
\end{proof}

\begin{corollary}
  Let $\F$ be algebraically closed and
  $A$ a finite-dimensional $\F$-algebra.
  Then the set of isomorphism classes of
  irreducible $A$-modules is finite
  and non-empty.
\end{corollary}

\begin{proof}
  First this set is nonempty since $A$ is
  nonzero, so it has an
  irreducible subrepresentation.
  To see that it is finite,
  note that for all collections
  $U_1, \dots, U_k$, the map
  $A \to \bigoplus_{i = 1}^k \End_\F(U_i)$
  is surjective, so
  \[
    \dim A \ge \sum_{i = 1}^k (\dim U_i)^2.
  \]
  This proves the desired result, since
  $A$ is finite-dimensional.
\end{proof}

\section{Simple Algebras}

\begin{definition}
  An algebra $A$ is \emph{simple} if
  the only two-sided ideals are
  $\{0\}$ and $A$ (i.e. $A$ is irreducible
  as a bimodule over itself).
\end{definition}

\begin{theorem}
  Let $\F$ be an algebraically closed
  field and $A$ a finite-dimensional
  $\F$-algebra. Then the following are
  equivalent:
  \begin{enumerate}
    \item $A$ is simple;
    \item $A \cong \End_\F(U)$ for
      some finite-dimensional vector space
      $U$.
  \end{enumerate}
\end{theorem}

\begin{proof}
  $(1 \Rightarrow 2)$: The algebra
  $A$ has an irreducible representation
  $U$, i.e. we have a map
  $A \to \End_\F(U)$. Since $A$ is
  simple, this map must have trivial kernel,
  i.e. it is injective.
  We also already
  know that it is surjective, so
  this map is an isomorphism.

  $(2 \Rightarrow 1)$: Assume $I$ is a
  two-sided ideal in $\End_\F(U) \cong U \otimes U^*$
  and view $I \subseteq U \otimes U^*$.
  Show as an exercise
  that we must have $I = \{0\}$.
\end{proof}

\begin{theorem}
  Every finite-dimensional module
  $V$ for $A = \End_\F(U)$ is isomorphic
  to a direct sum of several copies of $U$.
\end{theorem}

\begin{proof}
  Recall that every finitely generated
  module $V$ is a quotient of
  $A^{\oplus \ell}$ for some $\ell \in \N$.
  We can write $A = U \otimes U^*$.
  Let $A^{\oplus \ell} = U \otimes M$ and
  consider the quotient map
  $\pi : U \otimes M \to V$. Then
  $\ker \pi \subseteq U \otimes M$
  must be of the form
  $U \oplus M_0$, so
  we have $V \cong (U \otimes M) / (U \otimes M_0) = U \otimes (M / M_0)$.
\end{proof}
