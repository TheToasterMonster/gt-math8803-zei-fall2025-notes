\documentclass[12pt, letterpaper, oneside]{book}
\usepackage[margin={0.6in, 0.75in}]{geometry}
\usepackage{microtype}
% \usepackage{kpfonts}
\usepackage{amsmath, amssymb, amsthm}
\usepackage{parskip}
\usepackage[many]{tcolorbox}
\usepackage{footnote}
\usepackage{cancel}
\usepackage{titlesec}
\usepackage{pgffor}
\usepackage[shortlabels, inline]{enumitem}
\usepackage{hyperref}
\usepackage{tikz-cd}

\usepackage[overload]{textcase}

\renewcommand{\chaptername}{Lecture}
\newtheorem{axiom}{Axiom}[chapter]
\newtheorem{theorem}{Theorem}[chapter]
\newtheorem{prop}{Proposition}[chapter]
\newtheorem{corollary}{Corollary}[theorem]
\newtheorem{lemma}{Lemma}[chapter]
\newtheorem{conjecture}{Conjecture}[theorem]
\theoremstyle{definition}
\newtheorem{definition}{Definition}[chapter]
\newtheorem{exercise}{Exercise}[chapter]
\newtheorem{example}{Example}[definition]
\newtheorem*{remark}{Remark}

\tcbset{sharp corners, breakable, enhanced, parbox=false}
\newtcolorbox{mybox}[3][]
{
  colframe = #2!150,
  colback  = #2!5,
  coltitle = #2!0!white,  
  title    = {#3},
  #1,
}

\titleformat{\chapter}[display]
    {\normalfont\huge\bfseries}{\chaptertitlename\ \thechapter}{20pt}{\Huge}
\titlespacing*{\chapter}{0pt}{0pt}{40pt}

\newcommand{\R}{\mathbb{R}}
\newcommand{\N}{\mathbb{N}}
\newcommand{\Z}{\mathbb{Z}}
\newcommand{\C}{\mathbb{C}}
\newcommand{\Q}{\mathbb{Q}}
\newcommand{\F}{\mathbb{F}}
\newcommand{\Mod}[1]{\ {\mathrm{mod}\ #1}}
\newcommand{\Pmod}[1]{\ (\mathrm{mod}\ #1)}

\DeclareMathOperator{\lcm}{lcm}
\DeclareMathOperator{\re}{Re}
\DeclareMathOperator{\im}{Im}
\DeclareMathOperator{\GL}{GL}
\DeclareMathOperator{\Mat}{Mat}
\DeclareMathOperator{\Hom}{Hom}
\DeclareMathOperator{\Bilin}{Bilin}
\DeclareMathOperator{\rank}{rank}
\DeclareMathOperator{\Tr}{Tr}
\DeclareMathOperator{\End}{End}

\title{MATH 8803: Representation Theory}
\author{Frank Qiang\\Instructor: Anton Zeitlin}
\date{Georgia Institute of Technology\\Fall 2025}

\begin{document}
  \maketitle

  \begingroup
  \let\cleardoublepage\clearpage
  \tableofcontents
  \endgroup

  % \foreach \i in {00, 01, 02, 03, 04, ..., 50} {%
  %   \edef\FileName{lectures/lecture\i.tex}%     The % here are necessary to eliminate any
  %   \IfFileExists{\FileName}{%  spurious spaces that may get inserted
  %      \input{\FileName}%       at these points
  %   }
  % }
  \chapter{Aug.~18 --- Historical Perspective}

\section{Origin of Representation Theory}

One motivation for representation theory is
symmetries in physics.
From a mathematical perspective, we consider
\emph{groups} and \emph{algebras} (a vector
space with a bilinear operation).
In this course, we will study two types of
groups:
\begin{enumerate}
  \item \emph{finite groups}, e.g. the symmetric group;
  \item \emph{Lie groups}, e.g. the rotation group.
\end{enumerate}

\begin{definition}
  A \emph{representation} of a group $G$
  is a homomorphism $G \to \End(V)$, where
  $V$ is some finite-dimensional vector space.
\end{definition}

The history of representation theory is as
follows:
\begin{enumerate}
    \item
    In the late 19th century, people
    were interested in \emph{crystallography},
    in particular crystallographic groups
    and their classification. There are
    related objects called \emph{Bieberbach groups} (e.g. $O(n)$ with translations, i.e.
    $\R^n \rtimes O(n)$).

    Sophus Lie discovered \emph{Lie groups}
    in his main manuscript ``Transformation
    groups.'' From Lie groups, one then derives
    \emph{Lie algebras}.

    \item
    In the early 20th century (1905),
    \emph{special relativity} was discovered,
    which involves the \emph{Lorentz group}
    $\SO(1, 3)$ (the transformations preserving
    the form $-t^2 + x^2 + y^2 + z^2$).
    This is a Lie group.

    Around the same time, E. Cartan developed
  the modern theory of \emph{semisimple Lie groups} and \emph{Lie algebras}, and
    H. Weyl studied their representations.

    \item
    In the period 1920--1930, quantum (``matrix'')
    mechanics was discovered. Here
    one has a Hilbert space $\mathcal{H}$ and
    a self-adjoint Hamiltonian (energy)
    operator $H$ on $\mathcal{H}$.
    The symmetry operator $A$ satisfies
    the commutator relation
    $[H, A] = 0$, and if we set $U = e^{iA}$,
    we have $U H U^\dagger = H$.

    \item
    After the discovery of \emph{spin} by W.
    Pauli, E. Wigner realized that spin was
    directly related to the representation
    theory of the universal cover
    $\pi : \SU(2) \to \SO(3)$.

    In the 1960s, there was a ``zoo'' of
    elementary particles. M. Gell-Mann and
    Y. Neeman realized that all of these
    can be described by representations of
    $SU(3)$. The led to the discovery of
    \emph{quarks} and the later
    notion of grand unified theories and
    string theory in the 1970s.

    There are also connections to condensed
    matter theory and quantum information.
\end{enumerate}

This course will cover the following topics:
\begin{enumerate}
  \item basics about associative algebras
    and their representations, finite groups
    and their representations in general,
    the symmetric group and its representations,
    Young tableaux;
  \item Lie groups and Lie algebras;
  \item the structure of semisimple Lie
    algebras;
  \item representations of $\SL(n)$.
\end{enumerate}

\section{Introduction to Lie Groups and Lie Algebras}
In general, groups are complicated, whereas
algebras are less complicated. We begin with
finite groups.

\begin{definition}
  Let $G$ be a finite group and $\F$ a field.
  The \emph{group algebra} $\F G$ is
  \[
    \F G = \left\{
      \sum_g a_g g : a_g \in \F
    \right\}.
  \]
  This forms an algebra over $\F$
  with the obvious multiplication operation.
\end{definition}

\begin{example}
  Consider the rotation group, generated by the
  matrices
  \[
    R_z(\theta) =
    \begin{pmatrix}
      \cos \theta & \sin \theta & 0 \\
      -\sin \theta & \cos \theta & 0 \\
      0 & 0 & 1
    \end{pmatrix}, \quad
    R_x(\phi) =
    \begin{pmatrix}
      1 & 0 & 0 \\
      0 & \cos \phi & -\sin \phi \\
      0 & \sin \phi & \cos \phi
    \end{pmatrix}, \quad
    R_y(\psi) =
    \begin{pmatrix}
      \cos \psi & 0 & -\sin \psi \\
      0 & 1 & 0 \\
      \sin \psi & 0 & \cos \psi
    \end{pmatrix}.
  \]
  Letting $\delta$ be an infinitesimal
  value and using
  a Taylor expansion, we can write
  \begin{align*}
    R_z(\delta \theta)
    &= 1 + \delta \theta \begin{pmatrix}
      0 & 1 & 0 \\
      -1 & 0 & 0 \\
      0 & 0 & 0
    \end{pmatrix} = 1 + \delta \theta M_z, \\
    R_x(\delta \phi)
    &= 1 + \delta \phi \begin{pmatrix}
      0 & 0 & 0 \\
      0 & 0 & -1 \\
      0 & 1 & 0
    \end{pmatrix} = 1 + \delta \phi M_x, \\
    R_y(\delta \psi)
    &= 1 + \delta \psi \begin{pmatrix}
      0 & 0 & -1 \\
      0 & 0 & 0 \\
      1 & 0 & 0
    \end{pmatrix} = 1 + \delta \psi M_y.
  \end{align*}
  We can measure the commutativity of these
  matrices via
  \begin{align*}
    R_x(\delta \phi) R_y(\delta \psi)
    R_x^{-1}(\delta \phi)
    R_y^{-1}(\delta \psi)
    &= (1 + M_x \delta \phi)
    (1 + M_y \delta \psi)
    (1 - M_x \delta \phi)
    (1 - M_y \delta \psi) \\
    &= 1 + \delta \phi \delta \psi (M_x M_y - M_y M_x).
  \end{align*}
\end{example}

\begin{exercise}
  Show that $[M_x, M_y] = -M_z$.
\end{exercise}

\begin{remark}
  Thus we have a vector space spanned by
  $M_x, M_y, M_z$ with an operation
  $[\cdot, \cdot]$ satisfying the identity
  $[M_x, M_y] = -M_z$. Note that this property
  is satisfied by the cross product on $\R^3$.
  The cross product also satisfies the
  following
  \emph{Jacobi identity}:
  \[
    [A, [B, C]] = [[A, B], C] + [B, [A, C]].
  \]
  The above properties define a
  \emph{Lie algebra}.
\end{remark}

\begin{definition}
  Let $\{e_k\}$ be a basis of a Lie algebra
  and $[e_i, e_j] = \sum_k c_{ij}^k e_k$.
  The \emph{universal enveloping algebra}
  of the Lie algebra is the free associative
  algebra on $\{e_k\}$, modulo the
  relations $[e_i, e_j] = \sum_k c_{ij}^k e_k$.
\end{definition}

\begin{remark}
  One way to return to the Lie group from
  the Lie algebra is exponentiation, e.g.
  $R_z(\theta) = e^{\theta M_z}$.
\end{remark}

\section{Algebras and Modules}

Let $k$ be a commutative ring (most of the time
$k = \C$). All rings will be associative and
unital.

\begin{definition}
  A \emph{(associative and unital) $k$-algebra}
  is a unital ring $A$ with a homomorphism
  $i : k \to A$ such that
  $i(r) \cdot a = a \cdot i(r)$, i.e. the
  image of $i$ commutes with $A$.
\end{definition}

\begin{example}
  Any ring is a $\Z$-algebra.
\end{example}

\begin{definition}
  A \emph{homomorphism} of $k$-algebras is a
  $k$-linear homomorphism of unital rings.
\end{definition}

\begin{definition}
  Let $A, B$ be unital rings, and $M$ an
  abelian group. Then
  \begin{enumerate}
    \item a \emph{left $A$-module structure}
      on $M$ is a $\Z$-bilinear map
      $A \times M \to M$, associative in
      the sense that
      \[
        a_1(a_2 m) = (a_1 a_2) m, \quad
        \text{for all } a_1, a_2 \in A,\, m \in M,
      \]
      and such that $1_A m = m$ for all $m \in M$;
    \item a \emph{right $A$-module structure}
      on $M$ is a $\Z$-bilinear map
      $M \times B \to M$, associative in
      the sense that
      \[
        (m b_1) b_2 = m (b_1 b_2), \quad
        \text{for all } b_1, b_2 \in B,\, m \in M,
      \]
      and such that $m 1_B = m$ for all $m \in M$;
    \item an \emph{$A$-$B$-bimodule structure}
      on $M$ is a left $A$-module and
      right $B$-module structure on $M$,
      along with the condition that
      $(am) b = a(mb)$ for all
      $a \in A$, $b \in B$, and $m \in M$.
  \end{enumerate}
\end{definition}

\begin{remark}
In general, an $A$-module will mean a left $A$-module by default.
\end{remark}

\begin{definition}
  Let $M, N$ be left $A$-modules.
  An \emph{$A$-module homomorphism} is a map
  $\varphi : M \to N$ such that
  $\varphi(am) = a \varphi(m)$ for all
  $a \in A$ and $m \in M$.
\end{definition}

\begin{example}
  A ring $A$ is both a left/right
  $A$-module and an $A$-$A$-bimodule
  (the \emph{regular bimodule}).
\end{example}

\begin{definition}
  The \emph{direct sum}
  $\bigoplus_{i \in I} M_i$ of
  left $A$-modules $M_i$ is the collection
  of $(m_i)_{i \in I}$ with finitely
  many nonzero entries, with
  component-wise addition and
  scalar multiplication.
\end{definition}

\begin{example}
  Let $I$ be an index set. Then
  $A^{\oplus I}$ is the
  \emph{coordinate $A$-module}.
\end{example}

\begin{definition}
  A \emph{submodule} of $M$ is a nontrivial
  subgroup closed under addition and invariant
  under the action of $A$.
\end{definition}

\begin{example}
  Submodules of the regular left/right
  $A$-module are the left/right ideals of $A$.
\end{example}

\begin{definition}
  Let $M$ be a left $A$-module and
  $M_0$ a submodule of $M$. The
  \emph{quotient module} $M / M_0$
  is the set of equivalence classes
  $m + M_0$, where the action of $A$ is
  given by $a(m + M_0) = am + M_0$.
\end{definition}

\begin{lemma}
  Let $M, N$ be $A$-modules and $M_0 \subseteq M$
  a submodule. Let
  $\varphi : M \to N$ be $A$-linear such that
  $\varphi(M_0) = \{0\}$. Then there exists a
  unique $A$-linear map $\underline{\varphi} : M / M_0 \to N$
  such that $\varphi = \underline{\varphi} \circ \pi$,
  where $\pi : M \to M / M_0$ is the
  canonical projection.
\end{lemma}

  \chapter{Aug.~20 --- Algebras and Modules}

\section{More on Algebras and Modules}
\begin{definition}
  A \emph{free} module is a module
  which has a basis.
\end{definition}

\begin{example}
  Consider the coordinate module
  $A^{\oplus I}$. Then a basis is given
  by $e_i = \{\delta_{ij}\}_{j \in I}$
    for $i \in I$.
\end{example}

\begin{prop}
  Let $M$ be a left $A$-module. Let
  $I$ be an index set and let $m_i \in M$
  for $i \in I$. Then
  \begin{enumerate}
    \item There exists a unique $A$-linear
      map $A^{\oplus I} \to M$ which sends
      $e_i \mapsto m_i$.
    \item This map is surjective if and only
      if the elements $m_i$ span $M$.
      In particular, every $M$ is isomorphic
      to a quotient of a free module.
    \item This map is an isomorphism
      if and only if $\{m_i\}$
      form a basis of $M$. In particular,
      every coordinate module is a
      free module.
  \end{enumerate}
\end{prop}

\begin{proof}
  Left as an exercise.
\end{proof}

\begin{example}
  Suppose $M$ is spanned by a single
  element $m$. Then $M \cong A / I$,
  where $I$ is the left ideal
  \[
    I = \{a \in A : am = 0\}.
  \]
\end{example}

\begin{example}
  We can now construct the following
  examples of algebras:
  \begin{enumerate}
    \item Let $\Mat_n(A)$ be the
      set of $n \times n$ matrices with
      entries in $A$. If $A$ is a
      $k$-algebra, then $\Mat_n(A)$
      is also a $k$-algebra.
    \item If $G$ is a group, then the
      group algebra $kG$ (for a ring $k$)
      given by
      \[
        kG = \left\{\sum_{g \in G} a_g g : a_g \in k\right\}
      \]
      is a free module with basis identified
      with the elements of $G$.

      The importance of of this
      object is as follows: Let $G$
      be a group and $B$ an algebra.
      Consider the set of maps satisfying
      $1_G \mapsto 1_B$ and respecting
      the group multiplication. This
      set is in bijection with maps
      $kG \to B$ (they extend by linearity).
      If $V$ is a vector space and
      $B = \End(V)$, then this statement
      says that there is a bijection
      between the representations of the
      group $G$
      and the representations of the group
      algebra $kG$.
    \item If $I$ is a two-sided ideal, then
      $A / I$ has a natural algebra
      structure.
    \item If $A_1, A_2$ are $k$-algebras,
      then the direct sum
      $A_1 \oplus A_2$ is again a
      $k$-algebra (with component-wise multiplication).
      One can extend this by induction to
      a finite direct sum, but note that
      we lose
      the multiplicative identity in
      an infinite direct sum (so we
      do not get an algebra in the infinite
      case).
  \end{enumerate}
\end{example}

\section{Module of Homomorphisms}
\begin{definition}
  Let $k$ be a commutative ring and
  $A$ a $k$-algebra. Let $M, N$ be
  left $A$-modules. Denote by
  $\Hom_A(M, N)$ the set of all
  $A$-module homomorphisms $M \to N$.
  Give $\Hom_A(M, N)$ a $k$-module
  structure via
  \[
    [\varphi_1 + \varphi_2](m) = \varphi_1(m) + \varphi_2(m), \quad
    [r \varphi](m) = r \varphi(m)
  \]
  for $\varphi_1, \varphi_2 \in \Hom_A(M, N)$,
  $r \in k$, and $m \in M$.
\end{definition}

\begin{remark}
  Let $L, M, N$ be left $A$-modules. Then
  we can define a $k$-bilinear map
  \begin{align*}
    \Hom_A(M, N) \times \Hom_A(L, M) &\longrightarrow \Hom_A(L, N) \\
    (\varphi, \psi) &\longmapsto \varphi \circ \psi.
  \end{align*}
\end{remark}

\begin{exercise}
  Let $N_2$ be an $A$-module,
  $N_1 \subseteq N_2$ an $A$-submodule,
  and $N_3 = N_2 / N_1$. Let
  $i : N_1 \hookrightarrow N_2$ be the
  inclusion and $\pi : N_2 \to N_3$ the
  projection. Define the maps
  \begin{align*}
    \widetilde{\iota} : \Hom(M, N_1) &\rightarrow \Hom(M, N_2) \\
    \varphi_1 &\longmapsto i \circ \varphi_1 \\
    \widetilde{\pi} : \Hom(M, N_2) &\rightarrow \Hom(M, N_3) \\
    \varphi_2 &\longmapsto \pi \circ \varphi_2.
  \end{align*}
  Then show that $\widetilde{\iota}$
  is injective and $\im \widetilde{\iota} = \ker \widetilde{\pi}$.
\end{exercise}

\begin{remark}
  Let $B$ be a $k$-algebra and $M$ and
  $A$-$B$-bimodule. Then for all
  $A$-modules $N$, we have that
  $\Hom_A(M, N)$
  is a left $B$-module via
  \[
    [b \varphi](m) = \varphi(mb).
  \]
  Similarly, if $N$ is an
  $A$-$C$-bimodule, then
  $\Hom_A(M, N)$ is a right $C$-module
  via
  \[
    [\varphi c](m) = \varphi(m)c.
  \]
  So if $M$ is an $A$-$B$-bimodule
  and $N$ an $A$-$C$-bimodule,
  then $\Hom_A(M, N)$ is a $B$-$C$-bimodule.
\end{remark}

\begin{remark}
  Let $M$ be a left $A$-module.
  We write $\End_A(M)$ in place of
  $\Hom_A(M, M)$, and composition gives
  $\End_A(M)$ the structure of a
  $k$-algebra. If $M = A^{\oplus n}$,
  then we can identify
  \[
    \End_A(M) = \Mat_n(A^{\mathrm{opp}}),
  \]
  where the opposite algebra exchanges
  the order of multiplication in the
  original algebra (this is because
  $\End_A(M)$ must respect the action by
  $A$). Then
  $M$ becomes an
  $A$-$(\Mat_n(A))^{\mathrm{opp}}$-bimodule.
\end{remark}

\begin{remark}
  If $M, N$ are two left $A$-modules, then
  $\Hom_A(M, N)$ 
  is an $\End_A(N)$-$\End_A(M)$-bimodule
  (by taking into account compositions).
\end{remark}

\section{Tensor Product of Modules}

\begin{remark}
  Let $A$ be a $k$-algebra,
  $M$ a right $A$-module, and $N$ a
  left $A$-module. We want to produce
  a $k$-module $M \otimes_A N$, which
  will be the \emph{tensor product} of
  $M$ and $N$ over $A$.
\end{remark}

\begin{definition}
  Let $L$ be a $k$-module. We say that
  a map $\varphi : M \times N \to L$
  is \emph{$A$-bilinear} if it is $k$-linear
  in both arguments and satisfies
  \[
    \varphi(ma, n) = \varphi(m, an)
  \]
  for any $a \in A$, $m \in M$, and
  $n \in N$.
\end{definition}

\begin{definition}[Universal property of the tensor product]
  There is an $A$-bilinear map
  \begin{align*}
    M \times N &\longrightarrow M \otimes_A N \\
    (m, n) &\longmapsto m \otimes n
  \end{align*}
  such that for any $A$-bilinear map
  $\varphi : M \times N \to L$, there
  exists a unique $k$-linear map
  $\psi : M \otimes_A N \to L$ such that
  $\varphi(m, n) = \psi(m \otimes n)$.
  As a diagram, this says that
  \begin{center}
  \begin{tikzcd}
    M \times N \ar[rr, "{(m, n) \mapsto m \otimes n}"] \ar[dr, "\varphi", swap] & & M \otimes_A N \ar[dl, "\psi"] \\
    & L
  \end{tikzcd}
  \end{center}
\end{definition}

\begin{exercise}
  If we choose $M \otimes'_A N$ with
  bilinear map $(m, n) \mapsto m \otimes' n$,
  then there exists a unique isomorphism
  $i : M \otimes_A N \to M \otimes'_A N$
  given by $i(m \otimes n) = m \otimes' n$.
\end{exercise}

\begin{corollary}
  Assume $M \otimes_A N$ satisfies
  the universal property. Then
  $\{m \otimes n\}$ span $M \otimes_A N$.
\end{corollary}

\begin{theorem}
  The tensor product $M \otimes_A N$
  exists for all right $A$-modules $M$
  and left $A$-modules $N$.
\end{theorem}

\begin{proof}
  We sketch the proof. First take
  $M$ to be free. Then we can define
  $M \otimes_A N$ as $N^{\oplus I}$,
  where we have
  $(e_i a_i) \otimes n = (a_i n)_{i \in I}$.
  The universal property is easy to check
  for this case, and the general
  case can be done by writing $M$ as
  a quotient of a free module.
\end{proof}

\begin{example}
  If $M, N$ are both free and
  $\{e_i\}_{i \in I}$, $\{f_j\}_{j \in J}$
  are bases of $M, N$, respectively,
  then $M \otimes_A N$ is a free
  $k$-module with basis vectors
  $\{e_i \otimes f_j\}_{i \in I, j \in J}$.
\end{example}

\begin{exercise}
  Let $M = A / I$, where $I$ is a right
  ideal. Show that
  $M \otimes_A N = N / IN$. Find out
  what happens when $N = A / J$, where
  $J$ is a left ideal, what can you say
  about $M \otimes_A N$ in terms of
  $A, I, J$?
\end{exercise}

\begin{prop}
  Assume $B$ is a $k$-algebra and
  $M$ a $B$-$A$-module. Then
  $M \otimes_A N$ is a left $B$-module.
\end{prop}

\begin{proof}
  Define $\varphi_b : M \times N \to M \otimes_A N$
  by $(m, n) \mapsto bm \otimes n$.
  This is bilinear, so by the universal
  property, there exists
  $\psi_b : M \otimes_A N \to M \otimes_A N$
  such that $\psi_b(m \otimes n) = bm \otimes n$, which gives
  the $B$-action.
\end{proof}

\begin{definition}
  Let $L$ be a $B$-module. A map
  $\varphi : M \times N \to L$ is
  called \emph{$B$-$A$-linear} if it
  is $k$-linear in both arguments and
  \[
    \varphi(ma, n) = \varphi(m, an), \quad
    \varphi(bm, n) = b \varphi(m, n)
  \]
  for all $m \in M$, $n \in N$, $b \in B$,
  and $a \in A$.
\end{definition}

\begin{prop}
  The left $B$-module $M \otimes_A N$
  has the following universal property:
  \begin{quote}
    Let $L$ be any left $B$-module
    and $\varphi : M \times N \to L$
    a $B$-$A$-linear map. Then there
    exists a unique $B$-linear map
    $\psi : M \otimes_A N \to L$
    such that $\psi(m \otimes n) = \varphi(m, n)$.
  \end{quote}
\end{prop}

\begin{example}
  Let $A_1, A_2$ be $k$-algebras. Then
  \begin{enumerate}
    \item $A_1 \otimes_k A_2$ has
      the structure of a $k$-algebra via
      \[
        (a_1 \otimes a_2)(b_1 \otimes b_2) = (a_1 b_1) \otimes (a_2 b_2),
      \]
      where $1 \otimes 1$ is a unit element.
    \item Let $M_i$ be a left $A_i$-module
      for $i = 1, 2$. Then
      $M_1 \otimes_k M_2$ is a module
      for $A_1 \otimes_k A_2$.
  \end{enumerate}
\end{example}

\section{Tensor-Hom Adjunction}
\begin{prop}[Tensor-Hom adjunction]
  Let $A, B$ be associative algebras,
  $N$ a $B$-module, $M$ an $A$-module,
  and $L$ an $A$-$B$-bimodule. Then
  \begin{enumerate}
    \item $L \otimes_B N$ is an
      $A$-module;
    \item $\Hom_A(L, M)$ is a
      $B$-module.
  \end{enumerate}
  Moreover, there is a natural
  $k$-linear isomorphism
  \[
    \Hom_A(L \otimes_B N, M) \overset{\cong}{\longrightarrow} \Hom_B(N, \Hom_A(L, M)).
  \]
\end{prop}

\begin{proof}
  By the universal property, there is a
  natural map
  \[
    \Hom_A(L \otimes_B N, M)
    \overset{\cong}{\longrightarrow} \Bilin_{A, B}(L \times N, M).
  \]
  So it suffices to find
  \begin{align*}
    \Hom_B(N, \Hom_A(L, M))
    &\overset{\cong}{\longrightarrow} \Bilin_{A, B}(L \times N, M) \\
    f &\longmapsto \varphi_f.
  \end{align*}
  Construct this map by
  $\psi_f(e, n) = [f(n)](e)$, with
  inverse $h \mapsto \psi(\cdot, h)$
  for $\psi \in \Bilin_{A, B}(L \times N, M)$.
\end{proof}

\begin{example}
  If we have an algebra homomorphism
  $B \to A$, where $A$ is a
  an $A$-$B$-bimodule. One can
  show as an exercise that
  $\Hom_A(A, M)$ is naturally identified
  with $M$ as an $A$-module and $B$-module.
  Thus by the Tensor-Hom adjunction,
  we have a natural isomorphism
  \[
    \Hom_A(A \otimes_B N, M)
    \overset{\cong}{\longrightarrow}
    \Hom_B(N, M).
  \]
\end{example}

\begin{definition}
  The $A$-module $A \otimes_B N$ is
  said to be \emph{induced} from $N$.
\end{definition}

\begin{remark}
  Assume there is
  $\Hom$ from $A \to B$. Then $B$
  is an $A$-$B$-bimodule. Take it as $L$
  in the Tensor-Hom adjunction.
  Note that $B \otimes_B N \cong N$
  as $A$-modules, and we have a natural
  isomorphism
  \[
    \Hom_A(N, M)
    \overset{\cong}{\longrightarrow}
    \Hom_B(N, \Hom_A(B, M)).
  \]
\end{remark}

\begin{definition}
  The $B$-module $\Hom_A(B, M)$ is
  said to be \emph{coinduced} from $M$.
\end{definition}

\end{document}
