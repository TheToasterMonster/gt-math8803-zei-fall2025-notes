\documentclass[12pt, letterpaper, oneside]{book}
\usepackage[margin={0.6in, 0.75in}]{geometry}
\usepackage{microtype}
% \usepackage{kpfonts}
\usepackage{amsmath, amssymb, amsthm}
\usepackage{parskip}
\usepackage[many]{tcolorbox}
\usepackage{footnote}
\usepackage{cancel}
\usepackage{titlesec}
\usepackage{pgffor}
\usepackage[shortlabels, inline]{enumitem}
\usepackage{hyperref}
\usepackage{tikz-cd}

\usepackage[overload]{textcase}

\renewcommand{\chaptername}{Lecture}
\newtheorem{axiom}{Axiom}[chapter]
\newtheorem{theorem}{Theorem}[chapter]
\newtheorem{prop}{Proposition}[chapter]
\newtheorem{corollary}{Corollary}[theorem]
\newtheorem{lemma}{Lemma}[chapter]
\newtheorem{conjecture}{Conjecture}[theorem]
\theoremstyle{definition}
\newtheorem{definition}{Definition}[chapter]
\newtheorem{exercise}{Exercise}[chapter]
\newtheorem{example}{Example}[definition]
\newtheorem*{remark}{Remark}

\tcbset{sharp corners, breakable, enhanced, parbox=false}
\newtcolorbox{mybox}[3][]
{
  colframe = #2!150,
  colback  = #2!5,
  coltitle = #2!0!white,  
  title    = {#3},
  #1,
}

\titleformat{\chapter}[display]
    {\normalfont\huge\bfseries}{\chaptertitlename\ \thechapter}{20pt}{\Huge}
\titlespacing*{\chapter}{0pt}{0pt}{40pt}

\newcommand{\R}{\mathbb{R}}
\newcommand{\N}{\mathbb{N}}
\newcommand{\Z}{\mathbb{Z}}
\newcommand{\C}{\mathbb{C}}
\newcommand{\Q}{\mathbb{Q}}
\newcommand{\F}{\mathbb{F}}
\newcommand{\ZZ}{\mathcal{Z}}
\newcommand{\Mod}[1]{\ {\mathrm{mod}\ #1}}
\newcommand{\Pmod}[1]{\ (\mathrm{mod}\ #1)}

\DeclareMathOperator{\lcm}{lcm}
\DeclareMathOperator{\re}{Re}
\DeclareMathOperator{\im}{Im}
\DeclareMathOperator{\id}{Id}
\DeclareMathOperator{\GL}{GL}
\DeclareMathOperator{\Char}{char}
\DeclareMathOperator{\Mat}{Mat}
\DeclareMathOperator{\Hom}{Hom}
\DeclareMathOperator{\Bilin}{Bilin}
\DeclareMathOperator{\rank}{rank}
\DeclareMathOperator{\tr}{tr}
\DeclareMathOperator{\End}{End}
\DeclareMathOperator{\rad}{rad}
\DeclareMathOperator{\sign}{sign}
\DeclareMathOperator{\refl}{refl}
\DeclareMathOperator{\Ind}{Ind}
\DeclareMathOperator{\Fun}{Fun}
\DeclareMathOperator{\Irr}{Irr}
\DeclareMathOperator{\diag}{diag}
\DeclareMathOperator{\triv}{triv}
\DeclareMathOperator{\Path}{Path}

\title{MATH 8803: Representation Theory}
\author{Frank Qiang\\Instructor: Anton Zeitlin}
\date{Georgia Institute of Technology\\Fall 2025}

\begin{document}
  \maketitle

  \begingroup
  \let\cleardoublepage\clearpage
  \tableofcontents
  \endgroup

  % \foreach \i in {00, 01, 02, 03, 04, ..., 50} {%
  %   \edef\FileName{lectures/lecture\i.tex}%     The % here are necessary to eliminate any
  %   \IfFileExists{\FileName}{%  spurious spaces that may get inserted
  %      \input{\FileName}%       at these points
  %   }
  % }
  \chapter{Aug.~18 --- Historical Perspective}

\section{Origin of Representation Theory}

One motivation for representation theory is
symmetries in physics.
From a mathematical perspective, we consider
\emph{groups} and \emph{algebras} (a vector
space with a bilinear operation).
In this course, we will study two types of
groups:
\begin{enumerate}
  \item \emph{finite groups}, e.g. the symmetric group;
  \item \emph{Lie groups}, e.g. the rotation group.
\end{enumerate}

\begin{definition}
  A \emph{representation} of a group $G$
  is a homomorphism $G \to \End(V)$, where
  $V$ is some finite-dimensional vector space.
\end{definition}

The history of representation theory is as
follows:
\begin{enumerate}
    \item
    In the late 19th century, people
    were interested in \emph{crystallography},
    in particular crystallographic groups
    and their classification. There are
    related objects called \emph{Bieberbach groups} (e.g. $O(n)$ with translations, i.e.
    $\R^n \rtimes O(n)$).

    Sophus Lie discovered \emph{Lie groups}
    in his main manuscript ``Transformation
    groups.'' From Lie groups, one then derives
    \emph{Lie algebras}.

    \item
    In the early 20th century (1905),
    \emph{special relativity} was discovered,
    which involves the \emph{Lorentz group}
    $\SO(1, 3)$ (the transformations preserving
    the form $-t^2 + x^2 + y^2 + z^2$).
    This is a Lie group.

    Around the same time, E. Cartan developed
  the modern theory of \emph{semisimple Lie groups} and \emph{Lie algebras}, and
    H. Weyl studied their representations.

    \item
    In the period 1920--1930, quantum (``matrix'')
    mechanics was discovered. Here
    one has a Hilbert space $\mathcal{H}$ and
    a self-adjoint Hamiltonian (energy)
    operator $H$ on $\mathcal{H}$.
    The symmetry operator $A$ satisfies
    the commutator relation
    $[H, A] = 0$, and if we set $U = e^{iA}$,
    we have $U H U^\dagger = H$.

    \item
    After the discovery of \emph{spin} by W.
    Pauli, E. Wigner realized that spin was
    directly related to the representation
    theory of the universal cover
    $\pi : \SU(2) \to \SO(3)$.

    In the 1960s, there was a ``zoo'' of
    elementary particles. M. Gell-Mann and
    Y. Neeman realized that all of these
    can be described by representations of
    $SU(3)$. The led to the discovery of
    \emph{quarks} and the later
    notion of grand unified theories and
    string theory in the 1970s.

    There are also connections to condensed
    matter theory and quantum information.
\end{enumerate}

This course will cover the following topics:
\begin{enumerate}
  \item basics about associative algebras
    and their representations, finite groups
    and their representations in general,
    the symmetric group and its representations,
    Young tableaux;
  \item Lie groups and Lie algebras;
  \item the structure of semisimple Lie
    algebras;
  \item representations of $\SL(n)$.
\end{enumerate}

\section{Introduction to Lie Groups and Lie Algebras}
In general, groups are complicated, whereas
algebras are less complicated. We begin with
finite groups.

\begin{definition}
  Let $G$ be a finite group and $\F$ a field.
  The \emph{group algebra} $\F G$ is
  \[
    \F G = \left\{
      \sum_g a_g g : a_g \in \F
    \right\}.
  \]
  This forms an algebra over $\F$
  with the obvious multiplication operation.
\end{definition}

\begin{example}
  Consider the rotation group, generated by the
  matrices
  \[
    R_z(\theta) =
    \begin{pmatrix}
      \cos \theta & \sin \theta & 0 \\
      -\sin \theta & \cos \theta & 0 \\
      0 & 0 & 1
    \end{pmatrix}, \quad
    R_x(\phi) =
    \begin{pmatrix}
      1 & 0 & 0 \\
      0 & \cos \phi & -\sin \phi \\
      0 & \sin \phi & \cos \phi
    \end{pmatrix}, \quad
    R_y(\psi) =
    \begin{pmatrix}
      \cos \psi & 0 & -\sin \psi \\
      0 & 1 & 0 \\
      \sin \psi & 0 & \cos \psi
    \end{pmatrix}.
  \]
  Letting $\delta$ be an infinitesimal
  value and using
  a Taylor expansion, we can write
  \begin{align*}
    R_z(\delta \theta)
    &= 1 + \delta \theta \begin{pmatrix}
      0 & 1 & 0 \\
      -1 & 0 & 0 \\
      0 & 0 & 0
    \end{pmatrix} = 1 + \delta \theta M_z, \\
    R_x(\delta \phi)
    &= 1 + \delta \phi \begin{pmatrix}
      0 & 0 & 0 \\
      0 & 0 & -1 \\
      0 & 1 & 0
    \end{pmatrix} = 1 + \delta \phi M_x, \\
    R_y(\delta \psi)
    &= 1 + \delta \psi \begin{pmatrix}
      0 & 0 & -1 \\
      0 & 0 & 0 \\
      1 & 0 & 0
    \end{pmatrix} = 1 + \delta \psi M_y.
  \end{align*}
  We can measure the commutativity of these
  matrices via
  \begin{align*}
    R_x(\delta \phi) R_y(\delta \psi)
    R_x^{-1}(\delta \phi)
    R_y^{-1}(\delta \psi)
    &= (1 + M_x \delta \phi)
    (1 + M_y \delta \psi)
    (1 - M_x \delta \phi)
    (1 - M_y \delta \psi) \\
    &= 1 + \delta \phi \delta \psi (M_x M_y - M_y M_x).
  \end{align*}
\end{example}

\begin{exercise}
  Show that $[M_x, M_y] = -M_z$.
\end{exercise}

\begin{remark}
  Thus we have a vector space spanned by
  $M_x, M_y, M_z$ with an operation
  $[\cdot, \cdot]$ satisfying the identity
  $[M_x, M_y] = -M_z$. Note that this property
  is satisfied by the cross product on $\R^3$.
  The cross product also satisfies the
  following
  \emph{Jacobi identity}:
  \[
    [A, [B, C]] = [[A, B], C] + [B, [A, C]].
  \]
  The above properties define a
  \emph{Lie algebra}.
\end{remark}

\begin{definition}
  Let $\{e_k\}$ be a basis of a Lie algebra
  and $[e_i, e_j] = \sum_k c_{ij}^k e_k$.
  The \emph{universal enveloping algebra}
  of the Lie algebra is the free associative
  algebra on $\{e_k\}$, modulo the
  relations $[e_i, e_j] = \sum_k c_{ij}^k e_k$.
\end{definition}

\begin{remark}
  One way to return to the Lie group from
  the Lie algebra is exponentiation, e.g.
  $R_z(\theta) = e^{\theta M_z}$.
\end{remark}

\section{Algebras and Modules}

Let $k$ be a commutative ring (most of the time
$k = \C$). All rings will be associative and
unital.

\begin{definition}
  A \emph{(associative and unital) $k$-algebra}
  is a unital ring $A$ with a homomorphism
  $i : k \to A$ such that
  $i(r) \cdot a = a \cdot i(r)$, i.e. the
  image of $i$ commutes with $A$.
\end{definition}

\begin{example}
  Any ring is a $\Z$-algebra.
\end{example}

\begin{definition}
  A \emph{homomorphism} of $k$-algebras is a
  $k$-linear homomorphism of unital rings.
\end{definition}

\begin{definition}
  Let $A, B$ be unital rings, and $M$ an
  abelian group. Then
  \begin{enumerate}
    \item a \emph{left $A$-module structure}
      on $M$ is a $\Z$-bilinear map
      $A \times M \to M$, associative in
      the sense that
      \[
        a_1(a_2 m) = (a_1 a_2) m, \quad
        \text{for all } a_1, a_2 \in A,\, m \in M,
      \]
      and such that $1_A m = m$ for all $m \in M$;
    \item a \emph{right $A$-module structure}
      on $M$ is a $\Z$-bilinear map
      $M \times B \to M$, associative in
      the sense that
      \[
        (m b_1) b_2 = m (b_1 b_2), \quad
        \text{for all } b_1, b_2 \in B,\, m \in M,
      \]
      and such that $m 1_B = m$ for all $m \in M$;
    \item an \emph{$A$-$B$-bimodule structure}
      on $M$ is a left $A$-module and
      right $B$-module structure on $M$,
      along with the condition that
      $(am) b = a(mb)$ for all
      $a \in A$, $b \in B$, and $m \in M$.
  \end{enumerate}
\end{definition}

\begin{remark}
In general, an $A$-module will mean a left $A$-module by default.
\end{remark}

\begin{definition}
  Let $M, N$ be left $A$-modules.
  An \emph{$A$-module homomorphism} is a map
  $\varphi : M \to N$ such that
  $\varphi(am) = a \varphi(m)$ for all
  $a \in A$ and $m \in M$.
\end{definition}

\begin{example}
  A ring $A$ is both a left/right
  $A$-module and an $A$-$A$-bimodule
  (the \emph{regular bimodule}).
\end{example}

\begin{definition}
  The \emph{direct sum}
  $\bigoplus_{i \in I} M_i$ of
  left $A$-modules $M_i$ is the collection
  of $(m_i)_{i \in I}$ with finitely
  many nonzero entries, with
  component-wise addition and
  scalar multiplication.
\end{definition}

\begin{example}
  Let $I$ be an index set. Then
  $A^{\oplus I}$ is the
  \emph{coordinate $A$-module}.
\end{example}

\begin{definition}
  A \emph{submodule} of $M$ is a nontrivial
  subgroup closed under addition and invariant
  under the action of $A$.
\end{definition}

\begin{example}
  Submodules of the regular left/right
  $A$-module are the left/right ideals of $A$.
\end{example}

\begin{definition}
  Let $M$ be a left $A$-module and
  $M_0$ a submodule of $M$. The
  \emph{quotient module} $M / M_0$
  is the set of equivalence classes
  $m + M_0$, where the action of $A$ is
  given by $a(m + M_0) = am + M_0$.
\end{definition}

\begin{lemma}
  Let $M, N$ be $A$-modules and $M_0 \subseteq M$
  a submodule. Let
  $\varphi : M \to N$ be $A$-linear such that
  $\varphi(M_0) = \{0\}$. Then there exists a
  unique $A$-linear map $\underline{\varphi} : M / M_0 \to N$
  such that $\varphi = \underline{\varphi} \circ \pi$,
  where $\pi : M \to M / M_0$ is the
  canonical projection.
\end{lemma}

  \chapter{Aug.~20 --- Algebras and Modules}

\section{More on Algebras and Modules}
\begin{definition}
  A \emph{free} module is a module
  which has a basis.
\end{definition}

\begin{example}
  Consider the coordinate module
  $A^{\oplus I}$. Then a basis is given
  by $e_i = \{\delta_{ij}\}_{j \in I}$
    for $i \in I$.
\end{example}

\begin{prop}
  Let $M$ be a left $A$-module. Let
  $I$ be an index set and let $m_i \in M$
  for $i \in I$. Then
  \begin{enumerate}
    \item There exists a unique $A$-linear
      map $A^{\oplus I} \to M$ which sends
      $e_i \mapsto m_i$.
    \item This map is surjective if and only
      if the elements $m_i$ span $M$.
      In particular, every $M$ is isomorphic
      to a quotient of a free module.
    \item This map is an isomorphism
      if and only if $\{m_i\}$
      form a basis of $M$. In particular,
      every coordinate module is a
      free module.
  \end{enumerate}
\end{prop}

\begin{proof}
  Left as an exercise.
\end{proof}

\begin{example}
  Suppose $M$ is spanned by a single
  element $m$. Then $M \cong A / I$,
  where $I$ is the left ideal
  \[
    I = \{a \in A : am = 0\}.
  \]
\end{example}

\begin{example}
  We can now construct the following
  examples of algebras:
  \begin{enumerate}
    \item Let $\Mat_n(A)$ be the
      set of $n \times n$ matrices with
      entries in $A$. If $A$ is a
      $k$-algebra, then $\Mat_n(A)$
      is also a $k$-algebra.
    \item If $G$ is a group, then the
      group algebra $kG$ (for a ring $k$)
      given by
      \[
        kG = \left\{\sum_{g \in G} a_g g : a_g \in k\right\}
      \]
      is a free module with basis identified
      with the elements of $G$.

      The importance of of this
      object is as follows: Let $G$
      be a group and $B$ an algebra.
      Consider the set of maps satisfying
      $1_G \mapsto 1_B$ and respecting
      the group multiplication. This
      set is in bijection with maps
      $kG \to B$ (they extend by linearity).
      If $V$ is a vector space and
      $B = \End(V)$, then this statement
      says that there is a bijection
      between the representations of the
      group $G$
      and the representations of the group
      algebra $kG$.
    \item If $I$ is a two-sided ideal, then
      $A / I$ has a natural algebra
      structure.
    \item If $A_1, A_2$ are $k$-algebras,
      then the direct sum
      $A_1 \oplus A_2$ is again a
      $k$-algebra (with component-wise multiplication).
      One can extend this by induction to
      a finite direct sum, but note that
      we lose
      the multiplicative identity in
      an infinite direct sum (so we
      do not get an algebra in the infinite
      case).
  \end{enumerate}
\end{example}

\section{Module of Homomorphisms}
\begin{definition}
  Let $k$ be a commutative ring and
  $A$ a $k$-algebra. Let $M, N$ be
  left $A$-modules. Denote by
  $\Hom_A(M, N)$ the set of all
  $A$-module homomorphisms $M \to N$.
  Give $\Hom_A(M, N)$ a $k$-module
  structure via
  \[
    [\varphi_1 + \varphi_2](m) = \varphi_1(m) + \varphi_2(m), \quad
    [r \varphi](m) = r \varphi(m)
  \]
  for $\varphi_1, \varphi_2 \in \Hom_A(M, N)$,
  $r \in k$, and $m \in M$.
\end{definition}

\begin{remark}
  Let $L, M, N$ be left $A$-modules. Then
  we can define a $k$-bilinear map
  \begin{align*}
    \Hom_A(M, N) \times \Hom_A(L, M) &\longrightarrow \Hom_A(L, N) \\
    (\varphi, \psi) &\longmapsto \varphi \circ \psi.
  \end{align*}
\end{remark}

\begin{exercise}
  Let $N_2$ be an $A$-module,
  $N_1 \subseteq N_2$ an $A$-submodule,
  and $N_3 = N_2 / N_1$. Let
  $i : N_1 \hookrightarrow N_2$ be the
  inclusion and $\pi : N_2 \to N_3$ the
  projection. Define the maps
  \begin{align*}
    \widetilde{\iota} : \Hom(M, N_1) &\rightarrow \Hom(M, N_2) \\
    \varphi_1 &\longmapsto i \circ \varphi_1 \\
    \widetilde{\pi} : \Hom(M, N_2) &\rightarrow \Hom(M, N_3) \\
    \varphi_2 &\longmapsto \pi \circ \varphi_2.
  \end{align*}
  Then show that $\widetilde{\iota}$
  is injective and $\im \widetilde{\iota} = \ker \widetilde{\pi}$.
\end{exercise}

\begin{remark}
  Let $B$ be a $k$-algebra and $M$ and
  $A$-$B$-bimodule. Then for all
  $A$-modules $N$, we have that
  $\Hom_A(M, N)$
  is a left $B$-module via
  \[
    [b \varphi](m) = \varphi(mb).
  \]
  Similarly, if $N$ is an
  $A$-$C$-bimodule, then
  $\Hom_A(M, N)$ is a right $C$-module
  via
  \[
    [\varphi c](m) = \varphi(m)c.
  \]
  So if $M$ is an $A$-$B$-bimodule
  and $N$ an $A$-$C$-bimodule,
  then $\Hom_A(M, N)$ is a $B$-$C$-bimodule.
\end{remark}

\begin{remark}
  Let $M$ be a left $A$-module.
  We write $\End_A(M)$ in place of
  $\Hom_A(M, M)$, and composition gives
  $\End_A(M)$ the structure of a
  $k$-algebra. If $M = A^{\oplus n}$,
  then we can identify
  \[
    \End_A(M) = \Mat_n(A^{\mathrm{opp}}),
  \]
  where the opposite algebra exchanges
  the order of multiplication in the
  original algebra (this is because
  $\End_A(M)$ must respect the action by
  $A$). Then
  $M$ becomes an
  $A$-$(\Mat_n(A))^{\mathrm{opp}}$-bimodule.
\end{remark}

\begin{remark}
  If $M, N$ are two left $A$-modules, then
  $\Hom_A(M, N)$ 
  is an $\End_A(N)$-$\End_A(M)$-bimodule
  (by taking into account compositions).
\end{remark}

\section{Tensor Product of Modules}

\begin{remark}
  Let $A$ be a $k$-algebra,
  $M$ a right $A$-module, and $N$ a
  left $A$-module. We want to produce
  a $k$-module $M \otimes_A N$, which
  will be the \emph{tensor product} of
  $M$ and $N$ over $A$.
\end{remark}

\begin{definition}
  Let $L$ be a $k$-module. We say that
  a map $\varphi : M \times N \to L$
  is \emph{$A$-bilinear} if it is $k$-linear
  in both arguments and satisfies
  \[
    \varphi(ma, n) = \varphi(m, an)
  \]
  for any $a \in A$, $m \in M$, and
  $n \in N$.
\end{definition}

\begin{definition}[Universal property of the tensor product]
  There is an $A$-bilinear map
  \begin{align*}
    M \times N &\longrightarrow M \otimes_A N \\
    (m, n) &\longmapsto m \otimes n
  \end{align*}
  such that for any $A$-bilinear map
  $\varphi : M \times N \to L$, there
  exists a unique $k$-linear map
  $\psi : M \otimes_A N \to L$ such that
  $\varphi(m, n) = \psi(m \otimes n)$.
  As a diagram, this says that
  \begin{center}
  \begin{tikzcd}
    M \times N \ar[rr, "{(m, n) \mapsto m \otimes n}"] \ar[dr, "\varphi", swap] & & M \otimes_A N \ar[dl, "\psi"] \\
    & L
  \end{tikzcd}
  \end{center}
\end{definition}

\begin{exercise}
  If we choose $M \otimes'_A N$ with
  bilinear map $(m, n) \mapsto m \otimes' n$,
  then there exists a unique isomorphism
  $i : M \otimes_A N \to M \otimes'_A N$
  given by $i(m \otimes n) = m \otimes' n$.
\end{exercise}

\begin{corollary}
  Assume $M \otimes_A N$ satisfies
  the universal property. Then
  $\{m \otimes n\}$ span $M \otimes_A N$.
\end{corollary}

\begin{theorem}
  The tensor product $M \otimes_A N$
  exists for all right $A$-modules $M$
  and left $A$-modules $N$.
\end{theorem}

\begin{proof}
  We sketch the proof. First take
  $M$ to be free. Then we can define
  $M \otimes_A N$ as $N^{\oplus I}$,
  where we have
  $(e_i a_i) \otimes n = (a_i n)_{i \in I}$.
  The universal property is easy to check
  for this case, and the general
  case can be done by writing $M$ as
  a quotient of a free module.
\end{proof}

\begin{example}
  If $M, N$ are both free and
  $\{e_i\}_{i \in I}$, $\{f_j\}_{j \in J}$
  are bases of $M, N$, respectively,
  then $M \otimes_A N$ is a free
  $k$-module with basis vectors
  $\{e_i \otimes f_j\}_{i \in I, j \in J}$.
\end{example}

\begin{exercise}
  Let $M = A / I$, where $I$ is a right
  ideal. Show that
  $M \otimes_A N = N / IN$. Find out
  what happens when $N = A / J$, where
  $J$ is a left ideal, what can you say
  about $M \otimes_A N$ in terms of
  $A, I, J$?
\end{exercise}

\begin{prop}
  Assume $B$ is a $k$-algebra and
  $M$ a $B$-$A$-module. Then
  $M \otimes_A N$ is a left $B$-module.
\end{prop}

\begin{proof}
  Define $\varphi_b : M \times N \to M \otimes_A N$
  by $(m, n) \mapsto bm \otimes n$.
  This is bilinear, so by the universal
  property, there exists
  $\psi_b : M \otimes_A N \to M \otimes_A N$
  such that $\psi_b(m \otimes n) = bm \otimes n$, which gives
  the $B$-action.
\end{proof}

\begin{definition}
  Let $L$ be a $B$-module. A map
  $\varphi : M \times N \to L$ is
  called \emph{$B$-$A$-linear} if it
  is $k$-linear in both arguments and
  \[
    \varphi(ma, n) = \varphi(m, an), \quad
    \varphi(bm, n) = b \varphi(m, n)
  \]
  for all $m \in M$, $n \in N$, $b \in B$,
  and $a \in A$.
\end{definition}

\begin{prop}
  The left $B$-module $M \otimes_A N$
  has the following universal property:
  \begin{quote}
    Let $L$ be any left $B$-module
    and $\varphi : M \times N \to L$
    a $B$-$A$-linear map. Then there
    exists a unique $B$-linear map
    $\psi : M \otimes_A N \to L$
    such that $\psi(m \otimes n) = \varphi(m, n)$.
  \end{quote}
\end{prop}

\begin{example}
  Let $A_1, A_2$ be $k$-algebras. Then
  \begin{enumerate}
    \item $A_1 \otimes_k A_2$ has
      the structure of a $k$-algebra via
      \[
        (a_1 \otimes a_2)(b_1 \otimes b_2) = (a_1 b_1) \otimes (a_2 b_2),
      \]
      where $1 \otimes 1$ is a unit element.
    \item Let $M_i$ be a left $A_i$-module
      for $i = 1, 2$. Then
      $M_1 \otimes_k M_2$ is a module
      for $A_1 \otimes_k A_2$.
  \end{enumerate}
\end{example}

\section{Tensor-Hom Adjunction}
\begin{prop}[Tensor-Hom adjunction]
  Let $A, B$ be associative algebras,
  $N$ a $B$-module, $M$ an $A$-module,
  and $L$ an $A$-$B$-bimodule. Then
  \begin{enumerate}
    \item $L \otimes_B N$ is an
      $A$-module;
    \item $\Hom_A(L, M)$ is a
      $B$-module.
  \end{enumerate}
  Moreover, there is a natural
  $k$-linear isomorphism
  \[
    \Hom_A(L \otimes_B N, M) \overset{\cong}{\longrightarrow} \Hom_B(N, \Hom_A(L, M)).
  \]
\end{prop}

\begin{proof}
  By the universal property, there is a
  natural map
  \[
    \Hom_A(L \otimes_B N, M)
    \overset{\cong}{\longrightarrow} \Bilin_{A, B}(L \times N, M).
  \]
  So it suffices to find
  \begin{align*}
    \Hom_B(N, \Hom_A(L, M))
    &\overset{\cong}{\longrightarrow} \Bilin_{A, B}(L \times N, M) \\
    f &\longmapsto \varphi_f.
  \end{align*}
  Construct this map by
  $\psi_f(e, n) = [f(n)](e)$, with
  inverse $h \mapsto \psi(\cdot, h)$
  for $\psi \in \Bilin_{A, B}(L \times N, M)$.
\end{proof}

\begin{example}
  If we have an algebra homomorphism
  $B \to A$, where $A$ is a
  an $A$-$B$-bimodule. One can
  show as an exercise that
  $\Hom_A(A, M)$ is naturally identified
  with $M$ as an $A$-module and $B$-module.
  Thus by the Tensor-Hom adjunction,
  we have a natural isomorphism
  \[
    \Hom_A(A \otimes_B N, M)
    \overset{\cong}{\longrightarrow}
    \Hom_B(N, M).
  \]
\end{example}

\begin{definition}
  The $A$-module $A \otimes_B N$ is
  said to be \emph{induced} from $N$.
\end{definition}

\begin{remark}
  Assume there is
  $\Hom$ from $A \to B$. Then $B$
  is an $A$-$B$-bimodule. Take it as $L$
  in the Tensor-Hom adjunction.
  Note that $B \otimes_B N \cong N$
  as $A$-modules, and we have a natural
  isomorphism
  \[
    \Hom_A(N, M)
    \overset{\cong}{\longrightarrow}
    \Hom_B(N, \Hom_A(B, M)).
  \]
\end{remark}

\begin{definition}
  The $B$-module $\Hom_A(B, M)$ is
  said to be \emph{coinduced} from $M$.
\end{definition}

  \chapter{Aug.~25 --- Complete Reducibility}

\section{Reducibility of Modules}

\begin{remark}
  Consider an associative algebra
  $A$ over a field $\F$. We proceed
  to study completely reducible
  representations of $A$.
  Let $U$ be an $A$-module.
\end{remark}

\begin{definition}
  An $A$-module $U$ is \emph{irreducible}
  if it only has two distinct submodules
  ($\{0\}$ and $U$).
\end{definition}

\begin{remark}
  With this definition, $\{0\}$ is
  not irreducible.
\end{remark}

\begin{definition}
  An $A$-module $U$ is \emph{completely reducible}
  if for any submodule $U' \subseteq U$,
  there exists an $A$-submodule
  $U''$ such that $U = U' \oplus U''$.
\end{definition}

\begin{exercise}
  Show that any submodule and any quotient
  module of a completely reducible
  $A$-module is also completely reducible.
\end{exercise}

\begin{example}
  Consider $A = \End_\F(U)$. Then $U$ is
  an $A$-module and is irreducible
  (there is a linear operator
  $\alpha : U \to U$ taking
  $u \mapsto v$ for any $u, v \in U$, so
  there are no nontrivial invariant
  subspaces).
\end{example}

\begin{prop}
  Let $U_1, U_2$ be completely reducible
  $A$-modules. Then $U_1 \oplus U_2$ is
  completely reducible.
\end{prop}

\begin{proof}
  Left as an exercise.
\end{proof}

\begin{corollary}
  Let $U$ be a finite-dimensional
  $A$-module. Then the following are
  equivalent:
  \begin{enumerate}
    \item $U$ is completely reducible;
    \item $U$ is isomorphic to a direct
      sum of irreducible submodules.
  \end{enumerate}
\end{corollary}

\begin{exercise}
  Show that every irreducible $A$-module is
  isomorphic to a quotient
  module for a regular module (i.e. one
  isomorphic to $A$).
  In particular, every irreducible module
  over a finite-dimensional associative
  $\F$-algebra is finite-dimensional.
\end{exercise}

\section{Schur's Lemma}
\begin{theorem}[Schur's lemma]
  Let $A$ be an associative $\F$-algebra
  and $U, V$ irreducible $A$-modules.
  Then
  \begin{enumerate}
    \item if $U, V$ are not
      isomorphic, then
      $\Hom_A(U, V) = 0$;
    \item $\End_A(U)$ is a skew
      field (i.e. a division ring).
      Furthermore, if
      $U$ is finite-dimensional
      and $\F$ is algebraically closed, then
      $\dim \End_A(U) = 1$.
  \end{enumerate}
\end{theorem}

\begin{proof}
  $(1)$ Assume we have a nonzero homomorphism
  $\varphi : U \to V$. Then
  $\ker \varphi \subsetneq U$, and
  $\im \varphi \subseteq V$ is
  nontrivial, so by irreducibility
  $\varphi$ must be an isomorphism.

  $(2)$ Let $\varphi \in \End_A(U)$.
  From $(1)$, we know that
  $\varphi$ is an isomorphism, so
  $\varphi$ has an inverse, i.e.
  $\End_A(U)$ is a skew field.
  For the second part, since
  $\F$ is algebraically closed, we can
  find an eigenvalue $z$ for $\varphi$.
  Then $\varphi - z \id_U$ is not
  invertible, so we have
  $\varphi - z \id_U = 0$ by $(1)$.
\end{proof}

\begin{exercise}
  Consider $1, i, j, k$, where
  $i^2 = j^2 = k^2 = -1$ and
  $ij = -ji = k$. The
  \emph{quaternion algebra} over $\R$ is
  given by
  \[
    \mathbb{H}_\R
    = \{q = w + xi + yj + zk : w, x, y, z \in \R\}
  \]
  Note that $\overline{q} = w - xi - yj - zk$
  satisfies
  $q \overline{q} = w^2 + x^2 + y^2 + z^2$,
  so $q^{-1} = \overline{q} / (w^2 + x^2 + y^2 + z^2)$, i.e.
  $\mathbb{H}_\R$ is a skew field.
  Show that
  $\End_{\mathbb{H}_\R}(\mathbb{H}_\R) \cong \mathbb{H}_\R^\mathrm{opp}$.
\end{exercise}

\begin{remark}
  We have an embedding
  $\mathbb{H}_\R \hookrightarrow \Mat_2(\C)$
  given by
  \[
    q \longmapsto
    \begin{pmatrix}
      w + xi & y + zi \\
      -y + zi & w - xi
    \end{pmatrix}.
  \]
  If we replace $\R$ with $\C$, then
  $\mathbb{H}_\C \cong \Mat_2(\C)$,
  which is reducible (consider the sum of
  column spaces).
\end{remark}

\begin{definition}
  Let $U$ be an $A$-module. We say that
  $U$ is \emph{endotrivial} if
  $\End_A(U)$ consists only of scalar maps,
  i.e. maps of the form $z \id$.
\end{definition}

\begin{remark}
  Suppose $\F$ is algebraically closed and
  uncountable (e.g. $\C$), $A$ has
  countable dimension over $\F$, and
  $U$ an irreducible $A$-module. Then
  $U$ is endotrivial.
\end{remark}

\begin{definition}
  Define the \emph{center} of $A$ to be
  \[
    \mathcal{Z}(A) = \{z \in A : za = az \text{ for all } a \in A\}.
  \]
  Note that this is a commutative algebra.
\end{definition}

\begin{exercise}
  Schur's lemma gives a description of
  the center of $A$.
  Let $U$ be an endotrivial $A$-module
  (e.g. a finite-dimensional module over
  if $\F$ is algebraically closed).
  Show that $z \in \mathcal{Z}(A)$ 
  acts as a scalar on $U$. We call the
  algebra homomorphism
  $\mathcal{Z}(A) \to \F$ the
  \emph{central character} of $U$.
\end{exercise}

\section{Completely Reducible Modules}

\begin{remark}
  Consider finite direct sums of
  endotrivial irreducible modules:
  \[
    \bigoplus_{i = 1}^k U_i \otimes M_i,
  \]
  where the $U_i$ are endotrivial
  modules and the $M_i$ are
  vector spaces known as
  \emph{multiplicity spaces}. Note that
  $U_1^{\oplus i} = U_1 \otimes \F^i$.
  The
  $A$-action on the direct sum for
  $a \in A$ is given by
  \[
    a(u_1 \otimes m_1, \dots, u_k \otimes m_k)
    = (au_1 \otimes m_1, \dots, au_k \otimes m_k).
  \]
  We will use Schur's lemma to understand
  homomorphisms between such modules.

  Write $U^j = \bigoplus_{i = 1}^k U_i \otimes M_i^j$
  for $j = 1, 2$. We can produce a linear
  map
  \[
    \bigoplus_{i = 1}^k \Hom_\F(M_i^1, M_i^2)
    \longrightarrow \Hom_A(U^1, U^2)
  \]
  in the following manner:
  For $\underline{\varphi} = (\varphi_1, \dots, \varphi_k) \in \bigoplus_{i = 1}^k \Hom_\F(M_i^1, M_i^2)$,
  we can define
  \[
    \psi_{\underline{\varphi}}\left(
      \sum_{i = 1}^k u_i \otimes m_i^1
    \right)
    = \sum_{i = 1}^k u_i \otimes \varphi_i(m_i^1).
  \]
\end{remark}

\begin{theorem}
  We have the following:
  \begin{enumerate}
    \item The map $\underline{\varphi} \mapsto \psi_{\underline{\varphi}}$
      defines a vector space isomorphism
      \[
        \bigoplus_{i = 1}^k \Hom_\F(M_i^1, M_i^2)
        \overset{\cong}\longrightarrow \Hom_A(U^1, U^2).
      \]
    \item Every $A$-module homomorphism
      $U_1 \to U_2$ sends
      $U_i \otimes M_i^1$ to
      $U_i \otimes M_i^2$ for any $i$.
  \end{enumerate}
\end{theorem}

\begin{proof}
  Left as an exercise (use Schur's lemma).
\end{proof}

\begin{corollary}\label{cor:hom-isomorphism}
  We have the following:
  \begin{enumerate}
    \item there is an isomorphism
      $\Hom_A(U_i, U) \xrightarrow{\cong} M_i$;
    \item there is an isomorphism
      $\bigoplus_{i = 1}^k U_i \otimes \Hom_A(U_i, U) \cong U$
      given by
      \[
        \sum_{i = 1}^k u_i \otimes \varphi_i
        \longmapsto \sum_{i = 1}^k \varphi_i(u_i).
      \]
  \end{enumerate}
\end{corollary}

\begin{prop}\label{prop:submodule-collection}
  For any $A$-submodule $U' \subseteq U$,
  there exists a unique collection of
  determined subspaces $M_i' \subseteq M_i$
  such that
  $U' = \bigoplus_{i = 1}^k U_i \otimes M_i'$
  as submodules of $U$.
\end{prop}

\begin{proof}
  Note that $\Hom_A(U_i, U') \subseteq \Hom_A(U_i, U)$,
  set $M_i' = \Hom_A(U_i, U')$,
  and use Corollary \ref{cor:hom-isomorphism}.
\end{proof}

\begin{theorem}
  Let $U_i$ be irreducible modules for
  $A$ and consider maps
  $\beta_i : A \to \End_\F(U_i)$. Set
  \[
    \beta = \beta_1 \oplus \cdots \oplus \beta_k : A \longrightarrow \bigoplus_{i = 1}^k \End_\F(U_i),
  \]
  where the $U_i$ are pairwise
  non-isomorphic. Then the
  homomorphism $\beta$ is surjective.
\end{theorem}

\begin{proof}
  Replace $A$ by $A /{\ker \beta}$, so that
  $\beta$ is injective. Then
  $\beta$ equips $\bigoplus_{i = 1}^k \End(U_i)$
  with an $A$-bimodule structure, and
  there is a natural isomorphism
  $\End_\F(U_i) \cong U_i \otimes U_i^*$.
  View $U_i$ as the multiplicity space
  for the right $A$-module and $U_i^*$ as
  the multiplicity space for the left
  $A$-module. By Proposition
  \ref{prop:submodule-collection},
  \[
    A = \bigoplus_{i = 1}^k U_i \otimes V_i
  \]
  as a left $A$-module for some
  $V_i \subseteq U_i^*$. Similarly for
  the right $A$-module, we have
  \[
    A = \bigoplus_{i = 1}^k W_i \otimes U_i^*
  \]
  for some $W_i \subseteq U_i$.
  Then we must have
  $U_i \oplus V_i = W_i \oplus U_i^*$,
  so $U_i \cong W_i$ and $V_i \cong U_i^*$
  (the identity $1 \in A$ guarantees that
  no component is zero). Thus
  $\beta$ is surjective.
\end{proof}

\begin{corollary}
  Let $\F$ be algebraically closed and
  $A$ a finite-dimensional $\F$-algebra.
  Then the set of isomorphism classes of
  irreducible $A$-modules is finite
  and non-empty.
\end{corollary}

\begin{proof}
  First this set is nonempty since $A$ is
  nonzero, so it has an
  irreducible subrepresentation.
  To see that it is finite,
  note that for all collections
  $U_1, \dots, U_k$, the map
  $A \to \bigoplus_{i = 1}^k \End_\F(U_i)$
  is surjective, so
  \[
    \dim A \ge \sum_{i = 1}^k (\dim U_i)^2.
  \]
  This proves the desired result, since
  $A$ is finite-dimensional.
\end{proof}

\section{Simple Algebras}

\begin{definition}
  An algebra $A$ is \emph{simple} if
  the only two-sided ideals are
  $\{0\}$ and $A$ (i.e. $A$ is irreducible
  as a bimodule over itself).
\end{definition}

\begin{theorem}
  Let $\F$ be an algebraically closed
  field and $A$ a finite-dimensional
  $\F$-algebra. Then the following are
  equivalent:
  \begin{enumerate}
    \item $A$ is simple;
    \item $A \cong \End_\F(U)$ for
      some finite-dimensional vector space
      $U$.
  \end{enumerate}
\end{theorem}

\begin{proof}
  $(1 \Rightarrow 2)$: The algebra
  $A$ has an irreducible representation
  $U$, i.e. we have a map
  $A \to \End_\F(U)$. Since $A$ is
  simple, this map must have trivial kernel,
  i.e. it is injective.
  We also already
  know that it is surjective, so
  this map is an isomorphism.

  $(2 \Rightarrow 1)$: Assume $I$ is a
  two-sided ideal in $\End_\F(U) \cong U \otimes U^*$
  and view $I \subseteq U \otimes U^*$.
  Show as an exercise
  that we must have $I = \{0\}$.
\end{proof}

\begin{theorem}
  Every finite-dimensional module
  $V$ for $A = \End_\F(U)$ is isomorphic
  to a direct sum of several copies of $U$.
\end{theorem}

  \chapter{Aug.~27 --- Semisimple Algebras}

\section{Semisimple Algebras}

\begin{definition}
  A finite-dimensional $\F$-algebra $A$
  is called \emph{semisimple} if it
  is isomorphic to a direct sum of
  simple algebras.
\end{definition}

\begin{remark}
  If $\F$ is algebraically closed, then
  $A$ is a direct sum of matrix
  algebras, i.e.
  $\bigoplus_{i = 1}^k \End_\F(U_i)$.
\end{remark}

\begin{theorem}\label{thm:semisimple-direct-sum}
  Let $U_1, \dots, U_k$ be
  finite-dimensional vector spaces
  over $\F$. Let $A = \bigoplus_{i = 1}^k \End_\F(U_i)$,
  so that $U_i$ is an irreducible
  $A$-module. Then every finite-dimensional
  $A$-module $V$ is isomorphic to a direct
  sum of several copies of $U_1, \dots, U_k$.
\end{theorem}

\begin{proof}
  Left as an exercise.
\end{proof}

\begin{corollary}
  Let $\F$ be algebraically closed, and
  $A$ be semisimple and finite-dimensional.
  Then
  \begin{enumerate}
    \item The number of isomorphism
      classes of irreducible
      $A$-modules is equal to
      $\dim \mathcal{Z}(A)$.
    \item Different irreducible
      modules have different central
      characters.
  \end{enumerate}
\end{corollary}

\begin{proof}
  $(1)$
  Let $A = \bigoplus_{i = 1}^k \End_\F(U_i)$.
  By Theorem \ref{thm:semisimple-direct-sum},
  the number of irreducible representations
  is $k$. We can also write
  \[
    \mathcal{Z}\left(\bigoplus_{i = 1}^k A_k\right)
    = \bigoplus_{i = 1}^k \mathcal{Z}(A_i),
  \]
  where $A_i = \End_\F(U_i)$.
  Since $\dim \mathcal{Z}(A_i) = 1$, we
  have $\dim \mathcal{Z}(\bigoplus_{i = 1}^k A_k) = k$
  as well.
  
  $(2)$ Use the projections
  $\mathcal{Z} \to \mathcal{Z}(A_i) \to \F$,
  which correspond to the central characteres.
\end{proof}

\section{Characterizations of Semisimple Algebras}
\begin{definition}
  Let $A$ be a finite-dimensional
  algebra. We say that a two-sided
  ideal $I \subseteq A$ is
  \emph{nilpotent} if $I^n = \{0\}$
  for some $n$.
\end{definition}

\begin{exercise}
  If $I, J$ are nilpotent, then
  show that $I + J$ is also nilpotent.
\end{exercise}

\begin{definition}
  The maximal nilpotent ideal of $A$,
  denoted $\rad(A)$, is
  called the \emph{radical} of $A$.
\end{definition}

\begin{theorem}
  Let $\F$ be algebraically closed and
  $A$ a finite-dimensional algebra.
  Then the following are equivalent:
  \begin{enumerate}
    \item $A$ is semisimple;
    \item all finite-dimensional
      representations of $A$ are completely
      reducible;
    \item $\rad(A) = \{0\}$.
  \end{enumerate}
\end{theorem}

\begin{proof}
  $(1 \Rightarrow 2)$ We have already
  shown this.

  $(2 \Rightarrow 3)$ Let $I = \rad(A)$,
  so $I^n = \{0\}$ for some $n \in \N$.
  Let $N$ be a finite-dimensional
  $A$-module. Then
  $I^\ell N$ is an $A$-submodule
  for $\ell = 0, \dots, n$. Since
  $N$ is completely reducible
  and $I^{\ell + 1}N \subseteq I^{\ell} N$, we have
  \[
    I^\ell N = N_\ell \oplus I^{\ell + 1}.
  \]
  Acting on both sides by $I$, we get
  $I N_{\ell} \subseteq I^{\ell + 1}$,
  so $IN_{\ell} = \{0\}$.
  Continuing, we get $IN = 0$, so
  $A = N$.

  $(3 \Rightarrow 1)$ Take $N_1, \dots, N_k$
  to be
  pairwise non-isomorphic irreducible
  $A$-modules. We have an
  epimorphism $A \to \bigoplus_{i = 1}^k \End_\F(N_i)$.
  Let $I$ be the kernel, so
  $I$ acts trivially on every
  irreducible $A$-module. We claim that
  $I$ is nilpotent.
  Take $A$ to be the regular module.
  Take a filtration
  \[
    A = A_0 \supseteq A_1 \supseteq \cdots
    \supseteq A_n = \{0\},
  \]
  where
  $A_i / A_{i + 1}$ is irreducible.
  Now $I$ acts trivially
  on $A_i / A_{i + 1}$, so
  $I A_i \subseteq A_{i + 1}$ for all
  $i$, thus $I^n = \{0\}$.
\end{proof}

\begin{remark}
  Assume $\Char(\F) = 0$.
  Consider the following bilinear
  form on $A$:
  \[
    (a, b)_U = \tr_U(ab),
  \]
  where $U$ is any $A$-module. Note that
  $U$ could be $A$.
\end{remark}

\begin{theorem}
  Let $\Char(\F) = 0$, and let
  $A$ be a finite-dimensional $\F$-algebra.
  Then $A$ is semisimple if and only if
  $(a, b)_A$ is nondegenerate.
\end{theorem}

\begin{proof}
  $(\Rightarrow)$ Assume $A$ is
  semisimple, so
  $A = \bigoplus_{i = 1}^k \End(U_i)$.
  Note that the restriction of
  $(\cdot, \cdot)_A$ to the
  direct summand $\End_\F(U_i)$
  coincides with
  $(\cdot, \cdot)_{\End_\F(U_i)}$.
  Let $E_{j \ell}$ denote the matrix with
  all $0$s except a single $1$ in the
  $(j, \ell)$ entry. Then we can compute
  that
  \[
    (E_{j \ell}, E_{j' \ell'})_{\End_\F(U_i)}
    = \delta_{ej'} \tr_{\End_\F(U_i)}(E_{j\ell'})
    = \delta_{e\ell} \delta_{j \ell'}
    \dim U_i.
  \]
  So if $\{E_{j \ell}\}$ is a basis, then
  $\{(\dim U_i)^{-1} E_{j \ell}\}$
  is the dual basis.
  This is nondenegerate if
  $\Char(\F) = 0$.

  $(\Leftarrow)$ Suppose $(\cdot, \cdot)_A$
  is nondegenerate. If $I$ is a nilpotent
  ideal, then for any $a \in I$
  such that $a^n = 0$. Then $\tr_A(a) = 0$
  for any $a \in I$, so
  $I \in \ker(\cdot, \cdot) = 0$.
  Since $(\cdot, \cdot)$ is nondegenerate,
  we have $I = \{0\}$.
\end{proof}

\section{Double Centralizer Theorem}

\begin{theorem}[Double centralizer theorem]
  Let $V$ be a finite-dimensional vector
  space over $\F$. Let
  $A \subseteq \End_\F(V)$ be a semisimple
  algebra, and
  set $B = \End_A(V)$. Then
  $A = \End_B(V)$.
\end{theorem}

\begin{proof}
  Let $A = \bigoplus_{i = 1}^k \End(U_i)$
  and $V$ be a faithful representation
  of $A$, so $V$ is completely reducible:
  \[
    V \cong \bigoplus_{i = 1}^k U_i \oplus M_i,
  \]
  where the $M_i$ are multiplicity spaces.
  Let $a = (\varphi_1, \dots, \varphi_k) \in A$
  (for $\varphi_i \in \End(U_i)$)
  act on $\End_{\F}(V)$ by
  \[
    (\varphi_1, \dots, \varphi_k)
    \longmapsto
    \sum_{i = 1}^k \varphi_i \otimes \id_{M_i}.
  \]
  Note that the $M_i$ are nonzero
  since $V$ is faithful. Then 
  $B = \bigoplus_{i = 1}^n \End(M_i)$
  embeds into $\End_\F(V)$ via
  \[
    (\psi_1, \dots, \psi_k)
    \longmapsto
    \sum_{i = 1}^k {\id_{U_i}} \otimes \psi_i,
  \]
  which completes the proof.
\end{proof}

\section{Representations of Finite Groups}

\begin{remark}
  Recall that to any group $G$ we can
  associate the group algebra
  $\F G$. For any representation of
  $G$, there is a representation
  of $\F G$ and vice versa.
\end{remark}

\begin{remark}
  Consider the following operations
  with representations. Let $U, V$
  be representations of $G$.
  \begin{enumerate}
    \item the \emph{tensor product} $U \otimes_\F V$, where
      $g(U \otimes V) = (g U \otimes gV)$;
    \item the \emph{dual} $U^*$
      defined by
      $\langle g \alpha, u \rangle = \langle \alpha, g^{-1} u \rangle$
      for $u \in U$, $\alpha \in U^*$,
      $g \in G$;
    \item $\Hom_{\F}(U, V)$, with
      action given by
      $[g \varphi](h) = g[\varphi(g^{-1} u)]$
      for $\varphi \in \Hom_\F(U, V)$.
  \end{enumerate}
\end{remark}

\begin{exercise}
  Show the following:
  \begin{enumerate}
    \item The tensor product of
      representations satisfies
      associativity, distributivity, and
      commutativity.
    \item There is an isomorphism
      of representations
      $U^* \otimes V \to \Hom(U, V)$.
    \item $\Hom_G(U, V) \subseteq \Hom(U, V)$
      coincides with the space
      of $G$-invariant elements.
  \end{enumerate}
\end{exercise}

\begin{remark}
  For the rest of this section,
  assume $\F$ is algebraically closed
  and $\Char \F = 0$.
\end{remark}

\begin{theorem}
  The group algebra $\F G$ is semisimple.
\end{theorem}

\begin{proof}
  It suffices to show that
  $(\cdot, \cdot)_{\F G}$ is nondegenerate.
  Take $g, g' \in G$, and note
  that $gg' : h \mapsto gg' h$, so
  \[
    (g, g')_{\F G}
    = \tr_{\F G}(gg') = \delta_{1, gg'} |G|,
  \]
  which is nondegenerate.
  Moreover,
  the basis $\{g\}$ in $\F G$
  corresponds to the dual basis
  $\{|G|^{-1}g^{-1}\}$.
\end{proof}

\begin{corollary}
  (Let $\F$ be algebraically closed
  and $\Char \F = 0$.)
  \begin{enumerate}
    \item Every finite-dimensional
      representation of $G$ is completely
      reducible.
    \item The number of isomorphism classes
      of irreducible representations
      is equal to the number of conjugacy
      classes of $G$.
    \item If $U_1, \dots, U_k$
      are all of the pairwise
      non-isomorphic irreducible
      representations of $G$, then
      \[
        |G| = \sum_{i = 1}^k (\dim U_i)^2.
      \]
  \end{enumerate}
\end{corollary}

\begin{proof}
  $(1)$ This follows from the semisimplicity
  of $\F G$.

  $(2)$ It suffices to show that
  $\dim \mathcal{Z}(\F G)$ equals the
  number of conjugacy classes of $G$.
  We have
  \[
    \mathcal{Z}(\F G)
    = \left\{
      \sum_{g \in G} a_g g :
      a_g \text{ is constant on conjugacy classes}
    \right\},
  \]
  i.e. we must have
  $a_{hgh^{-1}} = a_g$ for any
  $h \in G$. So the dimension is
  the number of conjugacy classes.
  
  $(3)$ This automatically follows
  from looking at the dimension of
  $\F G$.
\end{proof}

  \chapter{Sept.~3 --- Representations of Finite Groups}

\section{Representations of \texorpdfstring{$S_4$}{S4}}
\begin{remark}
  We will write \emph{irrep} for
  ``irreducible representation.''
\end{remark}

\begin{example}
  Consider the symmetric group
  $S_4$, with $|S_4| = 24$. The conjugacy
  classes of $S_4$ are parametrized by
  partitions of $4$: If we have a partition
  \[
    \lambda = (\lambda_1, \lambda_2, \ldots, \lambda_k), \quad
    \lambda_1 \ge \lambda_2 \ge \cdots \ge \lambda_k \ge 1,
    \quad
    \lambda_1 + \lambda_2 + \cdots + \lambda_k = 4,
  \]
  then the corresponding conjugacy class
  has cycle type $\lambda$. For
  example, the conjugacy classes are given by
  \begin{enumerate}
    \item $\lambda_1 = 4$: $[4]$;
    \item $\lambda_1 = 3$, $\lambda_2 = 1$: $[3, 1]$;
    \item $\lambda_1 = 2$, $\lambda_2 = 2$: $[2, 2]$;
    \item $\lambda_1 = 2$, $\lambda_2 = 1$, $\lambda_3 = 1$: $[2, 1, 1]$;
    \item $\lambda_1 = 1$, $\lambda_2 = 1$, $\lambda_3 = 1$, $\lambda_4 = 1$: $[1, 1, 1, 1]$.
  \end{enumerate}
  In particular, this means that
  $S_4$ has $5$ irreps. We can enumerate
  them as follows:
  \begin{enumerate}
    \item We have the $1$-dimensional representations:
      the trivial representation and the
      sign $\sign_4$.
    \item Let $S_4$ act on $\C^4$
      by permuting the basis vectors.
      The span of
      $(x, x, x, x)$ gives a $1$-dimensional
      subrepresentation, but it has
      a unique $3$-dimensional complement
      $\refl_4$.
    \item We can take a tensor
      product $\refl_4 \otimes \sign_4$,
      which is also $3$-dimensional. One
      can check that this is different
      from $\refl_4$ by looking at the
      determinant.
    \item We have found two $1$-dimensional
      and two $3$-dimensional irreps,
      which account for
      $1 + 1 + 9 + 9 = 20$ of the
      $24$ dimensions. Thus there
      is a missing $2$-dimensional
      representation.

      Note that there is a projection
      $\pi : S_4 \to S_3$ which is a
      homomorphism with kernel
      $\Z / 2\Z \times \Z / 2\Z$.
      Figure this out and find the
      last irrep as an exercise.
  \end{enumerate}
\end{example}

\begin{exercise}
  Let $G$ be a finite abelian group.
  Prove that all irreps of $G$ are
  $1$-dimensional.
\end{exercise}

\section{Characters}

\begin{definition}
  Let $G$ be a group, and let $U$ a
  finite-dimensional representation of $G$.
  The \emph{character} $\chi_U : G \to \F$
  is defined by $\chi_U(g) = \tr_U(g)$.
\end{definition}

\begin{exercise}
  Prove the following:
  \begin{enumerate}
    \item $\chi_U$ is constant on
      conjugacy classes of $G$.
    \item $\chi_{U \oplus V} = \chi_U \oplus \chi_V$.
    \item $\chi_{U \otimes V} = \chi_U \chi_V$.
  \end{enumerate}
\end{exercise}

\begin{remark}
  For the rest of this section,
  assume $G$ is finite and $\F = \C$.
  So we know every representation
  of $G$ is completely reducible.
  Denote by $\C[G]$ the algebra of
  complex-valued functions on $G$, and
  $\C[G]^G$ the subalgebra of
  functions constant on conjugacy
  classes (i.e. the $G$-invariant functions).
  Clearly the character $\chi_U$ lies
  in $\C[G]^G$ for any finite-dimensional
  representation $U$ of $G$.
\end{remark}

\begin{definition}
  Define a Hermitian scalar product
  on $\C[G]^G$ (a priori only on the
  characters) by
  \[
    (\chi_1, \chi_2)
    = \frac{1}{|G|} \sum_{g \in G}
    \chi_1(g) \overline{\chi_2(g)}.
  \]
\end{definition}

\begin{prop}
  Let $U, V$ be finite-dimensional
  representations of $G$. Then
  \[
    (\chi_U, \chi_V) = \dim \Hom_G(U, V).
  \]
\end{prop}

\begin{proof}
  We first note that
  $\chi_{U^*} = \overline{\chi}_U$. To
  see this, observe that
  since $G$ is finite, we have
  $g^n = 1$ for some $n$. In particular,
  the eigenvalues $\lambda_i(g)$ of $g$
  have $|\lambda_i(g)| = 1$. Thus
  $\lambda_i(g^{-1}) = \overline{\lambda_i(g)}$, so we see the result after
  taking traces. Another
  way to see this is the following:
  For a representation $\rho : G \to U$,
  we can make each $\rho(g)$ into a unitary
  operator as follows. Begin with a
  pairing $\langle \cdot, \cdot \rangle_0$
  on $U$ and define
  \[
    \langle v, w \rangle
    = \frac{1}{|G|} \sum_{g \in G} \langle \rho(g) v, \rho(g) w \rangle_0, \quad
    v, w \in U.
  \]
  Then $\rho(g)$ is unitary with
  respect to $\langle \cdot, \cdot \rangle$,
  and we get the result.

  Continuing, we have
  $V \otimes U^* = \Hom_\C(U, V)$, so
  $\chi_{\Hom(U, V)} = \chi_V \overline{\chi}_U$.
  Consider the averaging element
  \[
    \epsilon = |G|^{-1} \sum_{g \in G} g \in \C[G].
  \]
  This is a projector on $G$-invariants
  ($W^G$) in any representation $W$. Thus
  $\tr_W(\epsilon) = \dim W^G$.
  Applying this to $W = \Hom(U, V)$ and noting
  that $\Hom_G(U, V) = \Hom(U, V)^G$, we
  get
  \[
    \dim \Hom_G(U, V)
    = \tr_{\Hom(U, V)}(\epsilon)
    = |G|^{-1} \sum_{g \in G} \chi_{\Hom(U, V)}(g)
    = |G|^{-1} \sum_{g \in G} \chi_V(g) \overline{\chi_U(g)}
    = (\chi_V, \chi_U),
  \]
  which proves the desired claim.
\end{proof}

\begin{corollary}
  The characters of irreps form an
  orthonormal basis in $\C[G]^G$.
\end{corollary}

\begin{proof}
  Schur's lemma implies orthonormality.
  Since the number of irreps equals the
  number of conjugacy classes, the
  characters must form a basis.
\end{proof}

\section{Induced Representations}

\begin{remark}
  In this
  section, we only assume $k$ is a
  commutative ring.

  Let $H \subseteq G$, where
  $H, G$ are finite groups, let
  $kH, kG$ be the
  corresponding group algebras, and let
  $U$ be a representation of $H$.
  Treating $kG$ as a $kG$-$kH$-bimodule,
  we can construct the tensor
  product
  \[
    kG \otimes_{kH} U.
  \]
  Similarly, treating $kG$ as a
  $kH$-$kG$-bimodule, we can construct
  the representation
  \[
    \Hom_{kH}(kG, U).
  \]
  In fact, these two representations are
  isomorphic, we call it the
  \emph{induced representation}, denoted
  $\Ind_H^G U$.
\end{remark}

\begin{prop}
  There is a natural
  isomorphism
  $kG \otimes_{kH} U \cong \Hom_{kH}(kG, U)$.
\end{prop}

\begin{proof}
  First treat $kG$ as a $kH$-$kG$-bimodule,
  so we can consider $\Hom_{kH}(kG, kH)$
  since $kG$, $kH$ are both left
  $kH$-modules. So for any element
  $\varphi : kG \to kH$, we have
  \[
    \varphi(hg) = h \varphi(g), \quad
    h \in H, g \in G.
  \]
  with a left $G$-action and
  right $H$-action given by
  \[
    [g \varphi](g')
    = \varphi(g' g) \quad
    \text{and} \quad
    [\varphi h](g') = \varphi(h g').
  \]
  Note that $kG$ is a free left $kH$-module
  with basis given by the orbits of $H$.
  Show as an exercise that
  \begin{align*}
  \Hom_{kH}(kG, kH)
  \otimes_{kH} U
  &\overset{\cong}{\longrightarrow}
  \Hom_{kH}(kG, U) \\
  \alpha \otimes u
  &\longmapsto
  (x \mapsto \alpha(x) u)
  \end{align*}
  is an isomorphism. From here it suffices
  to show that
  \[
    kG \overset{\cong}{\longrightarrow}
    \Hom_{kH}(kG, kH)
  \]
  as $kG$-$kH$-bimodules. Define this map
  via $g \mapsto \varphi_g \in \Hom_{kH}(kG, kH)$,
  where
  \[
    \varphi_g(g') =
    \begin{cases}
      g' g & \text{if } g' g \in H, \\
      0 & \text{otherwise}.
    \end{cases}
  \]
  We need to show that $\varphi$ is
  $H$-equivariant, $G$-equivariant, and
  an isomorphism of $k$-modules.

  To see $H$-equivariance, note that
  $\varphi_{gh}(g')$ and $\varphi_g(g') h$
  are nonzero and equal if and only if
  $g g' \in H$. For the
  $G$-equivariance, note that
  $\varphi_{g_1 g}(g')$ and
  $[g_1 \varphi_g](g')$ are given by
  \begin{align*}
    \varphi_{g_1 g} (g')
    = g' g_1 g
    \quad &\text{if } g' g_1 g \in H, \\
    [g_1 \varphi_g](g')
    = g' g_1 g
    \quad &\text{if } g' g_1 g \in H
  \end{align*}
  and zero otherwise, so they coincide.
  To prove that $\varphi$ is an
  isomorphism of $k$-modules, we need to
  check that the $\varphi_g$ form a
  basis in $\Hom_{kH}(kG, kH)$.
  Let $g_1, \dots, g_\ell$ be representatives
  of the left $H$-orbits in $G$. Then we
  claim that the following map is an
  isomorphism of $k$-modules:
  \begin{align*}
    \Hom_{kH}(kG, kH)
    &\overset{\cong}{\longrightarrow}
    (kH)^{\oplus \ell} \\
    \varphi \mapsto \{\varphi(g_i)\}_{i = 1}^\ell.
  \end{align*}
  This follows since for any $g \in G$ and
  $i \in \{1, \dots, \ell\}$,
  there is a unique element $h \in H$
  such that $h g_i = g^{-1}$, so
  $\varphi_g$ is sent to the corresponding
  summand.
\end{proof}

\begin{corollary}
  Let $U$ be a representation of
  $H$ and $V$ a representation of $G$. Then
  \begin{enumerate}
    \item $\Hom_G(\Ind_H^G(U), V) \cong \Hom_H(U, V)$;
    \item $\Hom_G(V, \Ind_H^G(U)) \cong \Hom_H(V, U)$.
  \end{enumerate}
\end{corollary}

\begin{proof}
  This follows from the Tensor-Hom
  adjunction, check it as an exercise.
\end{proof}

  \chapter{Sept.~8 --- Representations of \texorpdfstring{$S_n$}{Sn}}

\section{Motivation for Studying \texorpdfstring{$S_n$}{Sn} and Summary}

\begin{remark}
  The finite \emph{simple} groups
  (those with no nontrivial normal subgroups)
  are classified as follows:
  \begin{enumerate}
    \item abelian groups: cyclic groups of
      finite order;
    \item alternating groups:
      $U_n \subseteq S_n$ (the subgroup of
      even permutations) for $n \ge 5$;
    \item $26$ exceptional finite simple
      groups;
    \item finite simple groups of
      \emph{Lie type} (analogues of
      Lie groups for finite fields).
  \end{enumerate}
  The final parts of the classification
  were done by
  Gorenstein (1960--1980s) and
  Aschbacher-Smith (2004).
\end{remark}
  
\begin{remark}
  We study $S_n$ because it is
  easier to work with than directly
  studying $U_n$, and we can recover
  representations of $U_n$
  from those of $S_n$ via
  Frobenius reciprocity.
\end{remark}

\begin{remark}
  We have previously seen the following
  using our abstract theory:
  \begin{enumerate}
    \item Representations of $S_n$
      are the same as representations of
      $\C S_n$.
    \item The algebra $\C S_n$ is
      semisimple:
      $\C S_n \cong \bigoplus_V \End_\C(V)$,
      where $V$ runs over the isomorphism
      classes of irreps of $S_n$.
    \item The number of irreps of $S_n$
      (up to isomorphism) coincides with
      the number of conjugacy classes.
  \end{enumerate}
\end{remark}

\begin{remark}
  In the case of $S_n$, the conjugacy
  classes are enumerated by
  partitions of $n$:
  \[
    (n_1, n_2, \dots, n_k), \quad
    n_1 \ge n_2 \ge \cdots \ge n_k.
  \]
  We can write repeated parts via
  $(m_1^{d_1}, \dots, m_e^{d_e})$, where
  $m_1 > m_2 > \cdots > m_e$. So for
  $S_6$, we have
  \[
    (2, 2, 1, 1) \longleftrightarrow
    (2^2, 1^2).
  \]
\end{remark}

\section{The Inductive Approach: Background}

\begin{remark}
  We will follow the \emph{inductive approach},
  due to Okounkov-Vershik. Consider
  the inclusions
  \[
    \{1\} = S_1 \subseteq S_2
    \subseteq \cdots \subseteq S_{n - 1} \subseteq S_n.
  \]
  Note that if $H \subseteq G$ are finite groups,
  then an irrep of $\C G$ decomposes
  into irreps of $\C H$.

  In general, if $B \subseteq A$ are
  finite-dimensional associative algebras
  and $\tau : B \to A$ is a homomorphism,
  then any $A$-module is also a $B$-module
  by the homomorphism $\tau$.
  We have isomorphisms
  \begin{align*}
    A &\overset{\cong}{\longrightarrow} \bigoplus_{V \in \Irr(A)} \End_\C(V), \\
    B &\overset{\cong}{\longrightarrow} \bigoplus_{U \in \Irr(B)} \End_\C(U).
  \end{align*}
  Let $M_{V, U} = \Hom_B(U, V)$ be
  multiplicity spaces. Then there is a
  $B$-linear isomorphism
  \begin{align*}
    \bigoplus_i U_i \otimes M_{V, U_i}
    &\overset{\cong}{\longrightarrow} V \\
    \sum_i u_i \otimes \varphi_i
    &\longmapsto \sum_i \varphi_i(u_i).
  \end{align*}
  We can compute $M_{V, U}$ from
  an algebraic perspective.
\end{remark}

\begin{definition}
  Define the \emph{centralizer}
  of $B$ in $A$ to be
  \[
    \mathcal{Z}_B(A)
    = \{a \in A : a \tau(b) = \tau(b) a \text{ for all } b \in B\}.
  \]
\end{definition}

\begin{exercise}
  Prove the following:
  \begin{enumerate}
    \item $\ZZ_A(A) = \ZZ(A)$.
    \item $\ZZ_B(A)$ is a subalgebra of $A$.
  \end{enumerate}
\end{exercise}

\begin{lemma}
  There is an isomorphism
  $\ZZ_B(A) \cong \bigoplus_{U, V} \End(M_{V, U})$,
  with $U, V$ such that $M_{V, U} \ne 0$.
\end{lemma}

\begin{proof}
  We have the isomorphism
  \[
    A \overset{\cong}{\longrightarrow}
    \bigoplus_V \End(V),
  \]
  and we can view $\tau : B \to A$ as
  $(\tau_V)_{V \in \Irr(A)}$, where
  $\tau_V : B \to \End(V)$. Similarly,
  we can
  view an element $a \in A$ as $(a_V) \in \bigoplus_V \End(V)$.
  Then $a \in \ZZ_B(A)$ if and only if
  $a_V \in \ZZ_B(\End(V))$ for all $V$, so
  \[
    \ZZ_B(A)
    = \bigoplus_V \ZZ_B(\End(V)).
  \]
  Then
  $\ZZ_B(\End(V)) \cong \End_B(V) \cong \bigoplus_U \End(M_{V, U})$,
  which completes the proof.
\end{proof}

\begin{remark}
  Show that the following actions of
  $\ZZ_B(A)$ on
  $\End(M_{V, U}) = \Hom_B(U, V)$
  are the same:
  \begin{enumerate}
    \item $\End_B(V)$ acts on
      $\Hom_B(U, V)$ via
      \begin{align*}
        \End_B(V) \times \Hom_B(U, V)
        &\longrightarrow \Hom_B(U, V) \\
        (\alpha, \varphi) &\longmapsto \alpha \circ \varphi;
      \end{align*}
    \item for $z \in \ZZ_B(A)$,
      $\varphi \in \Hom_B(U, V)$,
      we can define $z \varphi \in \Hom_B(U, V)$
      by
      \[
        [z \varphi](u) = z \varphi(u),
      \]
      where the right-hand side is
      the $A$-action on $V$.
  \end{enumerate}
\end{remark}

\begin{corollary}
  The following conditions are equivalent:
  \begin{enumerate}
    \item for all $U \in \Irr(B)$
      and $V \in \Irr(A)$, we have
      $\dim \Hom_B(U, V) \le 1$;
    \item $\ZZ_B(A)$ is commutative.
  \end{enumerate}
\end{corollary}

\begin{proof}
  $\ZZ_B(A) = \bigoplus_{U, V} \End(M_{V, U})$ is
  commutative if and only if
  $\End(M_{V, U})$ has dimension
  $1$ or $0$.
\end{proof}

\begin{example}
  Let $A = \Mat_4(\C) \oplus \Mat_3(\C)$
  and $B = \Mat_2(\C) \oplus \C^{\oplus 2}$.
  Define $\tau : B \to A$ by
  \[
    \tau(x_1, x_2, x_3)
    = (\diag(x_1, x_2, x_2), \diag(x_1, x_3)), \quad
    x_1 \in \Mat_2(\C), x_2, x_3 \in \C.
  \]
  We have $B$-modules
  $U_1, U_2, U_3$ of dimensions
  $2, 1, 1$ and $A$-modules
  $V_1, V_2$ of dimensions $4, 3$. Note
  that $M_{V_1, U_2}$ is $2$-dimensional,
  and $M_{V_1, U_1}$, $M_{V_2, U_1}$,
  $M_{V_2, U_3}$ are $1$-dimensional.
  So far, we have
  \[
    \ZZ_B(A) \cong \Mat_2(\C) \oplus \C^{\oplus 3}.
  \]
  To verify this directly, we know that
  $\ZZ_B(A)$ consists of pairs
  $(y_1, y_2) \in \Mat_4(\C) \oplus \Mat_3(\C)$
  such that $y_1$ commutes with
  $\diag(x_1, x_2, x_2)$ and $y_2$
  commutes with $\diag(x_1, x_3)$. So
  \[
    y_1 =
    \begin{pmatrix}
      a & 0 & 0 & 0 \\
      0 & a & 0 & 0 \\
      0 & 0 & b & c \\
      0 & 0 & d & e
    \end{pmatrix}, \quad
    y_2 =
    \begin{pmatrix}
      f & 0 & 0 \\
      0 & f & 0 \\
      0 & 0 & g
    \end{pmatrix}.
  \]
  So $\ZZ_B(A)$ is parametrized by the
  $2 \times 2$ matrix and the $3$ scalars
  $a, f, g$.
\end{example}

\section{The Inductive Approach: Representations of \texorpdfstring{$S_n$}{Sn}}

\begin{remark}
  Let $S_m \subseteq S_n$ for
  $m < n$, and let
  $\ZZ_m(n)$ be the corresponding
  centralizer for group algebras.
\end{remark}

\begin{lemma}
  Let $H \subseteq G$ be finite groups.
  Then $\Z_{\C H}(\C G) \subseteq \C G$
  consists of elements of the form
  $\sum_{g \in G} a_g g$ such that
  $a_{h g h^{-1}} = a_g$ for all
  $h \in H$. In particular,
  $\ZZ_{\C H}(\C G)$ has a basis indexed
  by the $H$-conjugacy classes in $G$,
  given by (for a conjugacy class $C$)
  \[
    C \longmapsto b_C =
    \sum_{g \in C} g \in \ZZ_{\C H}(\C G).
  \]
\end{lemma}

\begin{example}
  Note that for $\C S_m \subseteq \C S_n$,
  conjugation permutes the first $m$
  elements. For example, for
  $S_3 \subseteq S_6$, we can
  write a conjugacy class as
  $({*}\ {*}\ 4)(5\ {*})(6)$, which
  contains elements like
  $(1\ 2\ 4)(5\ 3)$ and
  $(2\ 3\ 4)(5\ 1)$.
  For $m = n - 1$, consider the conjugacy
  class $({*}\ n)$, which consists of
  \[
    (1\ n), \quad (2\ n), \quad \ldots, \quad (n - 1\ n).
  \]
  Then the basis element $b_{({*}\, n)}$
  (called the $n$th \emph{Jucys-Murphy element}) is given by
  \[
    b_{({*}\, n)} = \sum_{i = 1}^{n - 1} (i\ n).
  \]
\end{example}

\begin{remark}
  We will now determine
  algebra generators of
  $\ZZ_m(n)$. It contains
  \begin{enumerate}
    \item $\ZZ_m(m)$: the center of
      $\C S_m$;
    \item $S_{[m + 1, n]}$: the subgroup
      of $S_n$ containing permutations
      fixing $1, \dots, m$;
    \item $J_k = \sum_{i = 1}^{k - 1} (i\ k)$
      for $k = m + 1, \dots, n$.
  \end{enumerate}
  Note that $J_{m + 1}, \dots, J_n$
  pairwise commute (check this as
  an exercise).
\end{remark}

\begin{theorem}
  The algebra $\ZZ_m(n)$ is generated by
  the subalgebras $\ZZ_m(m)$,
  $\C S_{[m + 1, n]}$, and the elements
  $J_{m + 1}, \dots, J_n$.
\end{theorem}

  \chapter{Sept.~10 --- Representations of \texorpdfstring{$S_n$}{Sn}, Part 2}

\section{Properties of \texorpdfstring{$\C S_n$}{CSn}, Continued}

\begin{remark}
  We will now determine
  algebra generators of
  $\ZZ_m(n)$. It contains
  \begin{enumerate}
    \item $\ZZ_m(m)$: the center of
      $\C S_m$;
    \item $S_{[m + 1, n]}$: the subgroup
      of $S_n$ containing permutations
      fixing $1, \dots, m$;
    \item $J_k = \sum_{i = 1}^{k - 1} (i\ k)$
      for $k = m + 1, \dots, n$.
  \end{enumerate}
  Note that $J_{m + 1}, \dots, J_n$
  pairwise commute (check this as
  an exercise).
\end{remark}

\begin{theorem}
  The algebra $\ZZ_m(n)$ is generated by
  the subalgebras $\ZZ_m(m)$,
  $\C S_{[m + 1, n]}$, and the elements
  $J_{m + 1}, \dots, J_n$.
\end{theorem}

\begin{proof}
  Let $C$ be an $S_m$-conjugacy class in
  $S_n$.
  Define $\deg(C)$ to be the number of
  elements in $\{1, \dots, n\}$ which
  are moved by the corresponding
  permutations (for instance,
  $({*}\ n) = ({*}\ {*})$ has degree $2$).
  Note that we either have
  $\deg(C) = 0$ or $\deg(C) \ge 2$.

  Let $A$ be the subalgebra of
  $\ZZ_m(n)$ generated by
  $\ZZ_m(m)$, $\C S_{n-1}$, and
  $J_{m + 1}, \dots, J_n$. We need
  to show that $b_C \in A$ for every $C$.
  Assume it is not true, and pick
  $C$ of minimal degree such that
  $b_C \notin A$. First we show that
  $\deg(C) > 2$. If $\deg(C) = 2$, then
  we have two possibilities:
  \begin{enumerate}
    \item $C = ({*}\ k)$ for $k > m$.
      Then
      \[
        b_{({*}\, k)}
        = \sum_{i = 1}^m (i\ k)
        = J_k - \sum_{i = m + 1}^{k - 1} (i\ k).
      \]
      Then $J_k \in A$ and
      $\sum_{i = m + 1}^{k - 1} (i\ k) \in \C S_{[m + 1, n]} \subseteq A$, so
      we are good.
    \item $C = (k\ \ell)$ for
      $m < k < \ell \le n$. Then
      $b_{(k\, \ell)} \in \C S_{[m + 1, n]} \subseteq A$.
    \item $C = ({*}\ {*})$. Then
      $b_{({*}\, {*})} \in \ZZ_m(m) \subseteq A$.
  \end{enumerate}
  So $\deg(C) > 2$.
  Now assume that $C$ has more than
  $1$ cycle of degree $\ge 2$. Write
  $C = C' C''$, then
  \[
    b_{C'} b_{C''}
    = \alpha b_C + \sum_{C_0, \deg C_0 < \deg C} \alpha_{C_0} b_{C_0}.
  \]
  Since $b_{C'}, b_{C''}, b_{C_0} \in A$
  by minimality of $C$, we also get
  $\alpha b_C \in A$, so
  $b_C \in A$ since $\alpha \ne 0$ (note
  that we may have characteristic
  issues here if we are not working over $\C$).

  So  we may assume $C$ is a
  single cycle. Pick a cycle
  $(i_1\ i_2\ \cdots\ i_k) \in S_n$.
  Then if $j \notin \{i_1, \dots, i_k\}$,
  \[
    (i_1\ i_2\ \cdots\ i_k)
    (i_s\ j)
    = (i_1\ i_2\ \cdots\ i_{s - 1}\ j\ i_{s + 1}\ \cdots\ i_k).
  \]
  If $j \in \{i_1, \dots, i_k\}$,
  then $(i_1\ i_2\ \cdots\ i_k)(i_s\ j)$ either
  splits into two cycles or
  reduces the degree by $1$.

  So suppose a cycle in $C$ has
  elements from $\{1, \dots, m\}$
  and $k \in \{m + 1, \dots, n\}$.
  We can assume that $k$ is next to $*$.
  Denote by $C'$ the cycle obtained
  after eliminating $*$. Then
  \[
    b_{C'} b_{({*}\, k)}
    = \alpha b_C + \sum_{C_0} \alpha_{C_0} b_{C_0},
  \]
  where $C_0$ either contains disjoint
  cycles or cycles
  of smaller degree. Thus we get
  $b_{C'}, b_{({*}\, k)}, b_{C_0} \in A$
  by the minimality of $C$, so
  $b_C \in A$ as well.

  Thus we may assume the elements in our
  $1$-cycle $C$ sit in either
  $\{1, \dots, m\}$ or $\{m + 1, \dots, n\}$.
  In the first case, $b_C \in \ZZ_m(m) \subseteq A$, and
  in the second case,
  $b_C \in \C S_{[m + 1, n]} \subseteq A$.
\end{proof}

\begin{corollary}
  We have the following:
  \begin{enumerate}
    \item $\ZZ_{n - 1}(n)$ is
      commutative;
    \item for all $U \in \Irr(\C S_{n - 1})$
      and $V \in \Irr(\C S_{n})$,
      the multiplicity of $U$ in $V$ is
      either $0$ or $1$;
    \item the element $J_n$ acts on
      each irreducible $\C S_{n - 1}$-submodule
      of $V \in \Irr(\C S_n)$
      by a scalar.
  \end{enumerate}
\end{corollary}

\begin{proof}
  $(1)$ $\ZZ_{n - 1}(n)$ is generated
  by $\ZZ(n - 1)$ and $J_n$, which commute.

  $(2)$ This follows from the statement
  about abelian centralizers for algebras.

  $(3)$ This follows from Schur's lemma.
\end{proof}

\begin{example}
  We will determine how $J_n$
  acts on various modules and how they
  decompose:
  \begin{enumerate}
    \item $V = \refl_n$, which is a
      $\C S_n$-module and is given by
      \[
        \refl_n = \{(x_1, \dots, x_n) \in \C^n : x_1 + \dots + x_n = 0\}.
      \]
      As a $\C S_{n - 1}$-module,
      $\refl_n$ decomposes as follows
      \begin{itemize}
        \item $U_1 = \{(x_1, \dots, x_{n - 1}, 0) \in \C^n : x_1 + \dots + x_{n - 1} = 0\}$.
          This is $\refl_{n - 1}$.
        \item $U_0 = \{(-x, \dots, -x, (n - 1)x) \in \C^n\}$.
          This is the trivial representation.
      \end{itemize}
      Note that $J_n = \sum_{i = 1}^{n - 1} (i\ n)$
      acts on $(x_1, \dots, x_n)$ by
      \[
        (x_1, \dots, x_n)
        \longmapsto
        ((n - 2) x_1 + x_n, \dots, (n - 2) x_{n - 2} + x_n, x_1 + \dots + x_n).
      \]
      On $\refl_{n - 1}$, the eigenvalue
      is $n - 2$, and on the trivial
      subrepresentation,
      the eigenvalue is $-1$.
    \item When $n = 4$, there was a
      representation $V$ of dimension $2$,
      given by the pull-back of $\refl_3$
      under the
      projection $S_4 \to S_3$. The kernel
      of the projection is the normal
      subgroup
      \[
        \{e, (1\ 2)(3\ 4), (1\ 3)(2\ 4), (1\ 4)(2\ 3)\},
      \]
      where $S_3$ permutes
      $(1\ 2) (3\ 4)$, $(1\ 3)(2\ 4)$, and
      $(1\ 4)(2\ 3)$. Now
      \[
        J_4 = (1\ 4) + (2\ 4) + (3\ 4),
      \]
      and we are looking for an action
      of $J_4$ on $V$. We can take
      \[
        J_4|_V
        = (2\ 3) + (1\ 3) + (1\ 2),
      \]
      which is an element of $\C S_3$. Note
      that $\refl_3$ is given by
      \[
        \refl_3
        = \{(x_1, x_2, x_3) \in \C^3 : x_1 + x_2 + x_3 = 0\}.
      \]
      When $J_4|_V$ acts on
      $\refl_3$, we get
      $x_1 + x_2 + x_3 = 0$ in every
      coordinate,
      for any $(x_1, x_2, x_3) \in \refl_3$,
      so the eigenvalue in this case is $0$.
  \end{enumerate}
\end{example}

\section{Branching Graphs}

\begin{remark}
  Let $V^n$ be an irrep for $\C S_n$.
  We know that $V^n$ decomposes into a
  direct sum of non-isomorphic
  $\C S_{n - 1}$-modules. These then
  decompose into $\C S_{n - 2}$-modules,
  and so on.
\end{remark}

\begin{definition}
  The \emph{branching graph} is a
  directed graph, where the vertices
  are labeled by isomorphism classes
  of $\C S_n$-modules (for all $n$), and
  the edge $U \to V$ exists if
  $V$ is an irreducible module for
  $\C S_n$ and $U$ is an irreducible
  module for $\C S_{n - 1}$ which
  occurs in the decomposition of $V$.
\end{definition}

\begin{example}
  The following is the branching
  graph up to $S_4$:
  \begin{center}
    \begin{tikzcd}[sep=small]
      \triv_4 & & \refl_4 & \C^2 & \refl_4 \otimes \sign_4 & & \sign_4 \\
              & \triv_3 \arrow[lu] \arrow[ru] & & \refl_3 \arrow[lu] \arrow[u] \arrow[ru] & & \sign_3 \arrow[lu] \arrow[ru] \\
              & & \triv_2 \arrow[lu] \arrow[ru] & & \sign_2 \arrow[lu] \arrow[ru] \\
              & & & \triv_1 \arrow[lu] \arrow[ru]
    \end{tikzcd}
  \end{center}
  Note that there is a left-right
  symmetry in the graph, which
  comes from tensoring with $\sign_n$.
\end{example}

\begin{definition}
  Let $V^m \in \Irr(\C S_m)$
  and $V^n \in \Irr(\C S_n)$ for
  $m < n$. Define
  $\Path(V^m, V^n)$ to be the set of all
  paths from $V^m$ to $V^n$ in the
  branching graph. If $m = 1$, we write
  $\Path(V^n) = \Path(V^1, V^n)$, and
  we denote $\Path_n = \bigsqcup_{V^n \in \Irr(\C S_n)} \Path(V^n)$.
\end{definition}

\begin{remark}
  For $\overline{P} = (V^m \to V^{m + 1} \to \cdots \to V^n) \in \Path(V^m, V^n)$,
  denote by $V^m(\overline{P})$ a copy
  of $V^m$ in $V^n$ according to the path
  $\overline{P}$. Then we can write the
  decomposition of $V^n$ by
  \[
    V^n = \bigoplus_{V^m \in \Irr(\C S_m)}
    \bigoplus_{\overline{P} \in \Path(V^m, V^n)}
    V^m(\overline{P}).
  \]
\end{remark}

\begin{definition}
  Denote by $\varphi_{\overline{P}} : V^m \to V^n$
  the homomorphism sending $V^m$ to
  its copy in $V^n$ according to the
  path $\overline{P}$, which is
  defined uniquely
  up to rescaling, and define
  \[
    w_{\overline{P}}
    = (w_{m + 1}, \dots, w_n)
    \in \C^{n - m}
  \]
  where $w_k$ is the scalar
  by which $J_k$ acts on $V^{k - 1} \subseteq V^k$.
  Call $w_{\overline{P}}$ the
  \emph{weight} of $\overline{P}$.
\end{definition}

\begin{remark}
  Recall that
  $\Hom_{\C S_m}(V^m, V^n)$ is an
  irreducible $\ZZ_m(n)$-module from
  properties of centralizers.
\end{remark}

\begin{lemma}
  We have the following:
  \begin{enumerate}
    \item The elements $\varphi_{\overline{P}}$
      form a basis in
      $\Hom_{\C S_m}(V^m, V^n)$.
    \item Each $\varphi_{\overline{P}}$
      is an eigenvector for $J_k$
      with eigenvalue $w_k$, for each
      $k = m + 1, \dots, n$, where
      \[
        (w_{m + 1}, \dots, w_n) = w_{\overline{P}}.
      \]
  \end{enumerate}
\end{lemma}

\begin{proof}
  $(1)$ We can write
  \begin{align*}
    \Hom_{\C S_m}(V^m, V^n)
    &= \bigoplus_{V'^m \in \Irr(S_m)}
    \bigoplus_{\overline{P} \in \Path(V'^m, V^n)}
    \Hom(V^m, V'^m(\overline{P})) \\
    &= \bigoplus_{\overline{P} \in \Path(V^m, V^n)}
    \Hom(V^m, V^m(\overline{P})),
  \end{align*}
  where the second equality is
  by Schur's lemma. By
  Schur's lemma again,
  $\Hom(V^m, V^m(\overline{P})) \cong \C$.
  Since the $\varphi_{\overline{P}}$
  correspond to these summands, this
  proves $(1)$.

  $(2)$ For any $u \in V^m$, we have
  $[J_k \varphi_{\overline{P}}](u) = J_k [\varphi_{\overline{P}}(u)]$.
  By construction, $V^m(\overline{P})$
  lies in some copy of $V^{k - 1}$ in
  $V^k$ for $k = m + 1, \dots, n$,
  so $J_k \varphi_{\overline{P}} = w_k \varphi_{\overline{P}}$
  implies $(2)$.
\end{proof}

\end{document}
